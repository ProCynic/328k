
\documentclass[12pt,leqno]{article}

\usepackage{amsmath,amsfonts,amssymb,amscd,amsthm,amsbsy,upref}


\textheight=8.5truein
\textwidth=6.0truein
\hoffset=-.5truein
\voffset=-.5truein
\numberwithin{equation}{section}
\pagestyle{headings}
\footskip=36pt


\swapnumbers
\newtheorem{thm}{Theorem}[section]
\newtheorem{hthm}[thm]{*Theorem}
\newtheorem{lem}[thm]{Lemma}
\newtheorem{cor}[thm]{Corollary}
\newtheorem{prop}[thm]{Proposition}
\newtheorem{con}[thm]{Conjecture}
\newtheorem{exer}[thm]{Exercise}
\newtheorem{bpe}[thm]{Blank Paper Exercise}
\newtheorem{apex}[thm]{Applications Exercise}
\newtheorem{ques}[thm]{Question}
\newtheorem{scho}[thm]{Scholium}
\newtheorem*{Exthm}{Example Theorem}
\newtheorem*{Thm}{Theorem}
\newtheorem*{Con}{Conjecture}
\newtheorem*{Axiom}{Axiom}



\theoremstyle{definition}
\newtheorem*{Ex}{Example}
\newtheorem*{Def}{Definition}


\newcommand{\lcm}{\operatorname{lcm}}
\newcommand{\ord}{\operatorname{ord}}
\def\pfrac#1#2{{\left(\frac{#1}{#2}\right)}}


\makeindex

\begin{document}




\thispagestyle{plain}
\begin{flushright}
\large{\textbf{TYPE YOUR NAME HERE \\
HW 7: 2.1-2.5, 2.7-2.9\\
M328K \\
February 9th, 2012 \\}}
\end{flushright}

%\maketitle
\markboth{}{} \setcounter{section}{0} \baselineskip=18pt

\setcounter{tocdepth}{4}


%%%%%%%%This is where you can change the numbering to match the problem you
%%%%%%%%are on.  Set the section to  the current chapter.

\setcounter{section}{2}

%%%%%%%%%Now, set the theorem number to one less than the first theorem in
%%%%%%%%%the assignment.
\setcounter{thm}{0}

%%%Theorem 2.1%%%%%%%%%%%%%%%%%%%%%%%%%%%%%%%%%%%%%%%%%%%%%%%%%%%%%%%%%%%%%%%%%%%
\begin{thm}
If $n$ is a natural number greater than $1$, then there exists a
prime $p$ such that $p|n$.
\end{thm}

\begin{proof}[Proof]
Type your proof here!
\end{proof}

%%%Exercise 2.2%%%%%%%%%%%%%%%%%%%%%%%%%%%%%%%%%%%%%%%%%%%%%%%%%%%%%%%%%%%%%%%%%%%%%

\begin{exer}
Write down the primes less than $100$ without the aid of a
calculator or a table of primes and think about how you decide
whether each number you select is prime or not.
\end{exer}

\begin{proof}[Solution]
Type your solution here!
\end{proof}

%%%Theorem 2.3%%%%%%%%%%%%%%%%%%%%%%%%%%%%%%%%%%%%%%%%%%%%%%%%%%%%%%%%%%%%%%%%%%%%%

\begin{thm}
A natural number $n>1$ is prime if and only if for all primes $p
\leq \sqrt{n}$, $p$ does not divide $n$.
\end{thm}

\begin{proof}[Proof]
Type your proof here!
\end{proof}

%%%Exercise 2.4%%%%%%%%%%%%%%%%%%%%%%%%%%%%%%%%%%%%%%%%%%%%%%%%%%%%%%%%%%%%%%%%%%%%%

\begin{exer}
Use the preceding theorem to verify that $101$ is prime.
\end{exer}

\begin{proof}[Solution]
Type your solution here!
\end{proof}


%%%Exercise 2.5%%%%%%%%%%%%%%%%%%%%%%%%%%%%%%%%%%%%%%%%%%%%%%%%%%%%%%%%%%%%%%%%%%%%%

\begin{exer}[Sieve of Eratosthenes]\index{Sieve of Eratosthenes}
Write down all the natural numbers from $1$ to $100$, perhaps on a
$10 \times 10$ array. Circle the number $2$, the smallest prime.
Cross off all numbers divisible by $2$.  Circle $3$, the next number
that is not crossed out.  Cross off all larger numbers that are
divisible by $3$. Continue to circle the smallest number that is not
crossed out and cross out its multiples. Repeat.  Why are the
circled numbers all the primes less than $100$?
\end{exer}


%%% Note: I recommend printing this and crossing off the multiples of primes by hand. 
%% For the electronic version, you can just list the circled primes. Or, if you want to get fancy, 
% you can use \cancel{2} to cross out multiples. That one, for example, would put a line through 2. 
% The trick is, you have to be in math mode, so it's a little annoying!

\begin{proof}[Solution] $\phantom{fasdf}$ \vskip.01in
\begin{tabular}{cccccccccc}
1 & 2 & 3 & 4 & 5 & 6 & 7 & 8 & 9 & 10\\
11 & 12 & 13 & 14 & 15 &  16 & 17 & 18 & 19 & 20\\
21 & 22 & 23 & 24 & 25 & 26 & 27 & 28 & 29 & 30\\
31 & 32 & 33 & 34 & 35 & 36 & 37 & 38 & 39 & 40\\
41 & 42 & 43 & 44 & 45 & 46 & 47 & 48 & 49 & 50\\
51 & 52 & 53 & 54 & 55 & 56 & 57 & 58 & 59 & 60\\
61 & 62 & 63 & 64 & 65 & 66 & 67 & 68 & 69 & 70\\
71 & 72 & 73 & 74 & 75 & 76 & 77 & 78 & 79 & 80\\
81 & 82 & 83 & 84 & 85 & 86 & 87 & 88 & 89 & 90\\
91 & 92 & 93 & 94 & 95 & 96 & 97 & 98 & 99 & 100\\
\end{tabular}$\phantom{fasdf}$ \vskip.01in
\end{proof}


\setcounter{thm}{6}

%%%Theorem 2.7%%%%%%%%%%%%%%%%%%%%%%%%%%%%%%%%%%%%%%%%%%%%%%%%%%%%%%%%%%%%%%%%%%%%%

\begin{thm}[Fundamental Theorem of Arithmetic-Existence Part)]
Every natural number greater than $1$ is either a prime number or it
can be expressed as a finite product of prime numbers. That is, for
every natural number $n$ greater than $1$, there exist distinct
primes $p_1, p_2, \hdots, p_m$ and natural numbers $r_1, r_2,
\hdots, r_m$ such that \[n = p_1^{r_1}p_2^{r_2}\cdots p_m^{r_m}.\]
\end{thm}

\begin{proof}[Proof]
Type your proof here!
\end{proof}


%%%Lemma 2.8%%%%%%%%%%%%%%%%%%%%%%%%%%%%%%%%%%%%%%%%%%%%%%%%%%%%%%%%%%%%%%%%%%%%%

\begin{lem}
Let $p$ and $q_1, q_2, \hdots, q_n$ all be primes and let $k$ be a
natural number such that~$p k = q_1q_2 \cdots q_n$.  Then $p = q_i$
for some $i$.
\end{lem}

\begin{proof}[Proof]
Type your proof here!
\end{proof}


%%%Theorem 2.9%%%%%%%%%%%%%%%%%%%%%%%%%%%%%%%%%%%%%%%%%%%%%%%%%%%%%%%%%%%%%%%%%%%%%

\begin{thm}[Fundamental Theorem of Arithmetic-Uniqueness part]
Let $n$ be a natural number.  Let $\{p_1, p_2, \hdots, p_m\}$ and
$\{q_1, q_2, \hdots, q_s\}$ be sets of primes with $p_i \neq p_j$ if
$i \neq j$ and $q_i \neq q_j$ if $i \neq j$. Let $\{r_1, r_2,
\hdots, r_m\}$ and $\{t_1, t_2, \hdots, t_s\}$ be sets of natural
numbers such that
\begin{align*}
n & = p_1^{r_1}p_2^{r_2}\cdots p_m^{r_m} \\
  & = q_1^{t_1}q_2^{t_2}\cdots q_s^{t_s}.
\end{align*}
Then $m = s$ and $\{p_1, p_2, \hdots, p_m\} = \{q_1, q_2, \hdots,
q_s\}$. That is, the sets of primes are equal but their elements are
not necessarily listed in the same order; that is, $p_i$ may or may
not equal $q_i$.  Moreover, if~$p_i = q_j$ then $r_i = t_j$.  In
other words, if we express the same natural number as a product of
powers of distinct primes, then the expressions are identical except
for the ordering of the factors.
\end{thm}

\begin{proof}[Proof]
Type your proof here!
\end{proof}

\end{document}
