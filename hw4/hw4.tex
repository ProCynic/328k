
\documentclass[12pt,leqno]{article}

\usepackage{amsmath,amsfonts,amssymb,amscd,amsthm,amsbsy,upref}


\textheight=8.5truein
\textwidth=6.0truein
\hoffset=-.5truein
\voffset=-.5truein
\numberwithin{equation}{section}
\pagestyle{headings}
\footskip=36pt


\swapnumbers
\newtheorem{thm}{Theorem}[section]
\newtheorem{hthm}[thm]{*Theorem}
\newtheorem{lem}[thm]{Lemma}
\newtheorem{cor}[thm]{Corollary}
\newtheorem{prop}[thm]{Proposition}
\newtheorem{con}[thm]{Conjecture}
\newtheorem{exer}[thm]{Exercise}
\newtheorem{bpe}[thm]{Blank Paper Exercise}
\newtheorem{apex}[thm]{Applications Exercise}
\newtheorem{ques}[thm]{Question}
\newtheorem{scho}[thm]{Scholium}
\newtheorem*{Exthm}{Example Theorem}
\newtheorem*{Thm}{Theorem}
\newtheorem*{Con}{Conjecture}
\newtheorem*{Axiom}{Axiom}



\theoremstyle{definition}
\newtheorem*{Ex}{Example}
\newtheorem*{Def}{Definition}


\newcommand{\lcm}{\operatorname{lcm}}
\newcommand{\ord}{\operatorname{ord}}
\def\pfrac#1#2{{\left(\frac{#1}{#2}\right)}}


\makeindex

\begin{document}




\thispagestyle{plain}
\begin{flushright}
\large{\textbf{TYPE YOUR NAME HERE \\
HW 4: 1.26-1.27, 1.29-1.37\\
M328K \\
January 31st, 2012 \\}}
\end{flushright}

%\maketitle
\markboth{}{} \setcounter{section}{0} \baselineskip=18pt

\setcounter{tocdepth}{4}


%%%%%%%%This is where you can change the numbering to match the problem you
%%%%%%%%are on.  Set the section to  the current chapter.

\setcounter{section}{1}

%%%%%%%%%Now, set the theorem number to one less than the first theorem in
%%%%%%%%%the assignment.
\setcounter{thm}{25}

%%%%%%%%%%%%1.26%%%%%%%%%%%%%%%%%%%%%%%%%%%%%%%%%%%%%%%%%%%%%%%%%%%%%
\begin{thm}
Prove the existence part of the Division Algorithm. (\textit{Hint}:
Given $n$ and $m$, how will you define $q$?  Once you choose this
$q$, then how is $r$ chosen?  Then show that $0 \leq r \leq n-1$.)
\end{thm}

\begin{proof}[Proof]
Type your proof here!
\end{proof}

%%%%%%%%%%%%1.27%%%%%%%%%%%%%%%%%%%%%%%%%%%%%%%%%%%%%%%%%%%%%%%%%%%%%
\begin{thm}
Prove the uniqueness part of the Division Algorithm.

(\textit{Hint}: If $nq + r = nq' + r'$, then $nq -nq' = r' - r$. Use
what you know about $r$ and $r'$ as part of your argument that $q =
q'$.)
\end{thm}

\begin{proof}[Proof]
Type your proof here!
\end{proof}

%%%%%%%%%Now, set the theorem number to one less than the first theorem in
%%%%%%%%%the assignment.
\setcounter{thm}{28}

%%%%%%%%%%%%1.29%%%%%%%%%%%%%%%%%%%%%%%%%%%%%%%%%%%%%%%%%%%%%%%%%%%%%
\begin{ques}
Do every two integers have at least one common divisor?
\end{ques}

\begin{proof}[Solution]
Type your solution here!
\end{proof}

%%%%%%%%%%%%1.30%%%%%%%%%%%%%%%%%%%%%%%%%%%%%%%%%%%%%%%%%%%%%%%%%%%%%
\begin{ques}
Can two integers have infinitely many common divisors?
\end{ques}

\begin{proof}[Solution]
Type your solution here!
\end{proof}

%%%%%%%%%%%%1.31%%%%%%%%%%%%%%%%%%%%%%%%%%%%%%%%%%%%%%%%%%%%%%%%%%%%%
\begin{exer}
Find the following greatest common divisors. Which pairs are
relatively prime?
\begin{enumerate}
    \item[(1)] $(36,22)$
    \item[(2)] $(45,-15)$
    \item[(3)] $(-296,-88)$
    \item[(4)] $(0,256)$
    \item[(5)] $(15,28)$
    \item[(6)] $(1,-2436)$
\end{enumerate}
\end{exer}

%%%%%%%%%%%%1.32%%%%%%%%%%%%%%%%%%%%%%%%%%%%%%%%%%%%%%%%%%%%%%%%%%%%%
\begin{thm}
Let $a$, $n$, $b$, $r$, and $k$ be integers. If $a = nb + r$ and
$k|a$ and $k|b$, then $k|r$.
\end{thm}

\begin{proof}[Proof]
Type your proof here!
\end{proof}

%%%%%%%%%%%%1.33%%%%%%%%%%%%%%%%%%%%%%%%%%%%%%%%%%%%%%%%%%%%%%%%%%%%%
\begin{thm}
Let $a$, $b$, $n_1$, and $r_1$ be integers with $a$ and $b$ not both
$0$. If $a = n_1b + r_1$, then $(a, b) = (b , r_1)$.
\end{thm}

\begin{proof}[Proof]
Type your proof here!
\end{proof}

%%%%%%%%%%%%1.34%%%%%%%%%%%%%%%%%%%%%%%%%%%%%%%%%%%%%%%%%%%%%%%%%%%%%
\begin{exer}
 As an illustration of the above theorem, note that
\begin{align*}
51 & = 3 \cdot 15 + 6, \\
15 & = 2 \cdot 6 + 3, \\
6 & = 2 \cdot 3 + 0.
\end{align*}
Use the preceding theorem to show that if $a = 51$ and $b = 15$,
then $(51, 15) = (6, 3) = 3$.
\end{exer}

\begin{proof}[Solution]
Type your solution here!
\end{proof}

%%%%%%%%%%%%1.35%%%%%%%%%%%%%%%%%%%%%%%%%%%%%%%%%%%%%%%%%%%%%%%%%%%%%
\begin{exer}[Euclidean Algorithm]\index{Euclidean Algorithm} Using the
previous theorem and the Division Algorithm successively, devise a
procedure for finding the greatest common divisor of two integers.
\end{exer}

\begin{proof}[Solution]
Type your solution here!
\end{proof}

%%%%%%%%%%%%1.36%%%%%%%%%%%%%%%%%%%%%%%%%%%%%%%%%%%%%%%%%%%%%%%%%%%%%
\begin{exer}
Use the Euclidean Algorithm to find \end{exer}
\begin{enumerate}
    \item[(1)] $(96, 112)$
    \begin{proof}[Solution]
    Type your solution here!
    \end{proof}

    \item[(2)] $(175,24)$
    \begin{proof}[Solution]
    Type your solution here!
    \end{proof}

    \item[(3)] $(0,256)$
    \begin{proof}[Solution]
    Type your solution here!
    \end{proof}

    \item[(4)] $(-288, -166)$
    \begin{proof}[Solution]
    Type your solution here!
    \end{proof}

    \item[(5)] $(1,-2436)$
    \begin{proof}[Solution]
    Type your solution here!
    \end{proof}

\end{enumerate}

%%%%%%%%%%%%1.37%%%%%%%%%%%%%%%%%%%%%%%%%%%%%%%%%%%%%%%%%%%%%%%%%%%%%
\begin{exer}
Find integers $x$ and $y$ such that $175x + 24y = 1$.
\end{exer}

\begin{proof}[Solution]
Type your solution here!
\end{proof}


\end{document}
