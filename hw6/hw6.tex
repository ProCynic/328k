
\documentclass[12pt,leqno]{article}

\usepackage{amsmath,amsfonts,amssymb,amscd,amsthm,amsbsy,upref}


\textheight=8.5truein
\textwidth=6.0truein
\hoffset=-.5truein
\voffset=-.5truein
\numberwithin{equation}{section}
\pagestyle{headings}
\footskip=36pt


\swapnumbers
\newtheorem{thm}{Theorem}[section]
\newtheorem{hthm}[thm]{*Theorem}
\newtheorem{lem}[thm]{Lemma}
\newtheorem{cor}[thm]{Corollary}
\newtheorem{prop}[thm]{Proposition}
\newtheorem{con}[thm]{Conjecture}
\newtheorem{exer}[thm]{Exercise}
\newtheorem{bpe}[thm]{Blank Paper Exercise}
\newtheorem{apex}[thm]{Applications Exercise}
\newtheorem{ques}[thm]{Question}
\newtheorem{scho}[thm]{Scholium}
\newtheorem*{Exthm}{Example Theorem}
\newtheorem*{Thm}{Theorem}
\newtheorem*{Con}{Conjecture}
\newtheorem*{Axiom}{Axiom}



\theoremstyle{definition}
\newtheorem*{Ex}{Example}
\newtheorem*{Def}{Definition}


\newcommand{\lcm}{\operatorname{lcm}}
\newcommand{\ord}{\operatorname{ord}}
\def\pfrac#1#2{{\left(\frac{#1}{#2}\right)}}

\newcommand{\card}[1]{\left| #1 \right|}
\newcommand{\pair}[2]{\left\langle #1, #2 \right\rangle}


\makeindex

\begin{document}




\thispagestyle{plain}
\begin{flushright}
\large{\textbf{Geoffrey Parker - grp352 \\
HW 6: 1.38-1.40, 1.48, 1.50, 1.51, 1.53, 1.54\\
M328K \\
February 7th, 2012 \\}}
\end{flushright}

%\maketitle
\markboth{}{} \setcounter{section}{0} \baselineskip=18pt

\setcounter{tocdepth}{4}


%%%%%%%%This is where you can change the numbering to match the problem you
%%%%%%%%are on.  Set the section to  the current chapter.

\setcounter{section}{1}

%%%%%%%%%Now, set the theorem number to one less than the first theorem in
%%%%%%%%%the assignment.

\setcounter{thm}{37}
%%%%%%%%%%%%1.38%%%%%%%%%%%%%%%%%%%%%%%%%%%%%%%%%%%%%%%%%%%%%%%%%%%%%
\begin{thm}
Let $a$ and $b$ be integers. If $(a, b) = 1$, then there exist
integers $x$ and $y$ such that $ax + by = 1$.
\end{thm}

\begin{proof}[Proof]
Let $a$ and $b$ be integers with $(a, b) = 1$.  We will show that there exist integers $x$ and $y$ such that $ax + by = 1$.  First, let $S$ be the set of natural numbers of the form $ax + by$ for any integers $x$ and $y$.  Now we will show that $S$ is not empty.  Because $(a, b) = 1$, $a$ and $b$ cannot both be $0$, so assume that $\card{a} > 0$.  If $a$ is positive, then $a1 + b0$ will be in $S$, and if $a$ is negative, then $a(-1) + b0$ will be an element of $S$.\\

Since $S$ is a non-empty subset of the natural numbers, then by the Well Ordering Axiom $S$ has a least element, which we will call $k$.  Since $k$ is an element of $S$, it can be expressed as $k = ax_0 + bx_0$ for some integers $x_0$ and $y_0$.  Then by the Division Algorithm there exist integers $q$ and $r$ such that $a = qk + r$ and $0 \leq r < k$.  By substitution, this gives us $a = q(ax_0 + by_0) + r$, so $r = a - q(ax_0 + by_0) = a(1-qx_0) + b(y_0)$.  Now if we let $x_1 = 1-qx_0$ and $y_1 = y_0$, then we see that $r = ax_1 + by_1$.  Assume by way of contradiction that $r > 0$.  In this case, $r$ is a natural number, and thus an element of the set $S$.  However $r < k$, which is the least element of the set $S$, so we have a contradiction.  Because $0 \leq r$ and not $r > 0$, $r$ must be $0$.  Therefore $a = qk$, and so $k \mid a$.  By similar logic, we can also establish that $k \mid b$.  Now we have that $k \mid a$ and $k \mid b$, so $k$ is a common divisor of $a$ and $b$.  However, because $(a, b) = 1$, it must be that $k \leq 1$.  And because $k$ is an element of $S$, which is a subset of the natural numbers, $k \geq 1$.  Therefore $k = ax_0 + by_0 = 1$.
\end{proof}

\pagebreak
%%%%%%%%%%%%1.39%%%%%%%%%%%%%%%%%%%%%%%%%%%%%%%%%%%%%%%%%%%%%%%%%%%%%
\begin{thm}
Let $a$ and $b$ be integers. If there exist integers $x$ and $y$
with $ax + by = 1$, then $(a, b) = 1$.
\end{thm}

\begin{proof}[Proof]
Let $a$, $b$, $x$, and $y$ with $ax + by = 1$.  We will show that $(a, b) = 1$.  First, let $d = (a, b)$.  Since $d \mid a$ and $d \mid b$, then by theorem 1.6 $d \mid ax$ and $d \mid ay$.  So by theorem 1.1, $d \mid ax + by$, meaning $d \mid 1$.  Because $d = (a, b)$, $d$ cannot be $0$.  Therefore $d = (a, b) = 1$.
\end{proof}

%%%%%%%%%%%%1.40%%%%%%%%%%%%%%%%%%%%%%%%%%%%%%%%%%%%%%%%%%%%%%%%%%%%%
\begin{thm}
For any integers $a$ and $b$ not both $0$, there are integers $x$
and $y$ such that \[ax + by = (a, b).\]
\end{thm}

\begin{proof}[Proof]
Let $a$ and $b$ be integers not both $0$.  First, we will redifine the Euclidean Algorithm as a collection of sequences, with a couple of extensions.  Call it the Extended Euclidean Algorithm.  Let $i$, $j$, $q$, $r$, $x$, and $y$ be sequences of integers, defined as follows: $i_k = j_{k-1}$, $j_k = r_{k-1}$, use the division algorithm to find $q_k$ and $r_k$ such that $i_k = j_kq_k + r_k$, $x_k = x_{k-2} - q_kx_{k-1}$, and $y_k = y_{k-2} - q_ky_{k-1}$.  Now take these initial values: $i_2 = a$, $j_2 = b$, $x_0 = 1$, $y_0 = 0$, $x_1 = 0$, and $y_1 = 1$.  In effect, by filling out these sequences until you find a $k$ such that $r_k = 0$, you are performing the Euclidean Algorithm.  In addition, we will use induction to prove that for any $k \geq 2$, $r_k = ax_k + by_k$.\\

As a base case, take $k = 2$ and $k = 3$.  This gives us:
\begin{align*}
i_2 &= j_2q_2 + r_2\\
r_2 &= i_2 - q_2j_2\\
r_2 &= a - q_2b\\
r_2 &= (ax_0 + by_0) - q_2(ax_1 + by_1)\\
r_2 &= a(x_0 - q_2x_1) + b(y_0 + q_2y_1)\\
r_2 &= ax_2 + by_2\\
\\
i_3 &= j_2\\
j_3 &= r_2\\
i_3 &= j_3q_3 + r_3\\
r_3 &= i_3 - q_3j_3\\
r_3 &= b - q_3r_2\\
r_3 &= (ax_1 + by_1) - q_3(ax_2 + by_2)\\
r_3 &= a(x_1 - q_3x_2) + b(y_1 - q_3y_2)\\
r_3 &= ax_3 + by_3\\
\end{align*}

Our induction hypothesis is that there exists some integer $N \geq 3$ such that $r_N = ax_N + by_N$.  We must now show that $r_{N+1} = ax_{N+1} + by_{N+1}$.
\begin{align*}
r_N &= ax_N + by_N\\
i_{N+1} &= j_N\\
j_{N+1} &= r_N\\
i_{N+1} &= q_{N+1}j_{N+1} + r_{N+1}\\
r_{N+1} &= i_{N+1} - q_{N+1}j_{N+1}\\
r_{N+1} &= r_{N-1} - q_{N+1}r_N\\
r_{N+1} &= (ax_{N-1} + by_{N-1}) - q_{N+1}(ax_N + by_N)\\
r_{N+1} &= a(x_{N-1} - q_{N+1}x_N) + b(y_{N-1} - q_{N+1}y_N)\\
r_{N+1} &= ax_{N+1} + by_{N+1}
\end{align*}

Now suppose we let $M$ be an integer such that $r_M = 0$.  Since this is the Euclidean Algorithm such an $M$ must exist, and this means that $(a, b) = j_M = r_{M-1} = ax_{M-1} + by_{M-1}$.
\end{proof}



\setcounter{thm}{47}

%%%%%%%%%%%%1.48%%%%%%%%%%%%%%%%%%%%%%%%%%%%%%%%%%%%%%%%%%%%%%%%%%%%%
\begin{thm}
Given integers $a$, $b$, and $c$ with $a$ and $b$ not both 0, there
exist integers $x$ and $y$ that satisfy the equation~$ax + by = c$
if and only if $(a, b)|c$.
\end{thm}

\begin{proof}[Proof]
Let $a$, $b$, and $c$ be integers with $a$ and $b$ not both 0.\\ \\
First assume $(a, b)|c$.  Then, by theorem 1.40, there exist integers $x_0$ and $y_0$ that satisfy $ax_0 + by_0 = (a, b)$.  And by the definition of divides, $c = (a, b)k$ for some integer $k$.  So $c = k(ax_0 + by_0) = ax_0k + by_0k$.  Let the integers $x$ and $y$ be $x_0k$ and $y_0k$.  Therefore $ax + by = c$.\\ \\
Now assume by way of contradiction that $ax + by = c$ and $(a, b) \nmid c$.  Let $d = (a, b)$.  Then by the definition of divides, there exist integers $j$ and $k$ such that $c = jdx + kdy = d(jx + ky)$.  Since $(jx + ky)$ is an integer, $d \mid c$, which contradicts our assumption.  Therefore if $ax + by = c$, then $(a, b) \mid c$.
\end{proof}

\setcounter{thm}{49}

%%%%%%%%%%%%1.50%%%%%%%%%%%%%%%%%%%%%%%%%%%%%%%%%%%%%%%%%%%%%%%%%%%%%
\begin{exer}[Euler]
A farmer lays out the sum of $1,770$ crowns in purchasing horses and
oxen.  He pays $31$ crowns for each horse and $21$ crowns for each
ox.  What are the possible numbers of horses and oxen that the
farmer bought?
\end{exer}

\begin{proof}[Solution]
Let $x$ be the number of horses and $y$ be the number of oxen.  So $1770 = 31x + 21y$.\\
$x = 9$, $y = 71$\\
$x = 30$, $y = 40$\\
$x = 51$, $y = 9$
\end{proof}

\pagebreak
%%%%%%%%%%%%1.51%%%%%%%%%%%%%%%%%%%%%%%%%%%%%%%%%%%%%%%%%%%%%%%%%%%%%
\begin{thm}
Let $a$, $b$, $c$, $x_0$, and $y_0$ be integers with $a$ and $b$ not
both $0$ such that $ax_0 + by_0 = c$.  Then the integers
\[x = x_0 + \frac{b}{(a,b)}\ {\textstyle\textrm{and}}\
y = y_0 - \frac{a}{(a,b)}\] also satisfy the linear Diophantine
equation $ax + by = c$.
\end{thm}


\begin{proof}[Proof]
Let $a$, $b$, $c$, $x_0$, and $y_0$ be integers with $a$ and $b$ not both $0$ such that $ax_0 + by_0 = c$. Also let $x = x_0 + {b \over (a, b)}$ and $y = y_0 - { a \over (a, b)}$.  So all the following equations are equivalent:
\begin{align*}
ax_0 &+ by_0\\
ax_0 + by_0 &+ {ab \over (a, b)} - {ab \over (a, b)} \\
ax_0 + {ab \over (a, b)} &+ by_0 - {ab \over (a, b)} \\ 
a(x_0 + {b \over (a, b)}) &+ b(y_0 - { a \over (a, b)})\\
ax &+ by
\end{align*}

Therefore $ax + by = c$.

\end{proof}

\setcounter{thm}{52}

%%%%%%%%%%%%1.53%%%%%%%%%%%%%%%%%%%%%%%%%%%%%%%%%%%%%%%%%%%%%%%%%%%%%
\begin{thm}
Let $a$, $b$, and $c$ be integers with $a$ and $b$ not both $0$. If
$x=x_0$, $y=y_0$ is an integer solution to the equation $ax+by=c$
(that is, $ax_0 + by_0 = c$) then for every integer~$k$, the numbers
\[x = x_0 + \frac{kb}{(a,b)}\ {\textstyle\textrm{and}}\ y = y_0 -
\frac{ka}{(a,b)}\] are integers that also satisfy the linear
Diophantine equation $ax + by = c$.  Moreover, every solution to the
linear Diophantine equation $ax + by = c$ is of this form.
\end{thm}

\begin{proof}[Proof]
Let $a$, $b$, and $c$ be integers with $a$ and $b$ not both $0$.  Let $S$ be the set of all pairs of integers $\pair{x}{y}$ such that $x$ and $y$ satisfy the equation $ax + by = c$.  Let $\pair{x_0}{y_0}$ be an element of $S$.  Let $x_k = x_0 + {kb \over (a, b)}$ and $y_k = y_0 - {ka \over (a, b)}$.  We will show both that for all $k$, $\pair{x_k}{y_k}$ is an element of $S$ and that all elements of $S$ may be expressed in the form $\pair{x_k}{y_k}$.\pagebreak \\
First, for any $\pair{x_k}{y_k}$, the following equations are equivilant.

\begin{align*}
ax_k &+ by_k\\
a(x_0 + {kb \over (a, b)}) &+ b(y_0 - {ka \over (a, b)})\\
ax_0 + {akb \over (a, b)} &+ by_0 - {bka \over (a, b)} \\
ax_0 + by_0 &+ {abk \over (a, b)} - {abk \over (a, b)}\\
ax_0 &+ by_0
\end{align*}
Since $ax_0 + by_0 = c$, $ax_k + by_k = c$, so $\pair{x_k}{y_k}$ is an element of S.\\
\\

Now consider any pair of integers $\pair{x'}{y'}$ which is an element of $S$.  Once again, the following equations are all equivalent.
\begin{align*}
ax' &+ by'\\
ax' + by' &+ {abk \over (a, b)} - {abk \over (a, b)} &\textsl{for some integer}\ k\\
ax' + {akb \over (a, b)} &+ by' - {bka \over (a, b)}\\
a(x' + {kb \over (a, b)}) &+ b(y' - {ka \over (a, b)})\\
ax_k &+ by_k
\end{align*}

Therefore any pair of integers $\pair{x'}{y'}$ in $S$ can be expressed as $\pair{x_k}{y_k}$.
\end{proof}

%%%%%%%%%%%%1.54%%%%%%%%%%%%%%%%%%%%%%%%%%%%%%%%%%%%%%%%%%%%%%%%%%%%%
\begin{exer}
Find all integer solutions to the equation $24x+9y=33$.
\end{exer}

\begin{proof}[Proof]
Using the Extended Euclidean Algorithm from theorem 1.40, we find that $3 = 24(-1) + 9(3)$, where $(24, 9) = 3$.  From here we see that $33 = (3)11 = 3(24(-1) + 9(3)) = 24(-3) + 9(9)$.  Therefore by theorem 1.53, we see that all integer solutions to the equation $24x+9y=33$ will be of the form $x_k = -3 + {k9 \over 3} = -3 + 3k$ and $y_k = 9 - {k24 \over 3} = 9 - 8k$.
\end{proof}

\end{document}
