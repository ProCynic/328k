
\documentclass[12pt,leqno]{article}

\usepackage{amsmath,amsfonts,amssymb,amscd,amsthm,amsbsy,upref}


\textheight=8.5truein
\textwidth=6.0truein
\hoffset=-.5truein
\voffset=-.5truein
\numberwithin{equation}{section}
\pagestyle{headings}
\footskip=36pt


\swapnumbers
\newtheorem{thm}{Theorem}[section]
\newtheorem{hthm}[thm]{*Theorem}
\newtheorem{lem}[thm]{Lemma}
\newtheorem{cor}[thm]{Corollary}
\newtheorem{prop}[thm]{Proposition}
\newtheorem{con}[thm]{Conjecture}
\newtheorem{exer}[thm]{Exercise}
\newtheorem{bpe}[thm]{Blank Paper Exercise}
\newtheorem{apex}[thm]{Applications Exercise}
\newtheorem{ques}[thm]{Question}
\newtheorem{scho}[thm]{Scholium}
\newtheorem*{Exthm}{Example Theorem}
\newtheorem*{Thm}{Theorem}
\newtheorem*{Con}{Conjecture}
\newtheorem*{Axiom}{Axiom}



\theoremstyle{definition}
\newtheorem*{Ex}{Example}
\newtheorem*{Def}{Definition}


\newcommand{\lcm}{\operatorname{lcm}}
\newcommand{\ord}{\operatorname{ord}}
\def\pfrac#1#2{{\left(\frac{#1}{#2}\right)}}


\makeindex

\begin{document}




\thispagestyle{plain}
\begin{flushright}
\large{\textbf{TYPE YOUR NAME HERE \\
HW 6: 1.48, 1.50, 1.51, 1.53, 1.54\\
M328K \\
February 7th, 2012 \\}}
\end{flushright}

%\maketitle
\markboth{}{} \setcounter{section}{0} \baselineskip=18pt

\setcounter{tocdepth}{4}


%%%%%%%%This is where you can change the numbering to match the problem you
%%%%%%%%are on.  Set the section to  the current chapter.

\setcounter{section}{1}

%%%%%%%%%Now, set the theorem number to one less than the first theorem in
%%%%%%%%%the assignment.
\setcounter{thm}{47}

%%%%%%%%%%%%1.48%%%%%%%%%%%%%%%%%%%%%%%%%%%%%%%%%%%%%%%%%%%%%%%%%%%%%
\begin{thm}
Given integers $a$, $b$, and $c$ with $a$ and $b$ not both 0, there
exist integers $x$ and $y$ that satisfy the equation~$ax + by = c$
if and only if $(a, b)|c$.
\end{thm}

\begin{proof}[Proof]
Type your proof here!
\end{proof}

\setcounter{thm}{49}

%%%%%%%%%%%%1.50%%%%%%%%%%%%%%%%%%%%%%%%%%%%%%%%%%%%%%%%%%%%%%%%%%%%%
\begin{exer}[Euler]
A farmer lays out the sum of $1,770$ crowns in purchasing horses and
oxen.  He pays $31$ crowns for each horse and $21$ crowns for each
ox.  What are the possible numbers of horses and oxen that the
farmer bought?
\end{exer}

\begin{proof}[Solution]
Type your solution here!
\end{proof}

%%%%%%%%%%%%1.51%%%%%%%%%%%%%%%%%%%%%%%%%%%%%%%%%%%%%%%%%%%%%%%%%%%%%
\begin{thm}
Let $a$, $b$, $c$, $x_0$, and $y_0$ be integers with $a$ and $b$ not
both $0$ such that $ax_0 + by_0 = c$.  Then the integers
\[x = x_0 + \frac{b}{(a,b)}\ {\textstyle\textrm{and}}\
y = y_0 - \frac{a}{(a,b)}\] also satisfy the linear Diophantine
equation $ax + by = c$.
\end{thm}


\begin{proof}[Proof]
Type your proof here!
\end{proof}

\setcounter{thm}{52}

%%%%%%%%%%%%1.53%%%%%%%%%%%%%%%%%%%%%%%%%%%%%%%%%%%%%%%%%%%%%%%%%%%%%
\begin{thm}
Let $a$, $b$, and $c$ be integers with $a$ and $b$ not both $0$. If
$x=x_0$, $y=y_0$ is an integer solution to the equation $ax+by=c$
(that is, $ax_0 + by_0 = c$) then for every integer~$k$, the numbers
\[x = x_0 + \frac{kb}{(a,b)}\ {\textstyle\textrm{and}}\ y = y_0 -
\frac{ka}{(a,b)}\] are integers that also satisfy the linear
Diophantine equation $ax + by = c$.  Moreover, every solution to the
linear Diophantine equation $ax + by = c$ is of this form.
\end{thm}

\begin{proof}[Proof]
Type your proof here!
\end{proof}

%%%%%%%%%%%%1.54%%%%%%%%%%%%%%%%%%%%%%%%%%%%%%%%%%%%%%%%%%%%%%%%%%%%%
\begin{exer}
Find all integer solutions to the equation $24x+9y=33$.
\end{exer}

\begin{proof}[Proof]
Type your proof here!
\end{proof}

\end{document}
