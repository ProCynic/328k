\documentclass[12pt,leqno]{article}

\usepackage{amsmath,amsfonts,amssymb,amscd,amsthm,amsbsy,upref}


\textheight=8.5truein
\textwidth=6.0truein
\hoffset=-.5truein
\voffset=-.5truein
\numberwithin{equation}{section}
\pagestyle{headings}
\footskip=36pt


\swapnumbers
\newtheorem{thm}{Theorem}[section]
\newtheorem{hthm}[thm]{*Theorem}
\newtheorem{lem}[thm]{Lemma}
\newtheorem{cor}[thm]{Corollary}
\newtheorem{prop}[thm]{Proposition}
\newtheorem{con}[thm]{Conjecture}
\newtheorem{exer}[thm]{Exercise}
\newtheorem{bpe}[thm]{Blank Paper Exercise}
\newtheorem{apex}[thm]{Applications Exercise}
\newtheorem{ques}[thm]{Question}
\newtheorem{scho}[thm]{Scholium}
\newtheorem*{Exthm}{Example Theorem}
\newtheorem*{Thm}{Theorem}
\newtheorem*{Con}{Conjecture}
\newtheorem*{Axiom}{Axiom}



\theoremstyle{definition}
\newtheorem*{Ex}{Example}
\newtheorem*{Def}{Definition}


\newcommand{\lcm}{\operatorname{lcm}}
\newcommand{\ord}{\operatorname{ord}}
\def\pfrac#1#2{{\left(\frac{#1}{#2}\right)}}


\makeindex

\begin{document}




\thispagestyle{plain}
\begin{flushright}
\large{\textbf{Geoffrey Parker - grp352 \\
HW 13: 3.1-3.3\\
M328K \\
March 6th, 2012 \\}}
\end{flushright}

%\maketitle
\markboth{}{} \setcounter{section}{0} \baselineskip=18pt

\setcounter{tocdepth}{4}


%%%%%%%%This is where you can change the numbering to match the problem you
%%%%%%%%are on.  Set the section to  the current chapter.

\setcounter{section}{3}

%%%%%%%%%Now, set the theorem number to one less than the first theorem in
%%%%%%%%%the assignment.
\setcounter{thm}{0}

%%%Exercise 3.1%%%%%%%%%%%%%%%%%%%%%%%%%%%%%%%%%%%%%%%%%%%%%%%%%%%%%%%%%%%%%%%%%%%%%

\begin{exer}
Show that $41$ divides $2^{20}-1$ by following these steps.  Explain
why each step is true.
\begin{enumerate} \item $2^5 \equiv -9 \pmod{41}$.  \item $(2^5)^4
\equiv (-9)^4 \pmod{41}$.  \item $2^{20} \equiv 81^2 \pmod{41}
\equiv (-1)^2\pmod{41}$.  \item $2^{20} - 1 \equiv 0 \pmod{41}$.
\end{enumerate}
\end{exer}

\begin{proof}[Solution]
$ $\\
\begin{enumerate}
\item $2^5 \equiv -9 \pmod{41}$.  Because $2^5 - (-9) = 41$ and $41 \mid 41$.
\item $(2^5)^4 \equiv (-9)^4 \pmod{41}$.  By step 1 and thm 1.18.
\item $2^{20} \equiv 81^2 \pmod{41} \equiv (-1)^2\pmod{41}$.  By step 2 and thm 1.11.
\item $2^{20} - 1 \equiv 0 \pmod{41}$. Because $1 \equiv 1 \pmod{41}$ and thm 1.12.
\end{enumerate}
\end{proof}

%%%Question 3.2%%%%%%%%%%%%%%%%%%%%%%%%%%%%%%%%%%%%%%%%%%%%%%%%%%%%%%%%%%%%%%%%%%%%%
\begin{ques}
In your head, can you find the natural number $k$, $0 \leq k \leq
11$, such that $k \equiv 37^{453} \pmod{12}$?
\end{ques}

\begin{proof}[Solution]
\begin{align*}
1 &\equiv 37 \pmod{12} \\
1^k &\equiv 37^k \pmod{12} \\
1 &\equiv 37^{453} \pmod{12}
\end{align*}
\end{proof}

%%%Question 3.3%%%%%%%%%%%%%%%%%%%%%%%%%%%%%%%%%%%%%%%%%%%%%%%%%%%%%%%%%%%%%%%%%%%%%
\begin{ques}
In your head or using paper and pencil, but no calculator, can you
find the natural number $k$, $0 \leq k \leq 6$, such that $2^{50}
\equiv k \pmod{7}$.
\end{ques}

\begin{proof}[Solution]
Well, $50 \equiv 2 \pmod{3}$, so $2^{50} \equiv 4 \pmod{7}$.
\end{proof}
\end{document}
