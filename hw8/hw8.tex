
\documentclass[12pt,leqno]{article}

\usepackage{amsmath,amsfonts,amssymb,amscd,amsthm,amsbsy,upref}


\textheight=8.5truein
\textwidth=6.0truein
\hoffset=-.5truein
\voffset=-.5truein
\numberwithin{equation}{section}
\pagestyle{headings}
\footskip=36pt


\swapnumbers
\newtheorem{thm}{Theorem}[section]
\newtheorem{hthm}[thm]{*Theorem}
\newtheorem{lem}[thm]{Lemma}
\newtheorem{cor}[thm]{Corollary}
\newtheorem{prop}[thm]{Proposition}
\newtheorem{con}[thm]{Conjecture}
\newtheorem{exer}[thm]{Exercise}
\newtheorem{bpe}[thm]{Blank Paper Exercise}
\newtheorem{apex}[thm]{Applications Exercise}
\newtheorem{ques}[thm]{Question}
\newtheorem{scho}[thm]{Scholium}
\newtheorem*{Exthm}{Example Theorem}
\newtheorem*{Thm}{Theorem}
\newtheorem*{Con}{Conjecture}
\newtheorem*{Axiom}{Axiom}



\theoremstyle{definition}
\newtheorem*{Ex}{Example}
\newtheorem*{Def}{Definition}


\newcommand{\lcm}{\operatorname{lcm}}
\newcommand{\ord}{\operatorname{ord}}
\def\pfrac#1#2{{\left(\frac{#1}{#2}\right)}}


\makeindex

\begin{document}




\thispagestyle{plain}
\begin{flushright}
\large{\textbf{TYPE YOUR NAME HERE \\
HW 8: 2.7-2.9, 2.10-2.14, 2.19-2.21\\
M328K \\
February 14th, 2012 \\}}
\end{flushright}

%\maketitle
\markboth{}{} \setcounter{section}{0} \baselineskip=18pt

\setcounter{tocdepth}{4}


%%%%%%%%This is where you can change the numbering to match the problem you
%%%%%%%%are on.  Set the section to  the current chapter.

\setcounter{section}{2}

%%%%%%%%%Now, set the theorem number to one less than the first theorem in
%%%%%%%%%the assignment.
\setcounter{thm}{6}

%%%Theorem 2.7%%%%%%%%%%%%%%%%%%%%%%%%%%%%%%%%%%%%%%%%%%%%%%%%%%%%%%%%%%%%%%%%%%%%%

\begin{thm}[Fundamental Theorem of Arithmetic-Existence Part)]
Every natural number greater than $1$ is either a prime number or it
can be expressed as a finite product of prime numbers. That is, for
every natural number $n$ greater than $1$, there exist distinct
primes $p_1, p_2, \hdots, p_m$ and natural numbers $r_1, r_2,
\hdots, r_m$ such that \[n = p_1^{r_1}p_2^{r_2}\cdots p_m^{r_m}.\]
\end{thm}

\begin{proof}[Proof]
Type your proof here!
\end{proof}


%%%Lemma 2.8%%%%%%%%%%%%%%%%%%%%%%%%%%%%%%%%%%%%%%%%%%%%%%%%%%%%%%%%%%%%%%%%%%%%%

\begin{lem}
Let $p$ and $q_1, q_2, \hdots, q_n$ all be primes and let $k$ be a
natural number such that~$p k = q_1q_2 \cdots q_n$.  Then $p = q_i$
for some $i$.
\end{lem}

\begin{proof}[Proof]
Type your proof here!
\end{proof}


%%%Theorem 2.9%%%%%%%%%%%%%%%%%%%%%%%%%%%%%%%%%%%%%%%%%%%%%%%%%%%%%%%%%%%%%%%%%%%%%

\begin{thm}[Fundamental Theorem of Arithmetic-Uniqueness part]
Let $n$ be a natural number.  Let $\{p_1, p_2, \hdots, p_m\}$ and
$\{q_1, q_2, \hdots, q_s\}$ be sets of primes with $p_i \neq p_j$ if
$i \neq j$ and $q_i \neq q_j$ if $i \neq j$. Let $\{r_1, r_2,
\hdots, r_m\}$ and $\{t_1, t_2, \hdots, t_s\}$ be sets of natural
numbers such that
\begin{align*}
n & = p_1^{r_1}p_2^{r_2}\cdots p_m^{r_m} \\
  & = q_1^{t_1}q_2^{t_2}\cdots q_s^{t_s}.
\end{align*}
Then $m = s$ and $\{p_1, p_2, \hdots, p_m\} = \{q_1, q_2, \hdots,
q_s\}$. That is, the sets of primes are equal but their elements are
not necessarily listed in the same order; that is, $p_i$ may or may
not equal $q_i$.  Moreover, if~$p_i = q_j$ then $r_i = t_j$.  In
other words, if we express the same natural number as a product of
powers of distinct primes, then the expressions are identical except
for the ordering of the factors.
\end{thm}

\begin{proof}[Proof]
Type your proof here!
\end{proof}

\setcounter{thm}{9}

%%%Exercise 2.10%%%%%%%%%%%%%%%%%%%%%%%%%%%%%%%%%%%%%%%%%%%%%%%%%%%%%%%%%%%%%%%%%%%%%

\begin{exer}
Express $n = 12!$ as a product of primes.
\end{exer}

\begin{proof}[Solution]
Type your solution here!
\end{proof}

%%%Exercise 2.11%%%%%%%%%%%%%%%%%%%%%%%%%%%%%%%%%%%%%%%%%%%%%%%%%%%%%%%%%%%%%%%%%%%%%

\begin{exer}
Determine the number of zeroes at the end of $25!$.
\end{exer}

\begin{proof}[Solution]
Type your solution here!
\end{proof}

%%%Theorem 2.12%%%%%%%%%%%%%%%%%%%%%%%%%%%%%%%%%%%%%%%%%%%%%%%%%%%%%%%%%%%%%%%%%%%%%

\begin{thm}
Let $a$ and $b$ be natural numbers greater than 1 and let $p_1^{r_1}p_2^{r_2}\cdots p_m^{r_m}$ be the unique prime factorization of $a$ and let $q_1^{t_1}q_2^{t_2}\cdots q_s^{t_s}$ be the unique prime factorization of $b$.  Then $a \mid b$ if and only if for all $i \leq m$ there exists a $j\leq s$ such that $p_i = q_j$ and $r_i \leq t_j$.
\end{thm}

\begin{proof}[Proof]
Type your proof here!
\end{proof}

%%%Theorem 2.13%%%%%%%%%%%%%%%%%%%%%%%%%%%%%%%%%%%%%%%%%%%%%%%%%%%%%%%%%%%%%%%%%%%%%

\begin{thm}
If $a$ and $b$ are natural numbers and $a^2 \mid b^2$ then $a \mid b$.
\end{thm}

\begin{proof}[Proof]
Type your proof here!
\end{proof}

%%%Exercise 2.14%%%%%%%%%%%%%%%%%%%%%%%%%%%%%%%%%%%%%%%%%%%%%%%%%%%%%%%%%%%%%%%%%%%%%

\begin{exer}
Find $3^14 \cdot 7^22 \cdot 11^5 \cdot 17^3 \cdot 5^2 \cdot 11^4 \cdot 13^8 \cdot 17$.
\end{exer}

\begin{proof}[Solution]
Type your solution here!
\end{proof}

\setcounter{thm}{18}

%%%Theorem 2.19%%%%%%%%%%%%%%%%%%%%%%%%%%%%%%%%%%%%%%%%%%%%%%%%%%%%%%%%%%%%%%%%%%%%%

\begin{thm}
There do not exist natural numbers $m$ and $n$ such that $7m^2 = n^2$.
\end{thm}

\begin{proof}[Proof]
Type your proof here!
\end{proof}

%%%Theorem 2.20%%%%%%%%%%%%%%%%%%%%%%%%%%%%%%%%%%%%%%%%%%%%%%%%%%%%%%%%%%%%%%%%%%%%%

\begin{thm}
There do not exist natural numbers $m$ and $n$ such that $24m^3 = n^3$.
\end{thm}

\begin{proof}[Proof]
Type your proof here!
\end{proof}

%%%Exercise 2.21%%%%%%%%%%%%%%%%%%%%%%%%%%%%%%%%%%%%%%%%%%%%%%%%%%%%%%%%%%%%%%%%%%%%%

\begin{exer}
Show that $\sqrt{7}$ is irrational.  That is, there do not exist natural numbers $n$ and $m$ such that $\sqrt{7} = {n \over m}$.
\end{exer}

\begin{proof}[Solution]
Type your solution here!
\end{proof}

\end{document}