\documentclass[12pt,leqno]{article}

\usepackage{amsmath,amsfonts,amssymb,amscd,amsthm,amsbsy,upref}


\textheight=8.5truein
\textwidth=6.0truein
\hoffset=-.5truein
\voffset=-.5truein
\numberwithin{equation}{section}
\pagestyle{headings}
\footskip=36pt


\swapnumbers
\newtheorem{thm}{Theorem}[section]
\newtheorem{hthm}[thm]{*Theorem}
\newtheorem{lem}[thm]{Lemma}
\newtheorem{cor}[thm]{Corollary}
\newtheorem{prop}[thm]{Proposition}
\newtheorem{con}[thm]{Conjecture}
\newtheorem{exer}[thm]{Exercise}
\newtheorem{bpe}[thm]{Blank Paper Exercise}
\newtheorem{apex}[thm]{Applications Exercise}
\newtheorem{ques}[thm]{Question}
\newtheorem{scho}[thm]{Scholium}
\newtheorem*{Exthm}{Example Theorem}
\newtheorem*{Thm}{Theorem}
\newtheorem*{Con}{Conjecture}
\newtheorem*{Axiom}{Axiom}



\theoremstyle{definition}
\newtheorem*{Ex}{Example}
\newtheorem*{Def}{Definition}


\newcommand{\lcm}{\operatorname{lcm}}
\newcommand{\ord}{\operatorname{ord}}
\def\pfrac#1#2{{\left(\frac{#1}{#2}\right)}}


\makeindex

\begin{document}




\thispagestyle{plain}
\begin{flushright}
\large{\textbf{TYPE YOUR NAME HERE \\
HW 21: 4.7 - 4.11\\
M328K \\
April 10th, 2012 \\}}
\end{flushright}

%\maketitle
\markboth{}{} \setcounter{section}{0} \baselineskip=18pt

\setcounter{tocdepth}{4}


%%%%%%%%This is where you can change the numbering to match the problem you
%%%%%%%%are on.  Set the section to  the current chapter.

\setcounter{section}{4}

%%%%%%%%%Now, set the theorem number to one less than the first theorem in
%%%%%%%%%the assignment.
\setcounter{thm}{6}

%%%Question 4.7%%%%%%%%%%%%%%%%%%%%%%%%%%%%%%%%%%%%%%%%%%%%%%%%%%%%%%%%%%%%%%%%%%%%%

\begin{ques}
Choose some relatively prime natural numbers $a$ and $n$ and compute
the order of $a$ modulo $n$. Frame a conjecture concerning how large
the order of $a$ modulo $n$ can be, depending on $n$.
\end{ques}
\begin{proof}[Answer]
\end{proof}

%%%Theorem 4.8%%%%%%%%%%%%%%%%%%%%%%%%%%%%%%%%%%%%%%%%%%%%%%%%%%%%%%%%%%%%%%%%%%%%%

\begin{thm}
Let $a$ and $n$ be natural numbers with $(a, n) = 1$ and let $k =
\ord_n(a)$.  Then the numbers $a^1$, $a^2$, $\dots$, $a^k$ are
pairwise incongruent modulo $n$.
\end{thm}
\begin{proof}[Proof]
\end{proof}

%%%Theorem 4.9%%%%%%%%%%%%%%%%%%%%%%%%%%%%%%%%%%%%%%%%%%%%%%%%%%%%%%%%%%%%%%%%%%%%%
\begin{thm}
Let $a$ and $n$ be natural numbers with $(a, n) = 1$ and let $k =
\ord_n(a)$.  For any natural number $m$, $a^m$ is congruent modulo
$n$ to one of the numbers~$a^1$, $a^2$, $\hdots$, $a^k$.
\end{thm}
\begin{proof}[Proof]
\end{proof}

%%%Theorem 4.10%%%%%%%%%%%%%%%%%%%%%%%%%%%%%%%%%%%%%%%%%%%%%%%%%%%%%%%%%%%%%%%%%%%%%

\begin{thm}
Let $a$ and $n$ be natural numbers with $(a, n) = 1$, let $k =
\ord_n(a)$, and let $m$ be a natural number.  Then $a^m \equiv 1
\pmod{n}$ if and only if $k|m$.
\end{thm}
\begin{proof}[Proof]
\end{proof}

%%%Theorem 4.11%%%%%%%%%%%%%%%%%%%%%%%%%%%%%%%%%%%%%%%%%%%%%%%%%%%%%%%%%%%%%%%%%%%%%
\begin{thm}
Let $a$ and $n$ be natural numbers with $(a, n) = 1$. Then
$\ord_n(a) < n$.
\end{thm}
\begin{proof}[Proof]
\end{proof}
\end{document}
