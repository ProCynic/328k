\documentclass[12pt,leqno]{article}

\usepackage{amsmath,amsfonts,amssymb,amscd,amsthm,amsbsy,upref}


\textheight=8.5truein
\textwidth=6.0truein
\hoffset=-.5truein
\voffset=-.5truein
\numberwithin{equation}{section}
\pagestyle{headings}
\footskip=36pt


\swapnumbers
\newtheorem{thm}{Theorem}[section]
\newtheorem{hthm}[thm]{*Theorem}
\newtheorem{lem}[thm]{Lemma}
\newtheorem{cor}[thm]{Corollary}
\newtheorem{prop}[thm]{Proposition}
\newtheorem{con}[thm]{Conjecture}
\newtheorem{exer}[thm]{Exercise}
\newtheorem{bpe}[thm]{Blank Paper Exercise}
\newtheorem{apex}[thm]{Applications Exercise}
\newtheorem{ques}[thm]{Question}
\newtheorem{scho}[thm]{Scholium}
\newtheorem*{Exthm}{Example Theorem}
\newtheorem*{Thm}{Theorem}
\newtheorem*{Con}{Conjecture}
\newtheorem*{Axiom}{Axiom}



\theoremstyle{definition}
\newtheorem*{Ex}{Example}
\newtheorem*{Def}{Definition}


\newcommand{\lcm}{\operatorname{lcm}}
\newcommand{\ord}{\operatorname{ord}}
\def\pfrac#1#2{{\left(\frac{#1}{#2}\right)}}
\newcommand{\card}[1]{\left| #1 \right|}


\makeindex

\begin{document}

\thispagestyle{plain}
\begin{flushright}
\large{\textbf{Geoffrey Parker - grp352 \\
HW 21: 4.7 - 4.11\\
M328K \\
April 10th, 2012 \\}}
\end{flushright}

%\maketitle
\markboth{}{} \setcounter{section}{0} \baselineskip=18pt

\setcounter{tocdepth}{4}


%%%%%%%%This is where you can change the numbering to match the problem you
%%%%%%%%are on.  Set the section to  the current chapter.

\setcounter{section}{4}

%%%%%%%%%Now, set the theorem number to one less than the first theorem in
%%%%%%%%%the assignment.
\setcounter{thm}{6}

%%%Question 4.7%%%%%%%%%%%%%%%%%%%%%%%%%%%%%%%%%%%%%%%%%%%%%%%%%%%%%%%%%%%%%%%%%%%%%

\begin{ques}
Choose some relatively prime natural numbers $a$ and $n$ and compute
the order of $a$ modulo $n$. Frame a conjecture concerning how large
the order of $a$ modulo $n$ can be, depending on $n$.
\end{ques}
\begin{proof}[Answer]
Let $a = 3$ and $n = 7$.  So:
\[3^1 \equiv 3 \pmod{7}\]
\[3^2 \equiv 2 \pmod{7}\]
\[3^3 \equiv 6 \pmod{7}\]
\[3^4 \equiv 4 \pmod{7}\]
\[3^5 \equiv 5 \pmod{7}\]
\[3^6 \equiv 1 \pmod{7}\]
Conjecture: $ord_n(a) < n$.
\end{proof}

%%%Theorem 4.8%%%%%%%%%%%%%%%%%%%%%%%%%%%%%%%%%%%%%%%%%%%%%%%%%%%%%%%%%%%%%%%%%%%%%

\begin{thm}
Let $a$ and $n$ be natural numbers with $(a, n) = 1$ and let $k =
\ord_n(a)$.  Then the numbers $a^1$, $a^2$, $\dots$, $a^k$ are
pairwise incongruent modulo $n$.
\end{thm}
\begin{proof}[Proof]
et $a$ and $n$ be natural numbers with $(a, n) = 1$ and let $k = \ord_n(a)$.  We will show the numbers $a^1$, $a^2$, $\dots$, $a^k$ are pairwise incongruent modulo $n$.  Assume by way of contradiction that there exist two natural numbers $i$ and $j$ such that $1 \leq i, j \leq k$, $i \neq j$, and $a^i \equiv a^j \pmod{n}$.  Assume without loss of generality that $i > j$.  Let $m = i - j$ so $i = j + m$ and $0 < m < k$.  Then $a^i = a^ja^m$ and $a^j = a^j \cdot 1$.  Now substituting into our congurence, we obtain $a^ja^m \equiv a^j1 \pmod{n}$.  Therefore by theorem 4.5 $a^m \equiv 1 \pmod{n}$.  However, $m < k$ and $k$ is defined to be the smallest natural number such that $a^k$ is congruent to 1 modulo $n$, so we have a contradiction.
\end{proof}

\pagebreak
%%%Theorem 4.9%%%%%%%%%%%%%%%%%%%%%%%%%%%%%%%%%%%%%%%%%%%%%%%%%%%%%%%%%%%%%%%%%%%%%
\begin{thm}
Let $a$ and $n$ be natural numbers with $(a, n) = 1$ and let $k =
\ord_n(a)$.  For any natural number $m$, $a^m$ is congruent modulo
$n$ to one of the numbers~$a^1$, $a^2$, $\hdots$, $a^k$.
\end{thm}
\begin{proof}[Proof]
Let $a$ and $n$ be natural numbers with $(a, n) = 1$ and let $k = \ord_n(a)$.  Let $m$ be any arbitrary natural number.  We will show that $a^m$ is congruent modulo $n$ to one of the numbers~$a^1$, $a^2$, $\hdots$, $a^k$.  If $m \leq k$, then $a^m \equiv a^m \pmod{n}$ so we're done.  In the case that $m > k$, use the division algorithm to find two integers $q$Let $a$ and $n$ be natural numbers with $(a, n) = 1$. Then
$\ord_n(a) < n$. and $r$ such that $m = qk + r$ where $0 \leq r < k$.  If $r = 0$, then let $s = q - 1$ and $t = k$, otherwise let $s = q$ and $t = r$. So $m = sk + t$ and $0 < t \leq k$.   Now $a^{sk} = (a^k)^s$, and because $a^k \equiv 1 \pmod{n}$, we can say by theorem 1.18 that $(a^k)^s \equiv 1^s \pmod{n}$.  Then by theorem 1.??  $a^{ks}a^t \equiv 1a^t \pmod{n}$ or equivalently $a^m \equiv a^t$.  Therefore since $0 < t \leq k$, we have shown that $a^m$ is congruent modulo $n$ to one of the numbers~$a^1$, $a^2$, $\hdots$, $a^k$.
\end{proof}

%%%Theorem 4.10%%%%%%%%%%%%%%%%%%%%%%%%%%%%%%%%%%%%%%%%%%%%%%%%%%%%%%%%%%%%%%%%%%%%%

\begin{thm}
Let $a$ and $n$ be natural numbers with $(a, n) = 1$, let $k =
\ord_n(a)$, and let $m$ be a natural number.  Then $a^m \equiv 1
\pmod{n}$ if and only if $k|m$.
\end{thm}
\begin{proof}[Proof]
Let $a$ and $n$ be natural numbers with $(a, n) = 1$, let $k = \ord_n(a)$, and let $m$ be a natural number.  Use the division algorithm to find integers $q$ and $r$ such that $m = qk + r$ where $0 \leq r < k$.  Then $a^{qk} = (a^k)^q$ and since $a^k \equiv 1 \pmod{n}$,  by theorem 1.18 $(a^k)^q \equiv 1^q \pmod{n}$.  By theorem 1.??  $a^{qk}a^r \equiv 1a^r \pmod{n}$ and $a^m \equiv a^r \pmod{n}$.\\

 
If $k \mid m$, then $r$ will equal $0$, so $a^r = 1$ and $a^m \equiv 1 \pmod{n}$.\\

If $k \nmid m$ then assume by way of contradiction that $a^m \equiv 1 \pmod{n}$.  In this case, by theorem 1.11 $a^r \equiv 1 \pmod{n}$ and since $0 < r < k$, this contradicts the definition of $k$ as $\ord_n(a)$.
\end{proof}

\pagebreak
%%%Theorem 4.11%%%%%%%%%%%%%%%%%%%%%%%%%%%%%%%%%%%%%%%%%%%%%%%%%%%%%%%%%%%%%%%%%%%%%
\begin{thm}
Let $a$ and $n$ be natural numbers with $(a, n) = 1$. Then
$\ord_n(a) < n$.
\end{thm}
\begin{proof}[Proof]
Let $a$ and $n$ be natural numbers with $(a, n) = 1$. Let $k = \ord_n(a)$.  Assume by way of contradiction that $k \geq n$.  Let the set $S = \{a^1, a^2, \ldots , a^k\}$.  Because $k \geq n$,  $\card{S} > n-1$.  By the definition of complete residue systems, every natural number is congruent modulo $n$ to exactly one element of the canonical complete residue system modulo $n$.  However, because $(a, n) = 1$, there is no $s$ element of $S$ such that $s \equiv 0 \pmod{n}$. Therefore every element of $S$ is congruent modulo $n$ to exactly one element of $T = \{1, 2, \dots n - 1\}$. Note that $\card{T} = n-1$. So by the pigeon hole principle there must be some element $x$ of $T$ to which two different elements of $S$, call them $a^i$ and $a^j$ are congruent modulo $n$.  So by theorem 1.11 $a^i \equiv a^j \pmod{n}$.  However, since $(a, n) = 1$, theorem 4.8 states that all elements of $S$ are pairwise incongruent modulo $n$.  So we have a contradiction.
\end{proof}
\end{document}
