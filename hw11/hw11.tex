\documentclass[12pt,leqno]{article}

\usepackage{amsmath,amsfonts,amssymb,amscd,amsthm,amsbsy,upref}


\textheight=8.5truein
\textwidth=6.0truein
\hoffset=-.5truein
\voffset=-.5truein
\numberwithin{equation}{section}
\pagestyle{headings}
\footskip=36pt


\swapnumbers
\newtheorem{thm}{Theorem}[section]
\newtheorem{hthm}[thm]{*Theorem}
\newtheorem{lem}[thm]{Lemma}
\newtheorem{cor}[thm]{Corollary}
\newtheorem{prop}[thm]{Proposition}
\newtheorem{con}[thm]{Conjecture}
\newtheorem{exer}[thm]{Exercise}
\newtheorem{bpe}[thm]{Blank Paper Exercise}
\newtheorem{apex}[thm]{Applications Exercise}
\newtheorem{ques}[thm]{Question}
\newtheorem{scho}[thm]{Scholium}
\newtheorem*{Exthm}{Example Theorem}
\newtheorem*{Thm}{Theorem}
\newtheorem*{Con}{Conjecture}
\newtheorem*{Axiom}{Axiom}



\theoremstyle{definition}
\newtheorem*{Ex}{Example}
\newtheorem*{Def}{Definition}


\newcommand{\lcm}{\operatorname{lcm}}
\newcommand{\ord}{\operatorname{ord}}
\def\pfrac#1#2{{\left(\frac{#1}{#2}\right)}}

\newcommand{\modulo}[3]{$#1 \equiv #2$ (mod $#3$)}


\makeindex

\begin{document}




\thispagestyle{plain}
\begin{flushright}
\large{\textbf{Geoffrey Parker \\
HW 11: 2.37, 2.38, 2.41\\
M328K \\
February 28th, 2012 \\}}
\end{flushright}

%\maketitle
\markboth{}{} \setcounter{section}{0} \baselineskip=18pt

\setcounter{tocdepth}{4}


%%%%%%%%This is where you can change the numbering to match the problem you
%%%%%%%%are on.  Set the section to  the current chapter.

\setcounter{section}{2}

%%%%%%%%%Now, set the theorem number to one less than the first theorem in
%%%%%%%%%the assignment.
\setcounter{thm}{36}

%%%Theorem 2.37%%%%%%%%%%%%%%%%%%%%%%%%%%%%%%%%%%%%%%%%%%%%%%%%%%%%%%%%%%%%%%%%%%%%%

\begin{thm}
If $r_1, r_2, \hdots, r_m$ are natural numbers and each one is
congruent to $1$ modulo~$4$, then the product $r_1r_2 \cdots r_m$ is
also congruent to $1$ modulo $4$.
\end{thm}

\begin{proof}[Proof]
Let $r_1, r_2, \hdots, r_m$ be natural numbers with each one being congruent to $1$ modulo~$4$. We will show that the product $r_1r_2 \cdots r_m$ is also congruent to $1$ modulo $4$.  Define a sequence of $s$'s, with $s_0 = 1$ and every subsequent $s_i = s_{i-1}r_i$. By this definition, $s_m = r_1r_2 \cdots r_m$.  We will show by induction that $s_m \equiv 1$ (mod 4).  As a base case, $s_1 = 1\cdot r_1$, and $r_1 \equiv 1$ (mod 4) by definition.  Our induction hypothesis is that there exists some natural number $N$ such that $s_N \equiv 1$ (mod 4).  Then $s_{N+1} = s_Nr_{N+1}$.  Since $s_N \equiv 1$ (mod 4) and $r_{N+1} \equiv 1$ (mod 4), then by theorem 1.14 $s_Nr_{N+1} \equiv 1\cdot 1$ (mod 4).  Therefore $s_{N+1} \equiv 1$ (mod 4).  We have now shown by induction that $r_1r_2 \cdots r_m$ is congruent to $1$ modulo $4$.
\end{proof}


%%%Theorem 2.38%%%%%%%%%%%%%%%%%%%%%%%%%%%%%%%%%%%%%%%%%%%%%%%%%%%%%%%%%%%%%%%%%%%%%

\begin{thm}[Infinitude of $4k + 3$ Primes Theorem]\index{prime number!congruent to 3 modulo 4}
There are infinitely many prime numbers that are congruent to $3$
modulo~$4$.
\end{thm}

\begin{proof}[Proof]
Let $P$ be the set of primes numbers that are congruent to $3$ modulo $4$.  We will show that $P$ is infinite.  Since $3$ is prime and \modulo{3}{3}{4}, $P$ is not empty.  Assume by way of contradiction that $P$ is finite, with some cardinality $n$.  Let the natural number $Q$ be the product of all the primes in $P$.  By repeatidly applying theorem 1.14, we see that \modulo{Q}{3^n}{4}.  Now, we have two cases, $n$ is even and $n$ is odd.\\  If $n$ is odd, then \modulo{3^n}{3}{4}, so by theorem 1.11 \modulo{Q}{3}{4}.  Let $q$ be a prime divisor of  $Q + 4$. (TODO: \modulo{q}{3}{4}) Then we have $q \mid Q$ and $q \mid Q+4$.  Therefore $q \mid 4$, which is a contradiction because $q$ is prime and \modulo{q}{3}{4}.\\
If $n$ is even, then \modulo{3^n}{1}{4} so by theorem 1.11 \modulo{Q}{1}{4}.  Let $q$ be a prime divisor of  $Q + 2$. (TODO: \modulo{q}{3}{4}) Then we have $q \mid Q$ and $q \mid Q+2$.  Therefore $q \mid 2$, which is a contradiction because $q$ is prime and \modulo{q}{3}{4}.
\end{proof}


\setcounter{thm}{40}
\pagebreak
%%%Exercise 2.41%%%%%%%%%%%%%%%%%%%%%%%%%%%%%%%%%%%%%%%%%%%%%%%%%%%%%%%%%%%%%%%%%%%%%

\begin{exer}
Use  polynomial long division to compute $(x^m-1) \div (x-1)$.
\end{exer}

\begin{proof}[Solution]
\begin{align*}
x^m - 1 &= x^{m-1}(x-1) + x^{m-1} - 1\\
x^{m-1} - 1 &= x^{m-2}(x-1) + x^{m-2} - 1\\
&\cdots \\
x^2 - 1 &= x(x-1) + x - 1\\
x - 1 &= 1(x-1)
\end{align*}
Therefore:
\[(x^m-1) \div (x-1) = x^{m-1} + x^{m-2} + \cdots + x + 1\]
\end{proof}
\end{document}
