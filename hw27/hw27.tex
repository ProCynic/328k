\documentclass[12pt,leqno]{article}

\usepackage{amsmath,amsfonts,amssymb,amscd,amsthm,amsbsy,upref}


\textheight=8.5truein
\textwidth=6.0truein
\hoffset=-.5truein
\voffset=-.5truein
\numberwithin{equation}{section}
\pagestyle{headings}
\footskip=36pt


\swapnumbers
\newtheorem{thm}{Theorem}[section]
\newtheorem{hthm}[thm]{*Theorem}
\newtheorem{lem}[thm]{Lemma}
\newtheorem{cor}[thm]{Corollary}
\newtheorem{prop}[thm]{Proposition}
\newtheorem{con}[thm]{Conjecture}
\newtheorem{exer}[thm]{Exercise}
\newtheorem{bpe}[thm]{Blank Paper Exercise}
\newtheorem{apex}[thm]{Applications Exercise}
\newtheorem{ques}[thm]{Question}
\newtheorem{scho}[thm]{Scholium}
\newtheorem*{Exthm}{Example Theorem}
\newtheorem*{Thm}{Theorem}
\newtheorem*{Con}{Conjecture}
\newtheorem*{Axiom}{Axiom}



\theoremstyle{definition}
\newtheorem*{Ex}{Example}
\newtheorem*{Def}{Definition}


\newcommand{\lcm}{\operatorname{lcm}}
\newcommand{\ord}{\operatorname{ord}}
\def\pfrac#1#2{{\left(\frac{#1}{#2}\right)}}


\makeindex

\begin{document}




\thispagestyle{plain}
\begin{flushright}
  \large{\textbf{Geoffrey Parker - grp352\\
      HW 27: 5.6 - 5.7\\
      M328K \\
      May 3rd, 2012 \\}}
\end{flushright}

%\maketitle
\markboth{}{} \setcounter{section}{0} \baselineskip=18pt

\setcounter{tocdepth}{4}


%%%%%%%%This is where you can change the numbering to match the problem you
%%%%%%%%are on.  Set the section to  the current chapter.

\setcounter{section}{5}

%%%%%%%%%Now, set the theorem number to one less than the first theorem in
%%%%%%%%%the assignment.
\setcounter{thm}{5}

%%%Exercise 5.6%%%%%%%%%%%%%%%%%%%%%%%%%%%%%%%%%%%%%%%%%%%%%%%%%%%%%%%%%%%%%%%%%%%%%
\begin{exer}
  Describe an RSA Public Key Code System based on the primes $11$ and
  $17$.  Encode and decode several messages.
\end{exer}
\begin{proof}[Solution]
  Let $p = 11$ and $q = 17$.  To generate our keys, $n = pq = 187$ and $\phi(n) = 10 \cdot 16 = 160$.  Then let $e = 3$ and note that $(3, 160) = 1$.  Now we need to use the Extended Euclidean Algorithm to find $d$ such that $3d + (-160)y = 1$.
  \begin{align*}
    3 &= 1 \cdot -160 + 163\\
    -160 &= -1 \cdot 163 + 3\\
    163 &= 54 \cdot 3 + 1\\
    1 &= (-54)3 + 163\\
    163 &= 3 - (-160)\\
    1 &= (-54)3 + 3 + (-1)(-160)\\
    1 &= (-53)3 + (-1)(-160)\\
  \end{align*}
  So $d \equiv -53 \pmod{160}$, and we can add 160 to get $d = 107$.  So now we have our public key: $3, 187$ and our private key: $107, 187$.  Time to encrypt some messages.
  
  Let's encrypt the string ``RSA''.  Since we have such a small modulus, we will encrypt each character separately.  Encoded as a null terminated UTF-8 string, ``RSA'' is represented by the numbers $82, 83, 65, 0$.  Now we calculate each of these numbers raised to the $3$ modulo $187$ giving $92, 128, 109, 0$, our cyphertext.  We can then raise each of these to the $107$ modulo $187$ to recover $82, 83, 65, 0$
\end{proof}

\pagebreak
%%%Exercise 5.7%%%%%%%%%%%%%%%%%%%%%%%%%%%%%%%%%%%%%%%%%%%%%%%%%%%%%%%%%%%%%%%%%%%%%
\begin{exer}
  You are a secret agent. An evil spy with shallow number theory
  skills uses the RSA Public Key Coding System in which the public
  modulus is $n=1537$, and the encoding exponent is $E=47$. You
  intercept one of the encoded secret messages being sent to the evil
  spy, namely the number $570$. Using your superior number theory
  skills, decode this message, thereby saving countless people from
  the fiendish plot of the evil spy.
\end{exer}
\begin{proof}[Solution]
  To begin, factor $n = 1537 = 29\cdot 53$.  Then we use this $p$, $q$ and $e$ to generate $d$.  We find $\phi(n) = 28 * 52 = 1456$, so we know that $47d + (-1456)y = 1$.  Then we preform the Extended Euclidean Algorithm:
  \begin{align*}
    47 &= 1\cdot -1456 + 1503 & 1503 &= 47 - 1 \cdot (-1456)\\
    -1456 &= (-1)\cdot 1503 + 47\\
    1503 &= 31 \cdot 47 + 46 & 46 &= 1503 - 31 \cdot 47\\
    47 &= 1 \cdot 46 + 1 & 1 &= 47 - 1 \cdot 46\\
  \end{align*}
  So we can now back substitute to see that:
  \begin{align*}
    1 &= 47 - 1\cdot 46\\
    1 &= 47 - 1\cdot (1503 - 31 \cdot 47) & 1 &= 32 \cdot 47 - 1503\\
    1 &= 32\cdot 47 - (47 - 1 \cdot (-1456)) & 1 &= 31\cdot 47 + -1\cdot -1456
  \end{align*}
  So the decrypting exponent $d$ is $31$.  If our cyphertext is $570$, we can recover the original message $m = 570^{31} \pmod{1537}$.  So the original message is $131$.
\end{proof}
\end{document}
