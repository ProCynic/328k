\documentclass[12pt,leqno]{article}

\usepackage{amsmath,amsfonts,amssymb,amscd,amsthm,amsbsy,upref}


\textheight=8.5truein
\textwidth=6.0truein
\hoffset=-.5truein
\voffset=-.5truein
\numberwithin{equation}{section}
\pagestyle{headings}
\footskip=36pt


\swapnumbers
\newtheorem{thm}{Theorem}[section]
\newtheorem{hthm}[thm]{*Theorem}
\newtheorem{lem}[thm]{Lemma}
\newtheorem{cor}[thm]{Corollary}
\newtheorem{prop}[thm]{Proposition}
\newtheorem{con}[thm]{Conjecture}
\newtheorem{exer}[thm]{Exercise}
\newtheorem{bpe}[thm]{Blank Paper Exercise}
\newtheorem{apex}[thm]{Applications Exercise}
\newtheorem{ques}[thm]{Question}
\newtheorem{scho}[thm]{Scholium}
\newtheorem*{Exthm}{Example Theorem}
\newtheorem*{Thm}{Theorem}
\newtheorem*{Con}{Conjecture}
\newtheorem*{Axiom}{Axiom}



\theoremstyle{definition}
\newtheorem*{Ex}{Example}
\newtheorem*{Def}{Definition}


\newcommand{\lcm}{\operatorname{lcm}}
\newcommand{\ord}{\operatorname{ord}}
\def\pfrac#1#2{{\left(\frac{#1}{#2}\right)}}


\makeindex

\begin{document}




\thispagestyle{plain}
\begin{flushright}
\large{\textbf{Geoffrey Parker\\
HW 19: 3.27-3.29\\
M328K \\
April 3th, 2012 \\}}
\end{flushright}

%\maketitle
\markboth{}{} \setcounter{section}{0} \baselineskip=18pt

\setcounter{tocdepth}{4}


%%%%%%%%This is where you can change the numbering to match the problem you
%%%%%%%%are on.  Set the section to  the current chapter.

\setcounter{section}{3}

%%%%%%%%%Now, set the theorem number to one less than the first theorem in
%%%%%%%%%the assignment.
\setcounter{thm}{26}

%%%Theorem 3.27%%%%%%%%%%%%%%%%%%%%%%%%%%%%%%%%%%%%%%%%%%%%%%%%%%%%%%%%%%%%%%%%%%%%%
\begin{thm}
Let $a$, $b$, $m$, and $n$ be integers with $m > 0$ and $n > 0$.
Then the system
\[ \begin{array}{l}
x \equiv a \pmod{n} \\
x \equiv b \pmod{m} \end{array} \] has a solution if and only if
$(n, m)| a - b$.
\end{thm}

\begin{proof}[Proof]
Let $a$, $b$, $m$, and $n$ be integers with $m > 0$ and $n > 0$.

First assume that the system 
\[ \begin{array}{l}
x \equiv a \pmod{n} \\
x \equiv b \pmod{m} \end{array} \] has a solution $x$.  So by definition of congruence mod n we have $n \mid x - a$ and $m \mid x - b$.  And since $(n, m)$ divides both $n$ and $m$, we have $(n, m) \mid x - a$ and $(n, m) \mid x - b$.  Therefore by theorem 1.2 $(n, m) \mid x - b - (x - a)$, or $(n, m) \mid a - b$.

Now assume that $(n, m) \mid a - b$. Let $j$ and $k$ be integers such that $jn + km = (n, m)$. //TODO: finish
\end{proof}


%%%Theorem 3.28%%%%%%%%%%%%%%%%%%%%%%%%%%%%%%%%%%%%%%%%%%%%%%%%%%%%%%%%%%%%%%%%%%%%%
\begin{thm}
Let $a$, $b$, $m$, and $n$ be integers with $m > 0$, $n > 0$, and
$(m,n)=1$.  Then the system
\[ \begin{array}{l}
x \equiv a \pmod{n} \\
x \equiv b \pmod{m} \end{array} \] has a unique solution modulo
$mn$.
\end{thm}

\begin{proof}[Proof]
Let $a$, $b$, $m$, and $n$ be integers with $m > 0$, $n > 0$, and $(m,n)=1$.  Since $1 \mid a - b$, then by theorem 3.27 there must be a solution to the system
\[ \begin{array}{l}
x \equiv a \pmod{n} \\
x \equiv b \pmod{m} \end{array} \]
Let $x_0$ be any solution to this system.  Let $x_0'$ be the integer in the canonical complete residue system $mn$ such that $x_0 \equiv x_0' \pmod{mn}$. Let $x_1$ be any other solution to the system.  Let $x_1'$ be the integer in the canonical complete residue system $mn$ such that $x_1 \equiv x_1' \pmod{mn}$.  We will show that $x_0' = x_1'$.

Since $x_0$ and $x_1$ are both solutions to the system of equations, we know that $n \mid x_0 - a$ and $n \mid x_1 - a$, so by theorem 1.2 $n \mid (x_0 - a) - (x_1 - a)$ or $n \mid x_0 - x_1$.  Similarly, $m \mid x_0 - x_1$.  By theorem 1.42, this means that $nm \mid x_0 - x_1$.  By the definition of congruence mod n, $x_0 \equiv x_1 \pmod{nm}$.  So by theorem 1.11 $x_0 \equiv x_1' \pmod{nm}$.  And because $x_0'0$ and $x_1'$ are members of the canonical complete residue system mod $mn$, and $x_0$ is congruent to both of them modulo $mn$, it must be that $x_0' = x_1'$.
\end{proof}


%%%Theorem 3.29%%%%%%%%%%%%%%%%%%%%%%%%%%%%%%%%%%%%%%%%%%%%%%%%%%%%%%%%%%%%%%%%%%%%%
\begin{thm}[Chinese Remainder Theorem]
Suppose $n_1, n_2, \hdots, n_L$ are positive integers that are
pairwise relatively prime, that is, $(n_i, n_j)=1$ for $i\neq j$,
$1\leq i, j \leq L$.  Then the system of $L$ congruences
\[ \begin{array}{c}
x \equiv a_1 \pmod{n_1} \\
x \equiv a_2 \pmod{n_2} \\
\vdots \\
x \equiv a_L \pmod{n_L} \end{array} \] has a unique solution modulo
the product $n_1 n_2  n_3 \cdots n_L$.
\end{thm}

\begin{proof}[Proof]
Suppose $n_1, n_2, \hdots, n_L$ are positive integers that are pairwise relatively prime.  We will show by induction that the system $L$ congruences
\[ \begin{array}{c}
x \equiv a_1 \pmod{n_1} \\
x \equiv a_2 \pmod{n_2} \\
\vdots \\
x \equiv a_L \pmod{n_L} \end{array} \]
has a unique solution modulo the product $n_1 n_2 \cdots n_L$.  As our basecase, suppose $L = 2$.  In this case, because $(n_1, n_2) = 1$, theorem 3.28 says that there is a unique solution to the system of equations modulo $n_1 n_2$.  As our induction hypothesis, assume that there exists some $k \geq 2$ such that a system of $k$ equations will have $x'$, a unique solution modulo $n_1 n_2 \cdots n_k$.  Since all the $n$'s are pairwise coprime, then by lemma 1 $(n_{k+1}, n_1 n_2 \cdots n_k) = 1$.  //TODO: show that a solution $x_0$ to $k+1$ equations exists.  Let $x_1$ be any other solution to the system of $k+1$ equations.  For each integer $i$ from 1 to $k+1$, because $x_0 \equiv a_i \pmod{n_i}$ and $x_1 \equiv a_i \pmod{n_i}$, by theorem 1.11 $x_0 \equiv x_1 \pmod{n_i}$ and $n_i \mid x_0 - x_1$.  For convienence, let $s = n_1 n_2 \cdots n_{k+1}$.  So by lemma 2 $s \mid x_0 - x_1$ and $x_0 \equiv x_1 \pmod{s}$.  Let $x_0'$ and $x_1'$ be elements of the canonical complete residue set modulo $s$ such that $x_0 \equiv x_0' \pmod{s}$ and $x_1 \equiv x_1' \pmod{s}$.  By theorem 1.11 $x_0 \equiv x_1' \pmod{s}$.  Therefore $x_0' = x_1'$ and there is exactly one solution to the system of equations modulo s.


Lemma 1: Let $p$ be an integer and $n_1, n_2, \ldots , n_m$ be integers which are pairwise relativily prime.  Also, let $p$ be coprime with every $n_i$.  We will show that $(p, n_1 n_2 \cdots n_m) = 1$.  This will be a proof by induction.  As a base case, let $m = 1$.  So $(p, n_1) = 1$ by definition.  Our induction hypothesis is that there exists some $k \geq 1$ such that $(p, n_1 n_2 \cdots n_k) = 1$.  By definition, $(p, n_{k+1}) = 1$, so by theorem 1.43 $(p, n_1 n_2 \cdots n_{k+1}) = 1$. 


Lemma 2: Let $n_1, n_2, \ldots, n_m$ be integers which are pairwise relativley prime.  Let $x$ and $y$ be integers with $n_i \mid x - y$ for each $n_i$.  We will show by induction that $n_1 n_2 \cdots n_m \mid x - y$.  As our base case, if $m = 1$, then $n_1 \mid x - y$ by definition.  Our induction hypothesis is that there exists an integer $k \geq 1$ such that $n_1 n_2 \cdots n_k \mid x - y$.  Since $n_{k+1} \mid x - y$, then by theorem 1.42 $n_1 n_2 \cdots n_{k+1} \mid x - y$.
\end{proof}
\end{document}