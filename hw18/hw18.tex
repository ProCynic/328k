\documentclass[12pt,leqno]{article}

\usepackage{amsmath,amsfonts,amssymb,amscd,amsthm,amsbsy,upref}


\textheight=8.5truein
\textwidth=6.0truein
\hoffset=-.5truein
\voffset=-.5truein
\numberwithin{equation}{section}
\pagestyle{headings}
\footskip=36pt


\swapnumbers
\newtheorem{thm}{Theorem}[section]
\newtheorem{hthm}[thm]{*Theorem}
\newtheorem{lem}[thm]{Lemma}
\newtheorem{cor}[thm]{Corollary}
\newtheorem{prop}[thm]{Proposition}
\newtheorem{con}[thm]{Conjecture}
\newtheorem{exer}[thm]{Exercise}
\newtheorem{bpe}[thm]{Blank Paper Exercise}
\newtheorem{apex}[thm]{Applications Exercise}
\newtheorem{ques}[thm]{Question}
\newtheorem{scho}[thm]{Scholium}
\newtheorem*{Exthm}{Example Theorem}
\newtheorem*{Thm}{Theorem}
\newtheorem*{Con}{Conjecture}
\newtheorem*{Axiom}{Axiom}



\theoremstyle{definition}
\newtheorem*{Ex}{Example}
\newtheorem*{Def}{Definition}


\newcommand{\lcm}{\operatorname{lcm}}
\newcommand{\ord}{\operatorname{ord}}
\def\pfrac#1#2{{\left(\frac{#1}{#2}\right)}}


\makeindex

\begin{document}




\thispagestyle{plain}
\begin{flushright}
\large{\textbf{TYPE YOUR NAME HERE \\
HW 18: 3.23-3.25\\
M328K \\
March 29th, 2012 \\}}
\end{flushright}

%\maketitle
\markboth{}{} \setcounter{section}{0} \baselineskip=18pt

\setcounter{tocdepth}{4}


%%%%%%%%This is where you can change the numbering to match the problem you
%%%%%%%%are on.  Set the section to  the current chapter.

\setcounter{section}{3}

%%%%%%%%%Now, set the theorem number to one less than the first theorem in
%%%%%%%%%the assignment.
\setcounter{thm}{22}

%%%Question 3.23%%%%%%%%%%%%%%%%%%%%%%%%%%%%%%%%%%%%%%%%%%%%%%%%%%%%%%%%%%%%%%%%%%%%%
\begin{ques}
Let $a$, $b$, and $n$ be integers with $n > 0$.  How many solutions
are there to the linear congruence $ax \equiv b \pmod{n}$ in the
canonical complete residue system modulo $n$? Can you describe a
technique to find them?
\end{ques}

\begin{proof}[Solution]
Type your solution here!
\end{proof}


%%%Theorem 3.24%%%%%%%%%%%%%%%%%%%%%%%%%%%%%%%%%%%%%%%%%%%%%%%%%%%%%%%%%%%%%%%%%%%%%
\begin{thm}
Let $a$, $b$, and $n$ be integers with $n > 0$. Then,
\begin{enumerate}
\item The congruence $ax \equiv b \pmod{n}$ is solvable in integers if
and only if $(a,n)|b$.
\item If $x_0$ is a solution to the congruence $ax \equiv b\pmod{n}$,
then all solutions are given by \[ x_0 + \left(\frac{n}{(a,n)}\cdot
m\right)\pmod{n}\] for $m = 0,\ 1,\ 2,\ \hdots\ $, $(a, n)-1$.
\item If $ax \equiv b \pmod{n}$ has a solution, then there are exactly
$(a, n)$ solutions in the canonical complete residue system modulo
$n$.
\end{enumerate}
\end{thm}

\begin{proof}[Proof]
Type your proof here!
\end{proof}




%%%Exercise 3.25%%%%%%%%%%%%%%%%%%%%%%%%%%%%%%%%%%%%%%%%%%%%%%%%%%%%%%%%%%%%%%%%%%%%%
\begin{exer}
A band of $17$ pirates stole a sack of gold coins.  When they tried
to divide the fortune into equal portions, $3$ coins remained.  In
the ensuing brawl over who should get the extra coins, one pirate
was killed.  The coins were redistributed, but this time an equal
division left $10$ coins.  Again they fought about who should get
the remaining coins and another pirate was killed.  Now,
fortunately, the coins could be divided evenly among the surviving
$15$ pirates.  What was the fewest number of coins that could have
been in the sack?
\end{exer}

\begin{proof}[Solution]
Type your solution here!
\end{proof}
\end{document}