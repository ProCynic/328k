\documentclass[12pt,leqno]{article}

\usepackage{amsmath,amsfonts,amssymb,amscd,amsthm,amsbsy,upref, enumerate}


\textheight=8.5truein
\textwidth=6.0truein
\hoffset=-.5truein
\voffset=-.5truein
\numberwithin{equation}{section}
\pagestyle{headings}
\footskip=36pt


\swapnumbers
\newtheorem{thm}{Theorem}[section]
\newtheorem{hthm}[thm]{*Theorem}
\newtheorem{lem}[thm]{Lemma}
\newtheorem{cor}[thm]{Corollary}
\newtheorem{prop}[thm]{Proposition}
\newtheorem{con}[thm]{Conjecture}
\newtheorem{exer}[thm]{Exercise}
\newtheorem{bpe}[thm]{Blank Paper Exercise}
\newtheorem{apex}[thm]{Applications Exercise}
\newtheorem{ques}[thm]{Question}
\newtheorem{scho}[thm]{Scholium}
\newtheorem*{Exthm}{Example Theorem}
\newtheorem*{Thm}{Theorem}
\newtheorem*{Con}{Conjecture}
\newtheorem*{Axiom}{Axiom}



\theoremstyle{definition}
\newtheorem*{Ex}{Example}
\newtheorem*{Def}{Definition}


\newcommand{\lcm}{\operatorname{lcm}}
\newcommand{\ord}{\operatorname{ord}}
\def\pfrac#1#2{{\left(\frac{#1}{#2}\right)}}


\makeindex

\begin{document}




\thispagestyle{plain}
\begin{flushright}
\large{\textbf{Geoffrey Parker - grp352 \\
HW 18: 3.23-3.25\\
M328K \\
March 29th, 2012 \\}}
\end{flushright}

%\maketitle
\markboth{}{} \setcounter{section}{0} \baselineskip=18pt

\setcounter{tocdepth}{4}


%%%%%%%%This is where you can change the numbering to match the problem you
%%%%%%%%are on.  Set the section to  the current chapter.

\setcounter{section}{3}

%%%%%%%%%Now, set the theorem number to one less than the first theorem in
%%%%%%%%%the assignment.
\setcounter{thm}{22}

%%%Question 3.23%%%%%%%%%%%%%%%%%%%%%%%%%%%%%%%%%%%%%%%%%%%%%%%%%%%%%%%%%%%%%%%%%%%%%
\begin{ques}
Let $a$, $b$, and $n$ be integers with $n > 0$.  How many solutions
are there to the linear congruence $ax \equiv b \pmod{n}$ in the
canonical complete residue system modulo $n$? Can you describe a
technique to find them?
\end{ques}

\begin{proof}[Solution]
Find the two integers $q$ and $r$ such that $b = (a, n)q + r$.  If $r$ is not 0, theorem 3.20 says there are no solutions.  If $r$ is 0, then $b = (a, n)q$.\\

Use the Euclidean Algorithm to find two integers $x$ and $y$ such that $ax + ny = (a, n)$.  Then $ax - (a, n) = n(-y)$ and $axq - (a, n)q = n(-yq)$. Equivalently, $a(xq) - b = n(-yq)$ so $n \mid a(xq) - b$.  Therefore $a(xq) \equiv b \pmod{n}$.  Now use the division algorithm find $z$ such that $xq \equiv z \pmod{n}$.  Now $z$ is one solution in the canonical complete residue system modulo $n$.  To find the others, repeatedly add multiples of $n \over (a, n)$ until you have all $(a, n)$ solutions.
\end{proof}

%%%Theorem 3.24%%%%%%%%%%%%%%%%%%%%%%%%%%%%%%%%%%%%%%%%%%%%%%%%%%%%%%%%%%%%%%%%%%%%%
\begin{thm}
Let $a$, $b$, and $n$ be integers with $n > 0$. Then,
\begin{enumerate}
\item The congruence $ax \equiv b \pmod{n}$ is solvable in integers if
and only if $(a,n)|b$.
\item If $x_0$ is a solution to the congruence $ax \equiv b\pmod{n}$,
then all solutions are given by \[ x_0 + \left(\frac{n}{(a,n)}\cdot
m\right)\pmod{n}\] for $m = 0,\ 1,\ 2,\ \hdots\ $, $(a, n)-1$.
\item If $ax \equiv b \pmod{n}$ has a solution, then there are exactly
$(a, n)$ solutions in the canonical complete residue system modulo
$n$.
\end{enumerate}
\end{thm}

\pagebreak
\begin{proof}[Proof]
Let $a$, $b$, and $n$ be integers with $n > 0$.  By theorem 3.20, we know that the congruence $ax \equiv b \pmod{n}$ is solvable in integers if and only if $(a,n)|b$.  

Let $x_0$ be a solution to the congruence $ax \equiv b\pmod{n}$. Let $T$ be the canonical complete residue system modulo $(a, n)$.  Let $S$ be the set of integers $z$ such that $z$ is in the canonical complete residue system modulo $n$ and $x_0 + \left(\frac{n}{(a,n)}\cdot m\right) \equiv z \pmod{n}$ for any integer $m$ which is an element of $T$.  We will show both that every element of $S$ is a solution to $ax \equiv b\pmod{n}$ and that all solutions to $ax \equiv b\pmod{n}$ are elements of $S$.  Let $s$ be an arbitrary element of $S$.  So $n \mid \left(\frac{n}{(a,n)}\cdot t\right) - s$, or $x_0 + \left(\frac{n}{(a,n)}\cdot t\right) = cn + s$ for some $t \in T$ and some integer $c$.  And $s = x_0 + n\left(\frac{t}{(a,n)} - c\right)$.  Since $n \mid ax_0 - b$ and $n \mid n\left(\frac{t}{(a,n)} - c\right)$, by theorems 1.6 and 1.1 $n \mid ax_0 + an\left(\frac{t}{(a,n)} - c\right) - b$.  Rewriting gives us $n \mid a\left(x_0 + n\left(\frac{t}{(a,n)} - c\right)\right) - b$.  Substiuting in $s$ gives us $n \mid as - b$, so $as \equiv b \pmod{n}$.  Therefore every element of $S$ is a solution to $ax \equiv b \pmod{n}$.

Now consider an arbitrary integer $p$ such that $ap \equiv b \pmod{n}$.  Let $q$ be the integer such that $p = x_0 + \frac{qn}{(a, n)}$.  Since $T$ is a complete residue system modulo $(a, n)$, there exists some $q' \in T$ such that $q \equiv q' \pmod{(a, n)}$.  Let $p' = x_0 + \frac{q'n}{(a, n)}$.  So $p - p' = x_0 + \frac{qn}{(a, n)} - \left(x_0 + \frac{q'n}{(a, n)}\right) = n \cdot \frac{q - q'}{(a, n)}$.  Therefore $n \mid p - p'$ and $p \equiv p' \pmod{n}$.  Now let $r$ be the element of the canonical complete residue system modulo $n$ such that $p' \equiv r \pmod{n}$.  By the definitions of $p'$ and the set $S$, $r$ must be an element of $S$.  However, since $p \equiv p' \pmod{n}$, then by theorem 1.11 $p \equiv r \pmod{n}$.  So we have taken an arbitrary solution to the congruence $ax \equiv b \pmod{n}$ and shown that it is equivalent to an element of $S$.  Therefore all solutions are given by $S$.

Since all solutions in the canonical complete residue system mod $n$ are given by $S$, the number of solutions is the cardinality of $S$.  And since $S$ is defined to have exactly one elemnent for every element of the canoninical complete residue system modulo $(a, n)$, then if there is 1 solution, there are exactly $(a, n)$ solutions.
\end{proof}



\pagebreak
%%%Exercise 3.25%%%%%%%%%%%%%%%%%%%%%%%%%%%%%%%%%%%%%%%%%%%%%%%%%%%%%%%%%%%%%%%%%%%%%
\begin{exer}
A band of $17$ pirates stole a sack of gold coins.  When they tried
to divide the fortune into equal portions, $3$ coins remained.  In
the ensuing brawl over who should get the extra coins, one pirate
was killed.  The coins were redistributed, but this time an equal
division left $10$ coins.  Again they fought about who should get
the remaining coins and another pirate was killed.  Now,
fortunately, the coins could be divided evenly among the surviving
$15$ pirates.  What was the fewest number of coins that could have
been in the sack?
\end{exer}

\begin{proof}[Solution]
Let $n$ be the number of coins in the sack.  Our pirate story gives us these three equations:\\
$n \equiv 3 \pmod{17}$\\
$n \equiv 10 \pmod{16}$\\
$n \equiv 0 \pmod{15}$ \\

The answer is 3930.

\end{proof}
\end{document}