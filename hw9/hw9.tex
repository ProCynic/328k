\documentclass[12pt,leqno]{article}

\usepackage{amsmath,amsfonts,amssymb,amscd,amsthm,amsbsy,upref}


\textheight=8.5truein
\textwidth=6.0truein
\hoffset=-.5truein
\voffset=-.5truein
\numberwithin{equation}{section}
\pagestyle{headings}
\footskip=36pt


\swapnumbers
\newtheorem{thm}{Theorem}[section]
\newtheorem{hthm}[thm]{*Theorem}
\newtheorem{lem}[thm]{Lemma}
\newtheorem{cor}[thm]{Corollary}
\newtheorem{prop}[thm]{Proposition}
\newtheorem{con}[thm]{Conjecture}
\newtheorem{exer}[thm]{Exercise}
\newtheorem{bpe}[thm]{Blank Paper Exercise}
\newtheorem{apex}[thm]{Applications Exercise}
\newtheorem{ques}[thm]{Question}
\newtheorem{scho}[thm]{Scholium}
\newtheorem*{Exthm}{Example Theorem}
\newtheorem*{Thm}{Theorem}
\newtheorem*{Con}{Conjecture}
\newtheorem*{Axiom}{Axiom}



\theoremstyle{definition}
\newtheorem*{Ex}{Example}
\newtheorem*{Def}{Definition}


\newcommand{\lcm}{\operatorname{lcm}}
\newcommand{\ord}{\operatorname{ord}}
\def\pfrac#1#2{{\left(\frac{#1}{#2}\right)}}


\makeindex

\begin{document}




\thispagestyle{plain}
\begin{flushright}
\large{\textbf{Geoffrey Parker - grp352 \\
HW 9: 2.25-2.31\\
M328K \\
February 16th, 2012 \\}}
\end{flushright}

%\maketitle
\markboth{}{} \setcounter{section}{0} \baselineskip=18pt

\setcounter{tocdepth}{4}


%%%%%%%%This is where you can change the numbering to match the problem you
%%%%%%%%are on.  Set the section to  the current chapter.

\setcounter{section}{2}

%%%%%%%%%Now, set the theorem number to one less than the first theorem in
%%%%%%%%%the assignment.
\setcounter{thm}{24}

%%%Theorem 2.25%%%%%%%%%%%%%%%%%%%%%%%%%%%%%%%%%%%%%%%%%%%%%%%%%%%%%%%%%%%%%%%%%%%%%

\begin{thm}
Let $a$, $b$, and $n$ be integers.  If $a \mid n$, $b \mid n$, and $(a, b) = 1$, then $ab \mid n$.
\end{thm}

\begin{proof}[Proof]
Let $a$, $b$, and $n$ be integers where $a \mid n$, $b \mid n$, and $(a, b) = 1$. We will show that $ab \mid n$.  
Let $p_1^{r_1}p_2^{r_2}\cdots p_m^{r_m}$ be the unique prime factorization of $a$, $q_1^{r_1}q_2^{r_2}\cdots q_s^{t_s}$ be the unique prime factorization of $b$, and $x_1^{y_1}x_2^{y_2}\cdots x_z^{y_z}$ be the unique prime factorization of $n$.  Now, since $a$ and $b$ are coprime, there cannot be any $i$ and $j$ such that \\$p_i = q_j$, because if there were, $(a, b)$ would be $p_i$.  This means that the unique prime factorization of $ab$ is $p_1^{r_1}q_1^{r_1}p_2^{r_2}q_2^{r_2}\cdots p_m^{r_m}q_s^{t_s}$, with $s$ and $m$ possibly being different and none of these terms combining.  However, by theorem 2.12, we have that for all integers $i \leq m$ there exists a $u \leq y$ such that $p_i = x_u$ and $r_i \leq y_u$, and for all integers $j \leq s$ there exists an integer $v \leq z$ such that $q_j = x_v$ and $t_j \leq y_v$.  Therefore given the unique prime factorization of $ab$ and theorem 2.12, this means that $ab \mid n$.
\\ \\
OR
\\ \\
This is just theorem 1.42, restated exactly.
\end{proof}


%%%Theorem 2.26%%%%%%%%%%%%%%%%%%%%%%%%%%%%%%%%%%%%%%%%%%%%%%%%%%%%%%%%%%%%%%%%%%%%%

\begin{thm}
Let $p$ be prime and let $a$ be an integer.  Then $p$ does not divide $a$ if and only if $(a, p) = 1$.
\end{thm}

\begin{proof}[Proof]
Let $p$ be prime and let $a$ be an integer.\\
First, assume that $p \nmid a$.  Since $p$ is prime, the only divisors that $p$ has are 1 and itself.  And since $p \nmid a$, the only divisor $p$ and $a$ could possibly share is 1.  Therefore $(a, p) = 1$.\\
Now, assume that $(a, p) = 1$.  Since all numbers divide them selves, if $p \mid a$, then $(a, p)$ would be $p$.  Therefore $p \nmid a$.
\end{proof}

\pagebreak
%%%Theorem 2.27%%%%%%%%%%%%%%%%%%%%%%%%%%%%%%%%%%%%%%%%%%%%%%%%%%%%%%%%%%%%%%%%%%%%%

\begin{thm}
Let $p$ be prime and let $a$ and $b$ be integers. If $p \mid ab$, then $p\mid a$ or $p \mid b$.
\end{thm}

\begin{proof}[Proof]
Let $p$ be prime and let $a$ and $b$ be integers with $p \mid ab$. We will show that $p\mid a$ or $p \mid b$. Assume by way of contradiction that $p\nmid a$ and $p \nmid b$.  Since $p$ is prime, the unique prime factorization of $p$ is just $p$.  By theorem 2.12 and our assumption, this means that $p$ is not a member of the unique prime factorizations of either $a$ or $b$.  This means in turn that $p$ is not a member of the unique prime factorization of $ab$.  And by theorem 2.12, this implies that $p \nmid ab$, so we have our contradiction.  Therefore $p\mid a$ or $p \mid b$.
\end{proof}

%%%Theorem 2.28%%%%%%%%%%%%%%%%%%%%%%%%%%%%%%%%%%%%%%%%%%%%%%%%%%%%%%%%%%%%%%%%%%%%%

\begin{thm}
Let $a$, $b$, and $c$ be integers.  If $(b, c) = 1$, then $(a, bc) = (a, b) \cdot (a, c)$.
\end{thm}

\begin{proof}[Solution]
Let $a$, $b$, and $c$ be integers with $(b, c) = 1$.  By definition, $(a, b) \mid a$, and by theorem 1.6 $(a, b) \mid bc$.  Similarly, $(a, c) \mid a$ and $(a, c) \mid bc$.  Now by definition of gcd, $((a, b), (a, c))$ will be the greatest of all the divisors common to $a$, $b$, and $c$, which since $(b, c) = 1$, is 1.  So $(a, b) \cdot (a, c)$ is a divisor of both $a$ and $bc$.  //TODO Finish
\end{proof}

%%%Theorem 2.29%%%%%%%%%%%%%%%%%%%%%%%%%%%%%%%%%%%%%%%%%%%%%%%%%%%%%%%%%%%%%%%%%%%%%

\begin{thm}
Let $a$, $b$ and $c$ be integers.  If $(a, b) = 1$ and $(a, c) = 1$, then \\$(a, bc) = 1$.
\end{thm}

\begin{proof}[Solution]
Let $a$, $b$ and $c$ be integers with $(a, b) = 1$ and $(a, c) = 1$. We will show that $(a, bc) = 1$.  Let $A$ be the set of primes in the unique prime factorization of $a$, $B$ for $b$ and $C$ for c.  So we can say that there is no prime $p$ which is a common element of either the sets $A$ and $B$ or the sets $A$ and $C$, because if there was, we would have $(a, b) = p$ or $(a, c) = p$.  Therefore $A \cap (B \cup C)$ is empty.  Now consider any integer $k$ which is a common divisor of $a$ and $bc$.  Let $K$ be the set of primes in the unique prime factorization of $k$.  By theorem 2.12, since $k \mid a$ and $k \mid bc$, every element of $K$ must be an element of both $A$ and $(B \cup C)$.  That is $K$ is a subset of $A \cap (B \cup C)$.  However, $A \cap (B \cup C)$ is empty, so $K$ must also be empty.  Therefore $k$ must be 1, that is the only common divisor of $a$ and $bc$ is 1, so $(a, bc) = 1$.
\end{proof}

%%%Theorem 2.30%%%%%%%%%%%%%%%%%%%%%%%%%%%%%%%%%%%%%%%%%%%%%%%%%%%%%%%%%%%%%%%%%%%%%

\begin{thm}
Let $a$ and $b$ be integers.  If $(a, b) = d$, then $({a \over d}, {b \over d}) = 1$.
\end{thm}

\begin{proof}[Proof]
Let $A$, $B$, and $D$  be prime factorizations of $a$, $b$, and $d$ respectively.  Theorem 2.12 assures us that there is no prime $p$ that is a part of $D$ but now $A$ or $B$.  So $a \over d$ will be $A \over D$, which can be found by subtracting expontents of common primes.  The same is true for $b$ and $d$.  //TODO finish

%Now consider some integer $k$ such that $k \mid a$ and $k \mid b$ and its prime factorization $K$.  



%Let $a$ and $b$ be integers, with $(a, b) = d$. We will show that $({a \over d}, {b \over d}) = 1$.  Let $k$ be any integer such that $k\mid a$ and $k\mid b$.  Let $p_1^{r_1}p_2^{r_2}\cdots p_m^{r_m}$ be the unique prime factorization of $a$, $q_1^{r_1}q_2^{r_2}\cdots q_s^{t_s}$ be the unique prime factorization of $b$, $x_1^{y_1}x_2^{y_2}\cdots x_z^{y_z}$ be the unique prime factorization of $d$, and $g_1^{h_1}g_2^{h_2}\cdots g_n^{h_n}$ be the unique prime factorization of $k$.  Now, because $d \mid a$ and $k \mid a$, by theorem 2.12 we have that for all integers $i \leq z$, there exists a $j \leq m$ and a $u \leq n$ such that $x_i = p_j = g_u$ and $y_i \leq r_j$ and $h_u \leq r_g$.  And because $k \leq d$m $h_u \leq y_i$  So the prime factorization of ${a \over d}$ will be a product of all the terms $p_j^{r_j - y_i}$ and all the $p_j^{r_j}$ where $p_j$ does not correspond to some $x_i$.  //TODO
\end{proof}

%%%Theorem 2.31%%%%%%%%%%%%%%%%%%%%%%%%%%%%%%%%%%%%%%%%%%%%%%%%%%%%%%%%%%%%%%%%%%%%%

\begin{thm}
Let $a$, $b$, $u$, and $v$ be integers.  If $(a, b) = 1$ and $u \mid a$ and $v \mid b$, then $(u, v) = 1$.
\end{thm}

\begin{proof}[Proof]
Let $a$, $b$, $u$, and $v$ be integers with $(a, b) = 1$ and $u \mid a$ and $v \mid b$. We will show $(u, v) = 1$. Let $A$, $B$, $U$, and $V$ represent the sets of primes in the unique prime factorizations of $a$, $b$, $u$, and $v$ respectively.  Let $p$ be any element of $A$.  If $p$ was also in $B$, then $(a, b)$ would be greater than or equal to $p$.  So $A \cap B$ is empty.  However, by theorem 2.12, $U$ is a subset of $A$ and $V$ is a subset of $B$, so $U \cap V$ must also be empty.  Therefore $(u, v) = 1$.
\end{proof}
\end{document}