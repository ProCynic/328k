\documentclass[12pt,leqno]{article}

\usepackage{amsmath,amsfonts,amssymb,amscd,amsthm,amsbsy,upref}


\textheight=8.5truein
\textwidth=6.0truein
\hoffset=-.5truein
\voffset=-.5truein
\numberwithin{equation}{section}
\pagestyle{headings}
\footskip=36pt


\swapnumbers
\newtheorem{thm}{Theorem}[section]
\newtheorem{hthm}[thm]{*Theorem}
\newtheorem{lem}[thm]{Lemma}
\newtheorem{cor}[thm]{Corollary}
\newtheorem{prop}[thm]{Proposition}
\newtheorem{con}[thm]{Conjecture}
\newtheorem{exer}[thm]{Exercise}
\newtheorem{bpe}[thm]{Blank Paper Exercise}
\newtheorem{apex}[thm]{Applications Exercise}
\newtheorem{ques}[thm]{Question}
\newtheorem{scho}[thm]{Scholium}
\newtheorem*{Exthm}{Example Theorem}
\newtheorem*{Thm}{Theorem}
\newtheorem*{Con}{Conjecture}
\newtheorem*{Axiom}{Axiom}



\theoremstyle{definition}
\newtheorem*{Ex}{Example}
\newtheorem*{Def}{Definition}


\newcommand{\lcm}{\operatorname{lcm}}
\newcommand{\ord}{\operatorname{ord}}
\def\pfrac#1#2{{\left(\frac{#1}{#2}\right)}}


\makeindex

\begin{document}




\thispagestyle{plain}
\begin{flushright}
\large{\textbf{TYPE YOUR NAME HERE \\
HW 15: 3.8-3.10\\
M328K \\
March 20th, 2012 \\}}
\end{flushright}

%\maketitle
\markboth{}{} \setcounter{section}{0} \baselineskip=18pt

\setcounter{tocdepth}{4}


%%%%%%%%This is where you can change the numbering to match the problem you
%%%%%%%%are on.  Set the section to  the current chapter.

\setcounter{section}{3}

%%%%%%%%%Now, set the theorem number to one less than the first theorem in
%%%%%%%%%the assignment.
\setcounter{thm}{7}

%%%Theorem 3.8%%%%%%%%%%%%%%%%%%%%%%%%%%%%%%%%%%%%%%%%%%%%%%%%%%%%%%%%%%%%%%%%%%%%%

\begin{thm}
Suppose $f(x) = a_nx^n + a_{n-1}x^{n-1} + \hdots + a_0$ is a
polynomial of degree $n>0$ with integer coefficients.  Let $a$, $b$,
and $m$ be integers with $m > 0$.  If $a \equiv b \pmod{m}$,
then~$f(a) \equiv f(b) \pmod{m}$.
\end{thm}

\begin{proof}[Proof]
Suppose $f(x) = c_nx^n + c_{n-1}x^{n-1} + \hdots + c_0$ is a polynomial of degree $n>0$ with integer coefficients.  Let $a$, $b$, and $m$ be integers with $m > 0$ and $a \equiv b \pmod{m}$.  We will show $f(a) \equiv f(b) \pmod{m}$.  First, since $a \equiv b \pmod{m}$, we know that $a^k \equiv b^k \pmod{m}$ for any integer $k$ by theorem blah.  Combining with theorem blah, we can see that for each term in $f$, $c_{n-j}a^{n-j} \equiv c_{n-j}b^{n-j} \pmod{m}$, where $j$ is some integer such that $0 \leq j \leq n$.  Now, by theorem blah, since $c_na^n \equiv c_nb^n \pmod{m}$ and $c_{n-1}a^{n-1} \equiv c_{n-1}b^{n-1} \pmod{m}$, we have that $c_na^n + c_{n-1}a^{n-1} \equiv c_nb^n + c_{n-1}b^{n-1} \pmod{m}$.  By theorem blah again, $c_na^n + c_{n-1}a^{n-1} + c_{n-2}a^{n-2} \equiv c_nb^n + c_{n-1}b^{n-1} + c_{n-2}b^{n-2} \pmod{m}$.  We can apply this same logic for each term in $f(x)$ to show that $f(a) \equiv f(b) \pmod{m}$.
\end{proof}

%%%Corollary 3.9%%%%%%%%%%%%%%%%%%%%%%%%%%%%%%%%%%%%%%%%%%%%%%%%%%%%%%%%%%%%%%%%%%%%%
\begin{cor}
Let the natural number $n$ be expressed in base $10$ as
\[n = a_k a_{k-1} \hdots a_1 a_0. \]
 Let $m = a_k + a_{k-1} + \hdots + a_1 + a_0$. Then $9|n$ if and only if $9|m$.
\end{cor}

\begin{proof}[Proof]
Let the natural number $n$ be expressed in base $10$ as
\[n = a_k a_{k-1} \hdots a_1 a_0. \]
Let $m = a_k + a_{k-1} + \hdots + a_1 + a_0$. We will show that $9 \mid n$ if and only if $9 \mid m$.\\

Let $f(x) = a_nx^n + a_{n-1}x^{n-1} + \hdots + a_0x^0$.  We can now express $n$ as $f(10)$ and $m$ as $f(1)$.  Since $10 - 1 = 9$, by definition of congruence mod n we have that $10 \equiv 1 \pmod{9}$.  So by theorem 3.8, $f(10) \equiv f(1) \pmod{9}$, or equivalently, $n \equiv m \pmod{9}$.  Therefore by definition of congruence mod n, $9 \mid n - m$.\\

First, assume that $9 \mid m$.  In this case, $9 \mid m$ and $9 \mid n-m$, so by theorem 1.1? $9 \mid m + (n-m)$, that is $9\mid n$.\\

Now assume that $9 \mid n$.  In this case, $9 \mid n$ and $9 \mid n-m$, so by theorem 1.2? $9 \mid n - (n-m)$, that is $9\mid m$.\\

Therefore $9|n$ if and only if $9|m$.
\end{proof}


%%%Corollary 3.10%%%%%%%%%%%%%%%%%%%%%%%%%%%%%%%%%%%%%%%%%%%%%%%%%%%%%%%%%%%%%%%%%%%%%
\begin{cor}
Let the natural number $n$ be expressed in base $10$ as
\[n = a_k a_{k-1} \hdots a_1 a_0. \]
If $m = a_k + a_{k-1} + \hdots + a_1 + a_0$. Then $3|n$ if and only
if $3|m$.
\end{cor}

\begin{proof}[Proof]
Let the natural number $n$ be expressed in base $10$ as
\[n = a_k a_{k-1} \hdots a_1 a_0. \]
If $m = a_k + a_{k-1} + \hdots + a_1 + a_0$. We will show that $3|n$ if and only if $3|m$.\\

Let $f(x) = a_nx^n + a_{n-1}x^{n-1} + \hdots + a_0x^0$.  We can now express $n$ as $f(10)$ and $m$ as $f(1)$.  Since $10 - 1 = 9$ and $3 \mid 9$, by definition of congruence mod n we have that $10 \equiv 1 \pmod{3}$.  So by theorem 3.8, $f(10) \equiv f(1) \pmod{3}$, or equivalently, $n \equiv m \pmod{3}$.  Therefore by definition of congruence mod n, $3 \mid n - m$.\\

First, assume that $3 \mid m$.  In this case, $3 \mid m$ and $3 \mid n-m$, so by theorem 1.1? $3 \mid m + (n-m)$, that is $3\mid n$.\\

Now assume that $3 \mid n$.  In this case, $3 \mid n$ and $3 \mid n-m$, so by theorem 1.2? $3 \mid n - (n-m)$, that is $3\mid m$.\\

Therefore $3|n$ if and only if $3|m$.
\end{proof}
\end{document}
