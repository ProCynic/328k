\documentclass[12pt,leqno]{article}

\usepackage{amsmath,amsfonts,amssymb,amscd,amsthm,amsbsy,upref}


\textheight=8.5truein
\textwidth=6.0truein
\hoffset=-.5truein
\voffset=-.5truein
\numberwithin{equation}{section}
\pagestyle{headings}
\footskip=36pt


\swapnumbers
\newtheorem{thm}{Theorem}[section]
\newtheorem{hthm}[thm]{*Theorem}
\newtheorem{lem}[thm]{Lemma}
\newtheorem{cor}[thm]{Corollary}
\newtheorem{prop}[thm]{Proposition}
\newtheorem{con}[thm]{Conjecture}
\newtheorem{exer}[thm]{Exercise}
\newtheorem{bpe}[thm]{Blank Paper Exercise}
\newtheorem{apex}[thm]{Applications Exercise}
\newtheorem{ques}[thm]{Question}
\newtheorem{scho}[thm]{Scholium}
\newtheorem*{Exthm}{Example Theorem}
\newtheorem*{Thm}{Theorem}
\newtheorem*{Con}{Conjecture}
\newtheorem*{Axiom}{Axiom}



\theoremstyle{definition}
\newtheorem*{Ex}{Example}
\newtheorem*{Def}{Definition}


\newcommand{\lcm}{\operatorname{lcm}}
\newcommand{\ord}{\operatorname{ord}}
\def\pfrac#1#2{{\left(\frac{#1}{#2}\right)}}


\makeindex

\begin{document}




\thispagestyle{plain}
\begin{flushright}
\large{\textbf{TYPE YOUR NAME HERE \\
HW 12: 2.42-2.44\\
M328K \\
March 1st, 2012 \\}}
\end{flushright}

%\maketitle
\markboth{}{} \setcounter{section}{0} \baselineskip=18pt

\setcounter{tocdepth}{4}


%%%%%%%%This is where you can change the numbering to match the problem you
%%%%%%%%are on.  Set the section to  the current chapter.

\setcounter{section}{2}

%%%%%%%%%Now, set the theorem number to one less than the first theorem in
%%%%%%%%%the assignment.
\setcounter{thm}{41}

%%%Theorem 2.42%%%%%%%%%%%%%%%%%%%%%%%%%%%%%%%%%%%%%%%%%%%%%%%%%%%%%%%%%%%%%%%%%%%%%

\begin{thm}
If $n$ is a natural number and $2^n - 1$ is prime, then $n$ must be
prime.
\end{thm}

\begin{proof}[Proof]
Type your proof here!
\end{proof}


%%%Theorem 2.43%%%%%%%%%%%%%%%%%%%%%%%%%%%%%%%%%%%%%%%%%%%%%%%%%%%%%%%%%%%%%%%%%%%%%

\begin{thm}
If $n$ is a natural number and $2^n+1$ is prime, then $n$ must be a
power of $2$.
\end{thm}

\begin{proof}[Proof]
Type your proof here!
\end{proof}


%%%Exercise 2.44%%%%%%%%%%%%%%%%%%%%%%%%%%%%%%%%%%%%%%%%%%%%%%%%%%%%%%%%%%%%%%%%%%%%%

\begin{exer}
Find the first few Mersenne primes and Fermat primes.
\end{exer}

\begin{proof}[Solution]
Type your solution here!
\end{proof}

\end{document}
