\documentclass[12pt,leqno]{article}

\usepackage{amsmath,amsfonts,amssymb,amscd,amsthm,amsbsy,upref}


\textheight=8.5truein
\textwidth=6.0truein
\hoffset=-.5truein
\voffset=-.5truein
\numberwithin{equation}{section}
\pagestyle{headings}
\footskip=36pt


\swapnumbers
\newtheorem{thm}{Theorem}[section]
\newtheorem{hthm}[thm]{*Theorem}
\newtheorem{lem}[thm]{Lemma}
\newtheorem{cor}[thm]{Corollary}
\newtheorem{prop}[thm]{Proposition}
\newtheorem{con}[thm]{Conjecture}
\newtheorem{exer}[thm]{Exercise}
\newtheorem{bpe}[thm]{Blank Paper Exercise}
\newtheorem{apex}[thm]{Applications Exercise}
\newtheorem{ques}[thm]{Question}
\newtheorem{scho}[thm]{Scholium}
\newtheorem*{Exthm}{Example Theorem}
\newtheorem*{Thm}{Theorem}
\newtheorem*{Con}{Conjecture}
\newtheorem*{Axiom}{Axiom}



\theoremstyle{definition}
\newtheorem*{Ex}{Example}
\newtheorem*{Def}{Definition}


\newcommand{\lcm}{\operatorname{lcm}}
\newcommand{\ord}{\operatorname{ord}}
\def\pfrac#1#2{{\left(\frac{#1}{#2}\right)}}


\makeindex

\begin{document}




\thispagestyle{plain}
\begin{flushright}
\large{\textbf{Geoffrey Parker - grp352 \\
HW 12: 2.42-2.44\\
M328K \\
March 1st, 2012 \\}}
\end{flushright}

%\maketitle
\markboth{}{} \setcounter{section}{0} \baselineskip=18pt

\setcounter{tocdepth}{4}


%%%%%%%%This is where you can change the numbering to match the problem you
%%%%%%%%are on.  Set the section to  the current chapter.

\setcounter{section}{2}

%%%%%%%%%Now, set the theorem number to one less than the first theorem in
%%%%%%%%%the assignment.
\setcounter{thm}{41}

%%%Theorem 2.42%%%%%%%%%%%%%%%%%%%%%%%%%%%%%%%%%%%%%%%%%%%%%%%%%%%%%%%%%%%%%%%%%%%%%

\begin{thm}
If $n$ is a natural number and $2^n - 1$ is prime, then $n$ must be
prime.
\end{thm}

\begin{proof}[Proof]
Let $n$ be a natural number with $2^n - 1$ being prime.  Assume by way of contradiction that $n$ is not prime.  Then by the definition of not prime there will be two natural numbers $x$ and $y$ such that $n = xy$ and $x > 1$.  So $2^n - 1 = 2^{xy} - 1 = (2^x)^y - 1$.  We can now use polynomial long division, similar to exercise 2.41, to find that: 
\[(2^x)^y - 1 = (2^x-1)((2^x)^{y-1} + (2^x)^{y-2} + \cdots + (2^x) + 1)\]
Since $x > 1$, $2^x-1 > 1$.  Now we have shown that $2^n - 1$ is not prime, which is a contradiction.  Therefore $n$ is prime.
\end{proof}


%%%Theorem 2.43%%%%%%%%%%%%%%%%%%%%%%%%%%%%%%%%%%%%%%%%%%%%%%%%%%%%%%%%%%%%%%%%%%%%%

\begin{thm}
If $n$ is a natural number and $2^n+1$ is prime, then $n$ must be a
power of $2$.
\end{thm}

\begin{proof}[Proof]
Let $n$ be a natural number with $2^n + 1$ being prime.  Assume by way of contradiction that $n$ is not a power of 2.  This means that the prime factorization of $n$ contains some term other than 2.  Let $x$ be the product of all the terms in the prime factorization of $n$ that are not 2.  Let $y$ be $2^k$ for some integer $k$ such that $n = xy$.  Now $2^n + 1 = (2^x)^y + 1$.  Preforming polynomial division gives us:
\[(2^x)^y + 1 = (2^x+1)((2^x)^{y-1} - (2^x)^{y-2} - \cdots - 2^x - 1\]
Since $x > 1$, we have found a factorization of $2^n + 1$.  Since $2^n + 1$ is prime, this is a contradiction.  Therefore $n$ is a power of 2.
\end{proof}


%%%Exercise 2.44%%%%%%%%%%%%%%%%%%%%%%%%%%%%%%%%%%%%%%%%%%%%%%%%%%%%%%%%%%%%%%%%%%%%%

\begin{exer}
Find the first few Mersenne primes and Fermat primes.
\end{exer}

\begin{proof}[Solution]
Mersenne: 3, 7, 31, 127, 8191, 131071, 524287.\\
Fermat: 3, 5, 17, 257, 65537.
\end{proof}

\end{document}
