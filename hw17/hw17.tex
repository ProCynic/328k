\documentclass[12pt,leqno]{article}

\usepackage{amsmath,amsfonts,amssymb,amscd,amsthm,amsbsy,upref}


\textheight=8.5truein
\textwidth=6.0truein
\hoffset=-.5truein
\voffset=-.5truein
\numberwithin{equation}{section}
\pagestyle{headings}
\footskip=36pt


\swapnumbers
\newtheorem{thm}{Theorem}[section]
\newtheorem{hthm}[thm]{*Theorem}
\newtheorem{lem}[thm]{Lemma}
\newtheorem{cor}[thm]{Corollary}
\newtheorem{prop}[thm]{Proposition}
\newtheorem{con}[thm]{Conjecture}
\newtheorem{exer}[thm]{Exercise}
\newtheorem{bpe}[thm]{Blank Paper Exercise}
\newtheorem{apex}[thm]{Applications Exercise}
\newtheorem{ques}[thm]{Question}
\newtheorem{scho}[thm]{Scholium}
\newtheorem*{Exthm}{Example Theorem}
\newtheorem*{Thm}{Theorem}
\newtheorem*{Con}{Conjecture}
\newtheorem*{Axiom}{Axiom}



\theoremstyle{definition}
\newtheorem*{Ex}{Example}
\newtheorem*{Def}{Definition}


\newcommand{\lcm}{\operatorname{lcm}}
\newcommand{\ord}{\operatorname{ord}}
\def\pfrac#1#2{{\left(\frac{#1}{#2}\right)}}


\makeindex

\begin{document}




\thispagestyle{plain}
\begin{flushright}
\large{\textbf{TYPE YOUR NAME HERE \\
HW 17: 3.180-3.22\\
M328K \\
March 27th, 2012 \\}}
\end{flushright}

%\maketitle
\markboth{}{} \setcounter{section}{0} \baselineskip=18pt

\setcounter{tocdepth}{4}


%%%%%%%%This is where you can change the numbering to match the problem you
%%%%%%%%are on.  Set the section to  the current chapter.

\setcounter{section}{3}

%%%%%%%%%Now, set the theorem number to one less than the first theorem in
%%%%%%%%%the assignment.
\setcounter{thm}{17}

%%%Exercise 3.18%%%%%%%%%%%%%%%%%%%%%%%%%%%%%%%%%%%%%%%%%%%%%%%%%%%%%%%%%%%%%%%%%%%%%
\begin{exer}
Find all solutions in the appropriate canonical complete residue
system modulo $n$ that satisfy the following linear congruences:
\begin{enumerate}
\item $26x \equiv 14 \pmod{3}$.
\begin{proof}[Solution]
Type your solution here!
\end{proof}
\item $2x \equiv 3 \pmod{5}$.
\begin{proof}[Solution]
Type your solution here!
\end{proof}
\item $4x \equiv 7 \pmod{8}$.
\begin{proof}[Solution]
Type your solution here!
\end{proof}
\item $24x \equiv 123 \pmod{213}$. (This congruence is tedious to do by
trial and error, so perhaps we should defer work on it for now and
instead try to develop some techniques that might help.)
\end{enumerate}
\end{exer}

%%%Theorem 3.19%%%%%%%%%%%%%%%%%%%%%%%%%%%%%%%%%%%%%%%%%%%%%%%%%%%%%%%%%%%%%%%%%%%%%
\begin{thm}
Let $a$, $b$, and $n$ be integers with $n > 0$.  Show that $ax
\equiv b \pmod{n}$ has a solution if and only if there exist
integers $x$ and $y$ such that $ax + ny = b$.
\end{thm}
\begin{proof}[Proof]
Type your proof here!
\end{proof}

%%%Theorem 3.20%%%%%%%%%%%%%%%%%%%%%%%%%%%%%%%%%%%%%%%%%%%%%%%%%%%%%%%%%%%%%%%%%%%%%
\begin{thm}
Let $a$, $b$, and $n$ be integers with $n > 0$. The equation $ax
\equiv b \pmod{n}$ has a solution if and only if $(a, n)|b$.
\end{thm}

\begin{proof}[Proof]
Type your proof here!
\end{proof}



%%%Question 3.21%%%%%%%%%%%%%%%%%%%%%%%%%%%%%%%%%%%%%%%%%%%%%%%%%%%%%%%%%%%%%%%%%%%%%
\begin{ques}
What does the preceding theorem tell us about the congruence $(4)$
in Exercise $3.18$ above?
\end{ques}

\begin{proof}[Solution]
Type your solution here!
\end{proof}


%%%Exercise 3.22%%%%%%%%%%%%%%%%%%%%%%%%%%%%%%%%%%%%%%%%%%%%%%%%%%%%%%%%%%%%%%%%%%%%%
\begin{exer}
Use the Euclidean Algorithm\index{Euclidean Algorithm} to find a
member $x$ of the canonical complete residue system modulo $213$
that satisfies $24x \equiv 123 \pmod{213}$. Find all members $x$ of
the canonical complete residue system modulo $213$ that satisfy $24x
\equiv 123 \pmod{213}$.
\end{exer}

\begin{proof}[Solution]
Type your solution here!
\end{proof}
\end{document}
