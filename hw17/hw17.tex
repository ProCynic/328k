\documentclass[12pt,leqno]{article}

\usepackage{amsmath,amsfonts,amssymb,amscd,amsthm,amsbsy,upref}


\textheight=8.5truein
\textwidth=6.0truein
\hoffset=-.5truein
\voffset=-.5truein
\numberwithin{equation}{section}
\pagestyle{headings}
\footskip=36pt


\swapnumbers
\newtheorem{thm}{Theorem}[section]
\newtheorem{hthm}[thm]{*Theorem}
\newtheorem{lem}[thm]{Lemma}
\newtheorem{cor}[thm]{Corollary}
\newtheorem{prop}[thm]{Proposition}
\newtheorem{con}[thm]{Conjecture}
\newtheorem{exer}[thm]{Exercise}
\newtheorem{bpe}[thm]{Blank Paper Exercise}
\newtheorem{apex}[thm]{Applications Exercise}
\newtheorem{ques}[thm]{Question}
\newtheorem{scho}[thm]{Scholium}
\newtheorem*{Exthm}{Example Theorem}
\newtheorem*{Thm}{Theorem}
\newtheorem*{Con}{Conjecture}
\newtheorem*{Axiom}{Axiom}



\theoremstyle{definition}
\newtheorem*{Ex}{Example}
\newtheorem*{Def}{Definition}


\newcommand{\lcm}{\operatorname{lcm}}
\newcommand{\ord}{\operatorname{ord}}
\def\pfrac#1#2{{\left(\frac{#1}{#2}\right)}}


\makeindex

\begin{document}




\thispagestyle{plain}
\begin{flushright}
\large{\textbf{Geoffrey Parker - grp352 \\
HW 17: 3.180-3.22\\
M328K \\
March 27th, 2012 \\}}
\end{flushright}

%\maketitle
\markboth{}{} \setcounter{section}{0} \baselineskip=18pt

\setcounter{tocdepth}{4}


%%%%%%%%This is where you can change the numbering to match the problem you
%%%%%%%%are on.  Set the section to  the current chapter.

\setcounter{section}{3}

%%%%%%%%%Now, set the theorem number to one less than the first theorem in
%%%%%%%%%the assignment.
\setcounter{thm}{17}

%%%Exercise 3.18%%%%%%%%%%%%%%%%%%%%%%%%%%%%%%%%%%%%%%%%%%%%%%%%%%%%%%%%%%%%%%%%%%%%%
\begin{exer}
Find all solutions in the appropriate canonical complete residue
system modulo $n$ that satisfy the following linear congruences:
\begin{enumerate}
\item $26x \equiv 14 \pmod{3}$.
\begin{proof}[Solution]
All integers $x$ such that $x \equiv 1 \pmod{3}$.
\end{proof}
\item $2x \equiv 3 \pmod{5}$.
\begin{proof}[Solution]
All integers $x$ such that $x \equiv 4 \pmod{5}$.
\end{proof}
\item $4x \equiv 7 \pmod{8}$.
\begin{proof}[Solution]
No solution.
\end{proof}
\item $24x \equiv 123 \pmod{213}$. (This congruence is tedious to do by
trial and error, so perhaps we should defer work on it for now and
instead try to develop some techniques that might help.)
\begin{proof}[Solution]
See 3.22.
\end{proof}
\end{enumerate}
\end{exer}

%%%Theorem 3.19%%%%%%%%%%%%%%%%%%%%%%%%%%%%%%%%%%%%%%%%%%%%%%%%%%%%%%%%%%%%%%%%%%%%%
\begin{thm}
Let $a$, $b$, and $n$ be integers with $n > 0$.  Show that $ax
\equiv b \pmod{n}$ has a solution if and only if there exist
integers $x$ and $y$ such that $ax + ny = b$.
\end{thm}
\begin{proof}[Proof]
Let $a$, $b$, and $n$ be integers with $n > 0$.  We will show that $ax \equiv b \pmod{n}$ has a solution if and only if there exist integers $x$ and $y$ such that $ax + ny = b$. \\
First, assume that $ax \equiv b \pmod{n}$ has a solution $x$.  Then $n \mid ax - b$.  Using the definition of divides, let $-y$ be the integer such that $n(-y) = ax - b$.  So $ax + ny = b$.\\
Now assume that there exist integers $x$ and $y$ such that $ax + ny = b$.  Then $ax - b = -ny$, so by the definition of divides $n \mid ax - b$.  Therefore by the definition of congruence mod n, $ax \equiv b \pmod{n}$.
\end{proof}
-
%%%Theorem 3.20%%%%%%%%%%%%%%%%%%%%%%%%%%%%%%%%%%%%%%%%%%%%%%%%%%%%%%%%%%%%%%%%%%%%%
\begin{thm}
Let $a$, $b$, and $n$ be integers with $n > 0$. The equation $ax
\equiv b \pmod{n}$ has a solution if and only if $(a, n)|b$.
\end{thm}

\begin{proof}[Proof]
Let $a$, $b$, and $n$ be integers with $n > 0$. We will show that the equation $ax \equiv b \pmod{n}$ has a solution if and only if $(a, n) \mid b$.\\

First, assume that $ax \equiv b \pmod{n}$ has a solution $x$.  By theorem ??, there exist integers $j$ and $k$ such that $aj + nk = (a, n)$.  So $n(-k) = aj - (a, n)$ and $n \mid aj - (a, n)$ Also, by the definition of congruence modulo n, $n \mid ax - b$.  So by theorem 1.1, $n \mid ax - b + aj - (a, n)$, and by the definition of divides $nm = a(x+j) - b - (a, n)$ for some integer $m$.  Then for some integers $c$ and $d$, 
\[c(a, n)m = d(a, n)(x+j) - b - (a, n).\]
Rearranging this gives us :
\[-cm(a, n) + d(x+j)(a, n) - (a, n) = b\]
or:
\[(a, n)(d(x+j) - cm - 1) = b.\]
Since $d(x+j) - cm - 1$ is an integer, $(a, n) \mid b$.\\

Now assume that $(a, n) \mid b$.  By the definition of divides there exists some integer $m$ such that $(a, n)m = b$.  And by theorem ??, there exist integers $j$ and $k$ such that $aj + nk = (a, n)$.  Multiplying both sides by $m$ gives us $m(aj + nk) = b$, or $ajm - b = nm(-k)$.  So by the definition of divides $n \mid ajm - b$, and if we let $x = jm$, then by the definition of congruence modulo n, we have $ax \equiv b \pmod{n}$.
\end{proof}



%%%Question 3.21%%%%%%%%%%%%%%%%%%%%%%%%%%%%%%%%%%%%%%%%%%%%%%%%%%%%%%%%%%%%%%%%%%%%%
\begin{ques}
What does the preceding theorem tell us about the congruence $(4)$
in Exercise $3.18$ above?
\end{ques}

\begin{proof}[Solution]
Because $(24, 213) = 3$ and $123 = 3 \times 41$, we know by theorem 3.20 that a solution to $24x \equiv 123 \pmod{213}$ exists.
\end{proof}


%%%Exercise 3.22%%%%%%%%%%%%%%%%%%%%%%%%%%%%%%%%%%%%%%%%%%%%%%%%%%%%%%%%%%%%%%%%%%%%%
\begin{exer}
Use the Euclidean Algorithm\index{Euclidean Algorithm} to find a
member $x$ of the canonical complete residue system modulo $213$
that satisfies $24x \equiv 123 \pmod{213}$. Find all members $x$ of
the canonical complete residue system modulo $213$ that satisfy $24x
\equiv 123 \pmod{213}$.
\end{exer}

\begin{proof}[Solution]
We will find all integers $x$ in the canonical complete residue system modulo $213$ such that $24x \equiv 123 \pmod{213}$ using the extended euclidean algorithm.
\end{proof}
\end{document}
