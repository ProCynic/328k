
\documentclass[11pt,leqno]{article}

\usepackage{amsmath,amsfonts,amssymb,amscd,amsthm,amsbsy,upref}


\textheight=8.5truein \textwidth=6.0truein \hoffset=-.5truein
\voffset=-.5truein \numberwithin{equation}{section}
\pagestyle{headings} \footskip=36pt


\swapnumbers
\newtheorem{thm}{Theorem}[section]
\newtheorem{hthm}[thm]{*Theorem}
\newtheorem{lem}[thm]{Lemma}
\newtheorem{cor}[thm]{Corollary}
\newtheorem{prop}[thm]{Proposition}
\newtheorem{con}[thm]{Conjecture}
\newtheorem{exer}[thm]{Exercise}
\newtheorem{bpe}[thm]{Blank Paper Exercise}
\newtheorem{apex}[thm]{Applications Exercise}
\newtheorem{ques}[thm]{Question}
\newtheorem{scho}[thm]{Scholium}
\newtheorem*{Exthm}{Example Theorem}
\newtheorem*{Thm}{Theorem}
\newtheorem*{Con}{Conjecture}
\newtheorem*{Axiom}{Axiom}



\theoremstyle{definition}
\newtheorem*{Ex}{Example}
\newtheorem*{Def}{Definition}


\newcommand{\lcm}{\operatorname{lcm}}
\newcommand{\ord}{\operatorname{ord}}
\def\pfrac#1#2{{\left(\frac{#1}{#2}\right)}}


\makeindex

\begin{document}

\thispagestyle{plain}
\begin{flushright}
\large{\textbf{Name \\
M328K  \\
Chapter 3\\
DUE DATE \\}}
\end{flushright}



\setcounter{tocdepth}{3}

\section*{Thinking Cyclically}


%%%%%%%%%%%%%%%%%%%%%%%%%%%%%%%%%%%%%%%%%%%%%%%%%%%%%555
\setcounter{section}{3}

%Here is where you can change what number you want the theorems to start at.  You put one less than the number of the first theorem.
\setcounter{thm}{0}




%%%Exercise 3.1%%%%%%%%%%%%%%%%%%%%%%%%%%%%%%%%%%%%%%%%%%%%%%%%%%%%%%%%%%%%%%%%%%%%%

\begin{exer}
Show that $41$ divides $2^{20}-1$ by following these steps.  Explain
why each step is true.
\begin{enumerate} \item $2^5 \equiv -9 \pmod{41}$.  \item $(2^5)^4
\equiv (-9)^4 \pmod{41}$.  \item $2^{20} \equiv 81^2 \pmod{41}
\equiv (-1)^2\pmod{41}$.  \item $2^{20} - 1 \equiv 0 \pmod{41}$.
\end{enumerate}
\end{exer}

\begin{proof}[Solution]
Type your solution here!
\end{proof}

%%%Question 3.2%%%%%%%%%%%%%%%%%%%%%%%%%%%%%%%%%%%%%%%%%%%%%%%%%%%%%%%%%%%%%%%%%%%%%
\begin{ques}
In your head, can you find the natural number $k$, $0 \leq k \leq
11$, such that $k \equiv 37^{453} \pmod{12}$?
\end{ques}

\begin{proof}[Solution]
Type your solution here!
\end{proof}

%%%Question 3.3%%%%%%%%%%%%%%%%%%%%%%%%%%%%%%%%%%%%%%%%%%%%%%%%%%%%%%%%%%%%%%%%%%%%%
\begin{ques}
In your head or using paper and pencil, but no calculator, can you
find the natural number $k$, $0 \leq k \leq 6$, such that $2^{50}
\equiv k \pmod{7}$.
\end{ques}

\begin{proof}[Solution]
Type your solution here!
\end{proof}

%%%Question 3.4%%%%%%%%%%%%%%%%%%%%%%%%%%%%%%%%%%%%%%%%%%%%%%%%%%%%%%%%%%%%%%%%%%%%%
\begin{ques}
Using paper and pencil, but no calculator, can you find the natural
number $k$, $0 \leq k \leq 11$, such that~$39^{453} \equiv k
\pmod{12}$.
\end{ques}

\begin{proof}[Solution]
Type your solution here!
\end{proof}

%%%Exercise 3.5%%%%%%%%%%%%%%%%%%%%%%%%%%%%%%%%%%%%%%%%%%%%%%%%%%%%%%%%%%%%%%%%%%%%%
\begin{exer}
Show that $39$ divides $17^{48}-5^{24}$.
\end{exer}

\begin{proof}[Solution]
Type your solution here!
\end{proof}

%%%Question 3.6 %%%%%%%%%%%%%%%%%%%%%%%%%%%%%%%%%%%%%%%%%%%%%%%%%%%%%%%%%%%%%%%%%%%%%
\begin{ques}[Describe technique]
Let $a$, $n$, and $r$ be natural numbers.  Describe how to find the
number $k$ ($0 \leq k \leq n - 1$) such that $k \equiv a^r \pmod{n}$
subject to the restraint that you never multiply numbers larger than
$n$ and that you only have to do about $\log_2{r}$ such
multiplications.
\end{ques}

\begin{proof}[Solution]
Type your solution here!
\end{proof}

%%%Question 3.7%%%%%%%%%%%%%%%%%%%%%%%%%%%%%%%%%%%%%%%%%%%%%%%%%%%%%%%%%%%%%%%%%%%%%
\begin{ques}
Let $f(x) = 13x^{49} - 27x^{27} + x^{14} - 6$.  Is it true that
\[f(98) \equiv f(-100) \pmod{99}?\]
\end{ques}

\begin{proof}[Solution]
Type your solution here!
\end{proof}
%%%Theorem 3.8%%%%%%%%%%%%%%%%%%%%%%%%%%%%%%%%%%%%%%%%%%%%%%%%%%%%%%%%%%%%%%%%%%%%%

\begin{thm}
Suppose $f(x) = a_nx^n + a_{n-1}x^{n-1} + \hdots + a_0$ is a
polynomial of degree $n>0$ with integer coefficients.  Let $a$, $b$,
and $m$ be integers with $m > 0$.  If $a \equiv b \pmod{m}$,
then~$f(a) \equiv f(b) \pmod{m}$.
\end{thm}

\begin{proof}[Proof]
Type your proof here!
\end{proof}

%%%Corollary 3.9%%%%%%%%%%%%%%%%%%%%%%%%%%%%%%%%%%%%%%%%%%%%%%%%%%%%%%%%%%%%%%%%%%%%%
\begin{cor}
Let the natural number $n$ be expressed in base $10$ as
\[n = a_k a_{k-1} \hdots a_1 a_0. \]
 Let $m = a_k + a_{k-1} + \hdots + a_1 + a_0$. Then $9|n$ if and only if $9|m$.
\end{cor}

\begin{proof}[Proof]
Type your proof here!
\end{proof}


%%%Corollary 3.10%%%%%%%%%%%%%%%%%%%%%%%%%%%%%%%%%%%%%%%%%%%%%%%%%%%%%%%%%%%%%%%%%%%%%
\begin{cor}
Let the natural number $n$ be expressed in base $10$ as
\[n = a_k a_{k-1} \hdots a_1 a_0. \]
If $m = a_k + a_{k-1} + \hdots + a_1 + a_0$. Then $3|n$ if and only
if $3|m$.
\end{cor}

\begin{proof}[Proof]
Type your proof here!
\end{proof}


%%%Theorem 3.11%%%%%%%%%%%%%%%%%%%%%%%%%%%%%%%%%%%%%%%%%%%%%%%%%%%%%%%%%%%%%%%%%%%%%
\begin{thm}
Suppose $f(x) = a_nx^n + a_{n-1}x^{n-1} + \hdots + a_0$ is a
polynomial of degree $n > 0$ and suppose $a_n > 0$.  Then there is
an integer $k$ such that if~$x > k$, then $f(x) > 0$.
\end{thm}

\begin{proof}[Proof]
Type your proof here!
\end{proof}


%%%Theorem 3.12%%%%%%%%%%%%%%%%%%%%%%%%%%%%%%%%%%%%%%%%%%%%%%%%%%%%%%%%%%%%%%%%%%%%%
\begin{thm}
Suppose $f(x) = a_nx^n + a_{n-1}x^{n-1} + \hdots + a_0$  is a
polynomial of degree $n > 0$ and suppose $a_n > 0$.  Then for any
number $M$ there is an integer~$k$ (which depends on $M$) such that
if~$x > k$, then $f(x) > M$.
\end{thm}

\begin{proof}[Proof]
Type your proof here!
\end{proof}


%%%Theorem 3.13%%%%%%%%%%%%%%%%%%%%%%%%%%%%%%%%%%%%%%%%%%%%%%%%%%%%%%%%%%%%%%%%%%%%%
\begin{thm}
Suppose $f(x) = a_nx^n + a_{n-1}x^{n-1} + \hdots + a_0$ is a
polynomial of degree $n > 0$ with integer coefficients.  Then $f(x)$
is a composite number for infinitely many integers~$x$.
\end{thm}

\begin{proof}[Proof]
Type your proof here!
\end{proof}

%%%Theorem 3.14%%%%%%%%%%%%%%%%%%%%%%%%%%%%%%%%%%%%%%%%%%%%%%%%%%%%%%%%%%%%%%%%%%%%%
\begin{thm}Given any integer $a$ and any natural number $n$, there exists a unique integer~$t$ in the set $\{0, 1, 2, \hdots, n - 1\}$ such that $a \equiv t \pmod{n}$.
\end{thm}

\begin{proof}[Proof]
Type your proof here!
\end{proof}



%%%Exercise 3.15%%%%%%%%%%%%%%%%%%%%%%%%%%%%%%%%%%%%%%%%%%%%%%%%%%%%%%%%%%%%%%%%%%%%%
\begin{exer}
Find three complete residue systems modulo $4$: the canonical
complete residue system, one containing negative numbers, and one
containing no two consecutive numbers.
\end{exer}

\begin{proof}[Solution]
Type your solution here!
\end{proof}


%%%Theorem 3.16%%%%%%%%%%%%%%%%%%%%%%%%%%%%%%%%%%%%%%%%%%%%%%%%%%%%%%%%%%%%%%%%%%%%%
\begin{thm}
Let $n$ be a natural number. Every complete residue system modulo
$n$ contains $n$ elements.
\end{thm}

\begin{proof}[Proof]
Type your proof here!
\end{proof}


%%%Theorem 3.17%%%%%%%%%%%%%%%%%%%%%%%%%%%%%%%%%%%%%%%%%%%%%%%%%%%%%%%%%%%%%%%%%%%%%
\begin{thm}
Let $n$ be a natural number. Any set of $n$ integers $\{a_1, a_2,
\hdots, a_n\}$ for which no two are congruent modulo $n$ is a
complete residue system modulo $n$.
\end{thm}

\begin{proof}[Proof]
Type your proof here!
\end{proof}



%%%Exercise 3.18%%%%%%%%%%%%%%%%%%%%%%%%%%%%%%%%%%%%%%%%%%%%%%%%%%%%%%%%%%%%%%%%%%%%%
\begin{exer}
Find all solutions in the appropriate canonical complete residue
system modulo $n$ that satisfy the following linear congruences:
\begin{enumerate}
\item $26x \equiv 14 \pmod{3}$.
\begin{proof}[Solution]
Type your solution here!
\end{proof}
\item $2x \equiv 3 \pmod{5}$.
\begin{proof}[Solution]
Type your solution here!
\end{proof}
\item $4x \equiv 7 \pmod{8}$.
\begin{proof}[Solution]
Type your solution here!
\end{proof}
\item $24x \equiv 123 \pmod{213}$. (This congruence is tedious to do by
trial and error, so perhaps we should defer work on it for now and
instead try to develop some techniques that might help.)
\end{enumerate}
\end{exer}

%%%Theorem 3.19%%%%%%%%%%%%%%%%%%%%%%%%%%%%%%%%%%%%%%%%%%%%%%%%%%%%%%%%%%%%%%%%%%%%%
\begin{thm}
Let $a$, $b$, and $n$ be integers with $n > 0$.  Show that $ax
\equiv b \pmod{n}$ has a solution if and only if there exist
integers $x$ and $y$ such that $ax + ny = b$.
\end{thm}
\begin{proof}[Proof]
Type your proof here!
\end{proof}

%%%Theorem 3.20%%%%%%%%%%%%%%%%%%%%%%%%%%%%%%%%%%%%%%%%%%%%%%%%%%%%%%%%%%%%%%%%%%%%%
\begin{thm}
Let $a$, $b$, and $n$ be integers with $n > 0$. The equation $ax
\equiv b \pmod{n}$ has a solution if and only if $(a, n)|b$.
\end{thm}

\begin{proof}[Proof]
Type your proof here!
\end{proof}



%%%Question 3.21%%%%%%%%%%%%%%%%%%%%%%%%%%%%%%%%%%%%%%%%%%%%%%%%%%%%%%%%%%%%%%%%%%%%%
\begin{ques}
What does the preceding theorem tell us about the congruence $(4)$
in Exercise $3.18$ above?
\end{ques}

\begin{proof}[Solution]
Type your solution here!
\end{proof}


%%%Exercise 3.22%%%%%%%%%%%%%%%%%%%%%%%%%%%%%%%%%%%%%%%%%%%%%%%%%%%%%%%%%%%%%%%%%%%%%
\begin{exer}
Use the Euclidean Algorithm\index{Euclidean Algorithm} to find a
member $x$ of the canonical complete residue system modulo $213$
that satisfies $24x \equiv 123 \pmod{213}$. Find all members $x$ of
the canonical complete residue system modulo $213$ that satisfy $24x
\equiv 123 \pmod{213}$.
\end{exer}

\begin{proof}[Solution]
Type your solution here!
\end{proof}


%%%Question 3.23%%%%%%%%%%%%%%%%%%%%%%%%%%%%%%%%%%%%%%%%%%%%%%%%%%%%%%%%%%%%%%%%%%%%%
\begin{ques}
Let $a$, $b$, and $n$ be integers with $n > 0$.  How many solutions
are there to the linear congruence $ax \equiv b \pmod{n}$ in the
canonical complete residue system modulo $n$? Can you describe a
technique to find them?
\end{ques}

\begin{proof}[Solution]
Type your solution here!
\end{proof}


%%%Theorem 3.24%%%%%%%%%%%%%%%%%%%%%%%%%%%%%%%%%%%%%%%%%%%%%%%%%%%%%%%%%%%%%%%%%%%%%
\begin{thm}
Let $a$, $b$, and $n$ be integers with $n > 0$. Then,
\begin{enumerate}
\item The congruence $ax \equiv b \pmod{n}$ is solvable in integers if
and only if $(a,n)|b$.
\item If $x_0$ is a solution to the congruence $ax \equiv b\pmod{n}$,
then all solutions are given by \[ x_0 + \left(\frac{n}{(a,n)}\cdot
m\right)\pmod{n}\] for $m = 0,\ 1,\ 2,\ \hdots\ $, $(a, n)-1$.
\item If $ax \equiv b \pmod{n}$ has a solution, then there are exactly
$(a, n)$ solutions in the canonical complete residue system modulo
$n$.
\end{enumerate}
\end{thm}

\begin{proof}[Proof]
Type your proof here!
\end{proof}




%%%Exercise 3.25%%%%%%%%%%%%%%%%%%%%%%%%%%%%%%%%%%%%%%%%%%%%%%%%%%%%%%%%%%%%%%%%%%%%%
\begin{exer}
A band of $17$ pirates stole a sack of gold coins.  When they tried
to divide the fortune into equal portions, $3$ coins remained.  In
the ensuing brawl over who should get the extra coins, one pirate
was killed.  The coins were redistributed, but this time an equal
division left $10$ coins.  Again they fought about who should get
the remaining coins and another pirate was killed.  Now,
fortunately, the coins could be divided evenly among the surviving
$15$ pirates.  What was the fewest number of coins that could have
been in the sack?
\end{exer}

\begin{proof}[Solution]
Type your solution here!
\end{proof}



%%%Exercise 3.26%%%%%%%%%%%%%%%%%%%%%%%%%%%%%%%%%%%%%%%%%%%%%%%%%%%%%%%%%%%%%%%%%%%%%
\begin{exer}[Brahmagupta, 7th century A.D.]
When eggs in a basket are removed two, three, four, five or six at a
time, there remain, respectively, one, two, three, four, or five
eggs.  When they are taken out seven at a time, none are left over.
Find the smallest number of eggs that could have been contained in
the basket.
\end{exer}

\begin{proof}[Solution]
Type your solution here!
\end{proof}



%%%Theorem 3.27%%%%%%%%%%%%%%%%%%%%%%%%%%%%%%%%%%%%%%%%%%%%%%%%%%%%%%%%%%%%%%%%%%%%%
\begin{thm}
Let $a$, $b$, $m$, and $n$ be integers with $m > 0$ and $n > 0$.
Then the system
\[ \begin{array}{l}
x \equiv a \pmod{n} \\
x \equiv b \pmod{m} \end{array} \] has a solution if and only if
$(n, m)| a - b$.
\end{thm}

\begin{proof}[Proof]
Type your proof here!
\end{proof}


%%%Theorem 3.28%%%%%%%%%%%%%%%%%%%%%%%%%%%%%%%%%%%%%%%%%%%%%%%%%%%%%%%%%%%%%%%%%%%%%
\begin{thm}
Let $a$, $b$, $m$, and $n$ be integers with $m > 0$, $n > 0$, and
$(m,n)=1$.  Then the system
\[ \begin{array}{l}
x \equiv a \pmod{n} \\
x \equiv b \pmod{m} \end{array} \] has a unique solution modulo
$mn$.
\end{thm}

\begin{proof}[Proof]
Type your proof here!
\end{proof}


%%%Theorem 3.29%%%%%%%%%%%%%%%%%%%%%%%%%%%%%%%%%%%%%%%%%%%%%%%%%%%%%%%%%%%%%%%%%%%%%
\begin{thm}[Chinese Remainder Theorem]
Suppose $n_1, n_2, \hdots, n_L$ are positive integers that are
pairwise relatively prime, that is, $(n_i, n_j)=1$ for $i\neq j$,
$1\leq i, j \leq L$.  Then the system of $L$ congruences
\[ \begin{array}{c}
x \equiv a_1 \pmod{n_1} \\
x \equiv a_2 \pmod{n_2} \\
\vdots \\
x \equiv a_L \pmod{n_L} \end{array} \] has a unique solution modulo
the product $n_1 n_2  n_3 \cdots n_L$.
\end{thm}

\begin{proof}[Proof]
Type your proof here!
\end{proof}


%%%BPE 3.30%%%%%%%%%%%%%%%%%%%%%%%%%%%%%%%%%%%%%%%%%%%%%%%%%%%%%%%%%%%%%%%%%%%%%
\begin{bpe}
After not looking at the material in this chapter for a day or two,
take a blank piece of paper and outline the development of that
material in as much detail as you can without referring to the text
or to notes. Places where you get stuck or can't remember highlight
areas that may call for further study.
\end{bpe}

\begin{proof}[Solution]
Type your solution here!
\end{proof}



\end{document}
