
\documentclass[11pt,leqno]{article}

\usepackage{amsmath,amsfonts,amssymb,amscd,amsthm,amsbsy,upref}


\textheight=8.5truein \textwidth=6.0truein \hoffset=-.5truein
\voffset=-.5truein \numberwithin{equation}{section}
\pagestyle{headings} \footskip=36pt


\swapnumbers
\newtheorem{thm}{Theorem}[section]
\newtheorem{hthm}[thm]{*Theorem}
\newtheorem{lem}[thm]{Lemma}
\newtheorem{cor}[thm]{Corollary}
\newtheorem{prop}[thm]{Proposition}
\newtheorem{con}[thm]{Conjecture}
\newtheorem{exer}[thm]{Exercise}
\newtheorem{bpe}[thm]{Blank Paper Exercise}
\newtheorem{apex}[thm]{Applications Exercise}
\newtheorem{ques}[thm]{Question}
\newtheorem{scho}[thm]{Scholium}
\newtheorem*{Exthm}{Example Theorem}
\newtheorem*{Thm}{Theorem}
\newtheorem*{Con}{Conjecture}
\newtheorem*{Axiom}{Axiom}



\theoremstyle{definition}
\newtheorem*{Ex}{Example}
\newtheorem*{Def}{Definition}


\newcommand{\lcm}{\operatorname{lcm}}
\newcommand{\ord}{\operatorname{ord}}
\def\pfrac#1#2{{\left(\frac{#1}{#2}\right)}}


\makeindex

\begin{document}

\thispagestyle{plain}
\begin{flushright}
\large{\textbf{Name \\
M328K - MWF \\
Chapter 5\\
DUE DATE \\}}
\end{flushright}



\setcounter{tocdepth}{3}




%%%%%%%%%%%%%%%%%%%%%%%%%%%%%%%%%%%%%%%%%%%%%%%%%%%%%555
\setcounter{section}{4}

%Here is where you can change what number you want the theorems to start at.  You put one less than the number of the first theorem.
\setcounter{thm}{0}


\section{Public Key Cryptography}


\section*{Public Key Codes and RSA}

%%%%%%%%%%%%%%%%%%%%%%%%%%%%%%%%%%%%%%%%%%%%%%%%

\subsection*{Public key codes}

%%%%%%%%%%%%%%%%%%%%%%%%%%%%%%%%%%%%%%%%%%%%%%%%%%%

\subsection*{Overview of RSA}
%%%%%%%%%%%%%%%%%%%%%%%%%%%%%%%%%%%%%%%%%%%%%%%%%%%%%%%%%%%%

\subsection*{Let's decrypt}

%%%Theorem 5.1%%%%%%%%%%%%%%%%%%%%%%%%%%%%%%%%%%%%%%%%%%%%%%%%%%%%%%%%%%%%%%%%%%%%%

\begin{thm}
If $p$ and $q$ are distinct prime numbers and $W$ is a natural
number with $(W, pq)=1$, then $W^{(p-1)(q-1)}\equiv 1 \pmod{pq}$.
\end{thm}

%%%Theorem 5.2%%%%%%%%%%%%%%%%%%%%%%%%%%%%%%%%%%%%%%%%%%%%%%%%%%%%%%%%%%%%%%%%%%%%%

\begin{thm}
Let $p$ and $q$ be distinct primes, $k$ be a natural number, and $W$
be a natural number less than $pq$. Then
\[W^{1+k(p-1)(q-1)}\equiv W\pmod{pq}.\]
\end{thm}

%%%Theorem 5.3%%%%%%%%%%%%%%%%%%%%%%%%%%%%%%%%%%%%%%%%%%%%%%%%%%%%%%%%%%%%%%%%%%%%%
\begin{thm}
Let $p$ and $q$ be distinct primes and $E$ be a natural number
relatively prime to $(p-1)(q-1)$. Then there exist natural numbers
$D$ and $y$ such that
\[ED = 1+y(p-1)(q-1).\]
\end{thm}

%%%Theorem 5.4%%%%%%%%%%%%%%%%%%%%%%%%%%%%%%%%%%%%%%%%%%%%%%%%%%%%%%%%%%%%%%%%%%%%%
\begin{thm}
Let $p$ and $q$ be distinct primes, $W$ be a natural number less
than $pq$, and $E$, $D$, and $y$ be natural numbers such that
$ED=1+y(p-1)(q-1)$. Then
\[W^{ED}\equiv W\pmod{pq}.\]
\end{thm}

%%%Exercise 5.5%%%%%%%%%%%%%%%%%%%%%%%%%%%%%%%%%%%%%%%%%%%%%%%%%%%%%%%%%%%%%%%%%%%%%
\begin{exer}
Consider two distinct primes $p$ and $q$. Describe every step of the
RSA Public Key Coding System. State what numbers you choose to make
public, what messages can be encoded, how messages should be
encoded, and how messages are decoded.  What number should be called
the \textit{encoding exponent} and what number should be called the
\textit{decoding exponent}?
\end{exer}

%%%Exercise 5.6%%%%%%%%%%%%%%%%%%%%%%%%%%%%%%%%%%%%%%%%%%%%%%%%%%%%%%%%%%%%%%%%%%%%%
\begin{exer}
Describe an RSA Public Key Code System based on the primes $11$ and
$17$.  Encode and decode several messages.
\end{exer}

%%%Exercise 5.7%%%%%%%%%%%%%%%%%%%%%%%%%%%%%%%%%%%%%%%%%%%%%%%%%%%%%%%%%%%%%%%%%%%%%
\begin{exer}
You are a secret agent. An evil spy with shallow number theory
skills uses the RSA Public Key Coding System in which the public
modulus is $n=1537$, and the encoding exponent is $E=47$. You
intercept one of the encoded secret messages being sent to the evil
spy, namely the number $570$. Using your superior number theory
skills, decode this message, thereby saving countless people from
the fiendish plot of the evil spy.
\end{exer}

%%%Exercise 5.8%%%%%%%%%%%%%%%%%%%%%%%%%%%%%%%%%%%%%%%%%%%%%%%%%%%%%%%%%%%%%%%%%%%%%
\begin{exer}
Suppose an RSA Public Key Coding System publishes $n$ (which is
equal to the product of two undisclosed primes $p$ and $q$) and $E$,
with $E$ relatively prime to $(p-1)(q-1)$. Suppose someone wants to
send a secret message and so encodes the message number $W$ (less
than $n$) by finding the number $m$ less than $n$ such that
$m\equiv W^E\pmod{n}$. Suppose you intercept this number $m$ and you
are able to factor $n$. How can you figure out the original message
$W$?
\end{exer}

%%%Exercise 5.9%%%%%%%%%%%%%%%%%%%%%%%%%%%%%%%%%%%%%%%%%%%%%%%%%%%%%%%%%%%%%%%%%%%%%
\begin{apex}
You have seen the application of number theory to RSA cryptography.
Find out all you can about the role of number theory in some other
types of ``codes" such as bar codes, ISBN codes, and credit card
number ``codes".
\end{apex}

%%%%%%%%%%%%%%%%%%%%%%%%%%%%%%%%%%%%%%%%%%%%%%%%%%%%%%%%%%%%%%%%%%%%%%%%%%



\end{document}
