
\documentclass[11pt,leqno]{article}

\usepackage{amsmath,amsfonts,amssymb,amscd,amsthm,amsbsy,upref}


\textheight=8.5truein \textwidth=6.0truein \hoffset=-.5truein
\voffset=-.5truein \numberwithin{equation}{section}
\pagestyle{headings} \footskip=36pt


\swapnumbers
\newtheorem{thm}{Theorem}[section]
\newtheorem{hthm}[thm]{*Theorem}
\newtheorem{lem}[thm]{Lemma}
\newtheorem{cor}[thm]{Corollary}
\newtheorem{prop}[thm]{Proposition}
\newtheorem{con}[thm]{Conjecture}
\newtheorem{exer}[thm]{Exercise}
\newtheorem{bpe}[thm]{Blank Paper Exercise}
\newtheorem{apex}[thm]{Applications Exercise}
\newtheorem{ques}[thm]{Question}
\newtheorem{scho}[thm]{Scholium}
\newtheorem*{Exthm}{Example Theorem}
\newtheorem*{Thm}{Theorem}
\newtheorem*{Con}{Conjecture}
\newtheorem*{Axiom}{Axiom}



\theoremstyle{definition}
\newtheorem*{Ex}{Example}
\newtheorem*{Def}{Definition}


\newcommand{\lcm}{\operatorname{lcm}}
\newcommand{\ord}{\operatorname{ord}}
\def\pfrac#1#2{{\left(\frac{#1}{#2}\right)}}


\makeindex

\begin{document}

\thispagestyle{plain}
\begin{flushright}
\large{\textbf{Name \\
M328K  \\
DUE DATE \\}}
\end{flushright}



\setcounter{tocdepth}{3}




%%%%%%%%%%%%%%%%%%%%%%%%%%%%%%%%%%%%%%%%%%%%%%%%%%%%%555
\setcounter{section}{3}

%Here is where you can change what number you want the theorems to start at.  You put one less than the number of the first theorem.
\setcounter{thm}{0}


\section{Fermat's Little Theorem and Euler's Theorem}


\section*{Abstracting the Ordinary}


\subsection*{Orders of an integer modulo $n$}

%%%Exercise 4.1%%%%%%%%%%%%%%%%%%%%%%%%%%%%%%%%%%%%%%%%%%%%%%%%%%%%%%%%%%%%%%%%%%%%%

\begin{exer}For $i = 0$, $1$, $2$, $3$, $4$, $5$, and $6$,
find the number in the canonical complete residue
system\index{canonical complete residue system modulo $n$} to which
$2^i$ is congruent modulo $7$.  In other words, compute
$2^0\pmod{7}, 2^1\pmod{7}, 2^2\pmod{7}, \dots, 2^6\pmod{7}$.
\end{exer}

%%%Theorem 4.2%%%%%%%%%%%%%%%%%%%%%%%%%%%%%%%%%%%%%%%%%%%%%%%%%%%%%%%%%%%%%%%%%%%%%

\begin{thm}
Let $a$ and $n$ be natural numbers with $(a, n) = 1$.  Then $(a^j,
n) = 1$ for any natural number $j$.
\end{thm}

%%%Theorem 4.3%%%%%%%%%%%%%%%%%%%%%%%%%%%%%%%%%%%%%%%%%%%%%%%%%%%%%%%%%%%%%%%%%%%%%

\begin{thm}
Let $a$, $b$, and $n$ be integers with $n>0$ and $(a, n) = 1$.  If
$a\equiv b \pmod{n}$, then $(b, n) = 1$.
\end{thm}

%%%Theorem 4.4%%%%%%%%%%%%%%%%%%%%%%%%%%%%%%%%%%%%%%%%%%%%%%%%%%%%%%%%%%%%%%%%%%%%%

\begin{thm}
Let $a$ and $n$ be natural numbers.  Then there exist natural
numbers $i$ and $j$, with $i \neq j$, such that $a^i \equiv a^j
\pmod{n}$.
\end{thm}

%%%Theorem 4.5%%%%%%%%%%%%%%%%%%%%%%%%%%%%%%%%%%%%%%%%%%%%%%%%%%%%%%%%%%%%%%%%%%%%%

\begin{thm}
Let $a$, $b$, $c$, and $n$ be integers with $n
> 0$.  If $ac \equiv bc \pmod{n}$ and $(c, n) = 1$, then $a \equiv b
\pmod{n}$.
\end{thm}

%%%Theorem 4.6%%%%%%%%%%%%%%%%%%%%%%%%%%%%%%%%%%%%%%%%%%%%%%%%%%%%%%%%%%%%%%%%%%%%%

\begin{thm}
Let $a$ and $n$ be natural numbers with $(a, n) = 1$.  Then there
exists a natural number $k$ such that $a^k \equiv 1 \pmod{n}$.
\end{thm}



%%%%%%%%%%%%%%%%%%%%%%%%%%%%%%%%%%%%%%%%%%%%%%%%%%

\subsection*{Fermat's Little Theorem}

%%%Question 4.7%%%%%%%%%%%%%%%%%%%%%%%%%%%%%%%%%%%%%%%%%%%%%%%%%%%%%%%%%%%%%%%%%%%%%

\begin{ques}
Choose some relatively prime natural numbers $a$ and $n$ and compute
the order of $a$ modulo $n$. Frame a conjecture concerning how large
the order of $a$ modulo $n$ can be, depending on $n$.
\end{ques}

%%%Theorem 4.8%%%%%%%%%%%%%%%%%%%%%%%%%%%%%%%%%%%%%%%%%%%%%%%%%%%%%%%%%%%%%%%%%%%%%

\begin{thm}
Let $a$ and $n$ be natural numbers with $(a, n) = 1$ and let $k =
\ord_n(a)$.  Then the numbers $a^1$, $a^2$, $\dots$, $a^k$ are
pairwise incongruent modulo $n$.
\end{thm}

%%%Theorem 4.9%%%%%%%%%%%%%%%%%%%%%%%%%%%%%%%%%%%%%%%%%%%%%%%%%%%%%%%%%%%%%%%%%%%%%
\begin{thm}
Let $a$ and $n$ be natural numbers with $(a, n) = 1$ and let $k =
\ord_n(a)$.  For any natural number $m$, $a^m$ is congruent modulo
$n$ to one of the numbers~$a^1$, $a^2$, $\hdots$, $a^k$.
\end{thm}

%%%Theorem 4.10%%%%%%%%%%%%%%%%%%%%%%%%%%%%%%%%%%%%%%%%%%%%%%%%%%%%%%%%%%%%%%%%%%%%%

\begin{thm}
Let $a$ and $n$ be natural numbers with $(a, n) = 1$, let $k =
\ord_n(a)$, and let $m$ be a natural number.  Then $a^m \equiv 1
\pmod{n}$ if and only if $k|m$.
\end{thm}

%%%Theorem 4.11%%%%%%%%%%%%%%%%%%%%%%%%%%%%%%%%%%%%%%%%%%%%%%%%%%%%%%%%%%%%%%%%%%%%%
\begin{thm}
Let $a$ and $n$ be natural numbers with $(a, n) = 1$. Then
$\ord_n(a) < n$.
\end{thm}

%%%Exercise 4.12%%%%%%%%%%%%%%%%%%%%%%%%%%%%%%%%%%%%%%%%%%%%%%%%%%%%%%%%%%%%%%%%%%%%%
\begin{exer}
Compute $a^{p-1} \pmod{p}$ for various numbers $a$ and primes $p$,
and make a conjecture.
\end{exer}

%%%Theorem 4.13%%%%%%%%%%%%%%%%%%%%%%%%%%%%%%%%%%%%%%%%%%%%%%%%%%%%%%%%%%%%%%%%%%%%%

\begin{thm}
Let $p$ be a prime and let $a$ be an integer not divisible by $p$;
that is, $(a, p) = 1$.  Then $\{a, 2a, 3a, \hdots, pa\}$ is a
complete residue system modulo $p$.
\end{thm}

%%%Theorem 4.14%%%%%%%%%%%%%%%%%%%%%%%%%%%%%%%%%%%%%%%%%%%%%%%%%%%%%%%%%%%%%%%%%%%%%

\begin{thm}
Let $p$ be a prime and let $a$ be an integer not divisible by $p$.
Then
\[a \cdot 2a \cdot 3a \cdot \hdots \cdot (p-1)a \equiv
    1 \cdot 2 \cdot 3 \cdot \hdots \cdot (p-1) \pmod{p}.\]
\end{thm}

%%%Theorem 4.15%%%%%%%%%%%%%%%%%%%%%%%%%%%%%%%%%%%%%%%%%%%%%%%%%%%%%%%%%%%%%%%%%%%%%

\begin{thm}[Fermat's Little Theorem, Version I]
If $p$ is a prime and $a$ is an integer relatively prime to $p$,
then $a^{(p-1)} \equiv 1 \pmod{p}$.
\end{thm}

%%%Theorem 4.16%%%%%%%%%%%%%%%%%%%%%%%%%%%%%%%%%%%%%%%%%%%%%%%%%%%%%%%%%%%%%%%%%%%%%
\begin{thm}[Fermat's Little Theorem, Version II]
If $p$ is a prime and $a$ is \emph{any} integer, then $a^p \equiv a
\pmod{p}$.
\end{thm}

%%%Theorem 4.17%%%%%%%%%%%%%%%%%%%%%%%%%%%%%%%%%%%%%%%%%%%%%%%%%%%%%%%%%%%%%%%%%%%%%
\begin{thm}
The two versions of  Fermat's Little Theorem stated above are
equivalent to one another, that is, each one can be deduced from the
other.
\end{thm}

%%%Theorem 4.18%%%%%%%%%%%%%%%%%%%%%%%%%%%%%%%%%%%%%%%%%%%%%%%%%%%%%%%%%%%%%%%%%%%%%
\begin{thm}
Let $p$ be a prime and $a$ be an integer. If $(a,p)=1$, then
$\operatorname{ord}_p(a)$ divides $p-1$, that is,
$\operatorname{ord}_p(a)| p-1$.
\end{thm}

%%%Exercise 4.19%%%%%%%%%%%%%%%%%%%%%%%%%%%%%%%%%%%%%%%%%%%%%%%%%%%%%%%%%%%%%%%%%%%%%
\begin{exer}
Compute each of the following without the aid of a calculator or
computer.
\begin{enumerate}
\item $512^{372} \pmod{13}$.
\item $3444^{3233} \pmod{17}$.
\item $123^{456} \pmod{23}$.
\end{enumerate}
\end{exer}

%%%Exercise 4.20%%%%%%%%%%%%%%%%%%%%%%%%%%%%%%%%%%%%%%%%%%%%%%%%%%%%%%%%%%%%%%%%%%%%%
\begin{exer}
Find the remainder upon division of $314^{159}$ by $31$.
\end{exer}

%%%Theorem 4.21%%%%%%%%%%%%%%%%%%%%%%%%%%%%%%%%%%%%%%%%%%%%%%%%%%%%%%%%%%%%%%%%%%%%%
\begin{thm}
Let $n$ and $m$ be natural numbers that are relatively prime, and
let $a$ be an integer.  If $x \equiv a \pmod{n}$ and $x \equiv a
\pmod{m}$, then $x \equiv a \pmod{nm}$.
\end{thm}


%%%Exercise 4.22%%%%%%%%%%%%%%%%%%%%%%%%%%%%%%%%%%%%%%%%%%%%%%%%%%%%%%%%%%%%%%%%%%%%%
\begin{exer}
Find the remainder when $4^{72}$ is divided by $91$ $(=7\cdot 13)$.
\end{exer}


%%%Exercise 4.23%%%%%%%%%%%%%%%%%%%%%%%%%%%%%%%%%%%%%%%%%%%%%%%%%%%%%%%%%%%%%%%%%%%%%

\begin{exer}
Find the natural number $k < 117$ such that $2^{117} \equiv k
\pmod{117}$. (Notice that $117$ is not prime.)
\end{exer}

%%%%%%%%%%%%%%%%%%%%%%%%%%%%%%%%%%%%%%%%%%%%%%%%%%%%%%%%%%%%%%

\subsection*{An alternative route to Fermat's Little Theorem}

%%%Theorem 4.24%%%%%%%%%%%%%%%%%%%%%%%%%%%%%%%%%%%%%%%%%%%%%%%%%%%%%%%%%%%%%%%%%%%%%
\begin{thm}[Binomial Theorem]
Let $a$ and $b$ be numbers and let $n$ be a natural number.  Then
\[(a+b)^n = \sum_{i=0}^n \binom{n}{i} a^{n-i}b^i.\]
\end{thm}

%%%Lemma 4.25%%%%%%%%%%%%%%%%%%%%%%%%%%%%%%%%%%%%%%%%%%%%%%%%%%%%%%%%%%%%%%%%%%%%%
\begin{lem}
If $p$ is prime and $i$ is a natural number less than $p$, then $p$
divides $\binom{p}{i}$.
\end{lem}

%%%Theorem 4.26%%%%%%%%%%%%%%%%%%%%%%%%%%%%%%%%%%%%%%%%%%%%%%%%%%%%%%%%%%%%%%%%%%%%%

\begin{thm}[Fermat's Little Theorem, Version II]
If $p$ is a prime and $a$ is an integer, then $a^p \equiv a
\pmod{p}$.
\end{thm}


%%%%%%%%%%%%%%%%%%%%%%%%%%%%%%%%%%%%%%%%%%%%%%%%%%%%%%%%%%%%%%%

\subsection*{Euler's Theorem and Wilson's Theorem}


%%%Question 4.27%%%%%%%%%%%%%%%%%%%%%%%%%%%%%%%%%%%%%%%%%%%%%%%%%%%%%%%%%%%%%%%%%%%%%
\begin{ques}
The numbers $1$, $5$, $7$, and $11$ are all the natural numbers less
than or equal to $12$ that are relatively prime to $12$, so
$\phi(12) = 4$.
\begin{enumerate}
\item What is $\phi(7)$?
\item What is $\phi(15)$?
\item What is $\phi(21)$?
\item What is $\phi(35)$?
\end{enumerate}
\end{ques}

%%%Theorem 4.28%%%%%%%%%%%%%%%%%%%%%%%%%%%%%%%%%%%%%%%%%%%%%%%%%%%%%%%%%%%%%%%%%%%%%
\begin{thm}
Let $a$, $b$, and $n$ be integers such that $(a, n) = 1$ and $(b, n)
= 1$. Then~\mbox{$(ab, n) = 1$}.
\end{thm}

%%%Theorem 4.29%%%%%%%%%%%%%%%%%%%%%%%%%%%%%%%%%%%%%%%%%%%%%%%%%%%%%%%%%%%%%%%%%%%%%
\begin{thm}
Let $a$, $b$, and $n$ be integers with $n > 0$. If $a \equiv b
\pmod{n}$ and $(a, n) = 1$, then~$(b, n) = 1$.
\end{thm}

%%%Theorem 4.30%%%%%%%%%%%%%%%%%%%%%%%%%%%%%%%%%%%%%%%%%%%%%%%%%%%%%%%%%%%%%%%%%%%%%
\begin{thm}
Let $a$, $b$, $c$, and $n$ be integers with $n
> 0$.  If $ab \equiv ac \pmod{n}$ and $(a, n) = 1$, then $b \equiv c
\pmod{n}$.
\end{thm}

%%%Theorem 4.31%%%%%%%%%%%%%%%%%%%%%%%%%%%%%%%%%%%%%%%%%%%%%%%%%%%%%%%%%%%%%%%%%%%%%
\begin{thm}
Let $n$ be a natural number and let $x_1$, $x_2$, $\hdots$,
$x_{\phi(n)}$ be the distinct natural numbers less than or equal to
$n$ that are relatively prime to $n$.  Let $a$ be a non-zero integer
relatively prime to $n$ and let $i$ and $j$ be different natural
numbers less than or equal to~$\phi(n)$.  Then~$ax_i \not \equiv
ax_j \pmod{n}$.
\end{thm}

%%%Theorem 4.32%%%%%%%%%%%%%%%%%%%%%%%%%%%%%%%%%%%%%%%%%%%%%%%%%%%%%%%%%%%%%%%%%%%%%
\begin{thm}[Euler's Theorem]
If $a$ and $n$ are integers with $n > 0$ and $(a, n) = 1$, then \[
a^{\phi(n)} \equiv 1 \pmod{n}. \]
\end{thm}

%%%Theorem 4.33%%%%%%%%%%%%%%%%%%%%%%%%%%%%%%%%%%%%%%%%%%%%%%%%%%%%%%%%%%%%%%%%%%%%%
\begin{cor}[Fermat's Little Theorem]
If $p$ is a prime and $a$ is an integer relatively prime to $p$,
then $a^{(p-1)} \equiv 1 \pmod{p}$.
\end{cor}


%%%Exercise 4.34%%%%%%%%%%%%%%%%%%%%%%%%%%%%%%%%%%%%%%%%%%%%%%%%%%%%%%%%%%%%%%%%%%%%%
\begin{exer}
Compute each of the following without the aid of a calculator or
computer.
\begin{enumerate}
\item $12^{49} \pmod{15}$.
\item $139^{112} \pmod{27}$.
\end{enumerate}
\end{exer}

%%%Exercise 4.35%%%%%%%%%%%%%%%%%%%%%%%%%%%%%%%%%%%%%%%%%%%%%%%%%%%%%%%%%%%%%%%%%%%%%
\begin{exer}
Find the last digit in the base $10$ representation of the integer
$13^{474}$.
\end{exer}

%%%Theorem 4.36%%%%%%%%%%%%%%%%%%%%%%%%%%%%%%%%%%%%%%%%%%%%%%%%%%%%%%%%%%%%%%%%%%%%%
\begin{thm}
Let $p$ be a prime and let $a$ be an integer such that $1 \leq a <
p$.  Then there exists a unique natural number $b$ less than $p$
such that $ab \equiv 1 \pmod{p}$.
\end{thm}

%%%Exercise 4.37%%%%%%%%%%%%%%%%%%%%%%%%%%%%%%%%%%%%%%%%%%%%%%%%%%%%%%%%%%%%%%%%%%%%%
\begin{exer}
Let $p$ be a prime. Show that the natural numbers $1$ and $p-1$ are
their own inverses modulo $p$.
\end{exer}


%%%Theorem 4.38%%%%%%%%%%%%%%%%%%%%%%%%%%%%%%%%%%%%%%%%%%%%%%%%%%%%%%%%%%%%%%%%%%%%%
\begin{thm}
Let $p$ be a prime and let $a$ and $b$ be integers such that $1 <
a,b < p-1$ and~\mbox{$ab \equiv 1 \pmod{p}$}. Then $a \neq b$.
\end{thm}

%%Exercise 4.39%%%%%%%%%%%%%%%%%%%%%%%%%%%%%%%%%%%%%%%%%%%%%%%%%%%%%%%%%%%%%%%%%%%%%
\begin{exer}
 Find all pairs of numbers $a$ and $b$ in
$\{2, 3, \dots, 11\}$ such that $ab \equiv 1 \pmod{13}$.
\end{exer}


%%%Theorem 4.40%%%%%%%%%%%%%%%%%%%%%%%%%%%%%%%%%%%%%%%%%%%%%%%%%%%%%%%%%%%%%%%%%%%%%
\begin{thm}
If $p$ is a prime larger than $2$, then $2 \cdot 3 \cdot 4 \cdot
\hdots \cdot (p-2) \equiv 1 \pmod{p}$.
\end{thm}

%%%Theorem 4.41%%%%%%%%%%%%%%%%%%%%%%%%%%%%%%%%%%%%%%%%%%%%%%%%%%%%%%%%%%%%%%%%%%%%%

\begin{thm}[Wilson's Theorem]
If $p$ is a prime, then $(p - 1)! \equiv -1 \pmod{p}$.
\end{thm}

%%%Theorem 4.42%%%%%%%%%%%%%%%%%%%%%%%%%%%%%%%%%%%%%%%%%%%%%%%%%%%%%%%%%%%%%%%%%%%%%
\begin{thm}[Converse of Wilson's Theorem]
If $n$ is a natural number such that $(n - 1)! \equiv -1 \pmod{n}$,
then $n$ is prime.
\end{thm}


%%%BPE 4.43%%%%%%%%%%%%%%%%%%%%%%%%%%%%%%%%%%%%%%%%%%%%%%%%%%%%%%%%%%%%%%%%%%%%%
\begin{bpe}
Chapter 4 is the culmination of all of your inquiries from the first
three chapters. After not looking at the material for a day or two,
take a blank piece of paper and outline the development of the first
four chapters in as much detail as you can without referring to the
text or to notes. Places where you get stuck or can't remember
highlight areas that may call for further study.
\end{bpe}

%%%%%%%%%%%%%%%%%%%%%%%%%%%%%%%%%%%%%%%%%%%%%%%%%%%%%%%%%


\end{document}
