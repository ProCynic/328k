\documentclass[12pt,leqno]{article}

\usepackage{amsmath,amsfonts,amssymb,amscd,amsthm,amsbsy,upref}


\textheight=8.5truein
\textwidth=6.0truein
\hoffset=-.5truein
\voffset=-.5truein
\numberwithin{equation}{section}
\pagestyle{headings}
\footskip=36pt


\swapnumbers
\newtheorem{thm}{Theorem}[section]
\newtheorem{hthm}[thm]{*Theorem}
\newtheorem{lem}[thm]{Lemma}
\newtheorem{cor}[thm]{Corollary}
\newtheorem{prop}[thm]{Proposition}
\newtheorem{con}[thm]{Conjecture}
\newtheorem{exer}[thm]{Exercise}
\newtheorem{bpe}[thm]{Blank Paper Exercise}
\newtheorem{apex}[thm]{Applications Exercise}
\newtheorem{ques}[thm]{Question}
\newtheorem{scho}[thm]{Scholium}
\newtheorem*{Exthm}{Example Theorem}
\newtheorem*{Thm}{Theorem}
\newtheorem*{Con}{Conjecture}
\newtheorem*{Axiom}{Axiom}



\theoremstyle{definition}
\newtheorem*{Ex}{Example}
\newtheorem*{Def}{Definition}


\newcommand{\lcm}{\operatorname{lcm}}
\newcommand{\ord}{\operatorname{ord}}
\def\pfrac#1#2{{\left(\frac{#1}{#2}\right)}}


\makeindex

\begin{document}




\thispagestyle{plain}
\begin{flushright}
\large{\textbf{Geoffrey Parker - grp352\\
HW 23: 4.18 - 4.23\\
M328K \\
April 19th, 2012 \\}}
\end{flushright}

%\maketitle
\markboth{}{} \setcounter{section}{0} \baselineskip=18pt

\setcounter{tocdepth}{4}


%%%%%%%%This is where you can change the numbering to match the problem you
%%%%%%%%are on.  Set the section to  the current chapter.

\setcounter{section}{4}

%%%%%%%%%Now, set the theorem number to one less than the first theorem in
%%%%%%%%%the assignment.
\setcounter{thm}{17}

%%%Theorem 4.18%%%%%%%%%%%%%%%%%%%%%%%%%%%%%%%%%%%%%%%%%%%%%%%%%%%%%%%%%%%%%%%%%%%%%
\begin{thm}
Let $p$ be a prime and $a$ be an integer. If $(a,p)=1$, then
$\operatorname{ord}_p(a)$ divides $p-1$, that is,
$\operatorname{ord}_p(a)| p-1$.
\end{thm}
\begin{proof}[Proof]
Let $p$ be a prime and $a$ be an integer with $(a,p)=1$.  Let $k = ord_p(a)$.  By Fermat's Little Theorem we know that $a^{p-1} \equiv 1 \pmod{p}$, so by theorem 4.10 $k \mid p-1$.
\end{proof}

%%%Exercise 4.19%%%%%%%%%%%%%%%%%%%%%%%%%%%%%%%%%%%%%%%%%%%%%%%%%%%%%%%%%%%%%%%%%%%%%
\begin{exer}
Compute each of the following without the aid of a calculator or
computer.
\begin{enumerate}
\item $512^{372} \pmod{13}$.
\item $3444^{3233} \pmod{17}$.
\item $123^{456} \pmod{23}$.
\end{enumerate}
\end{exer}
\begin{proof}[Solution]$ $\\
\begin{enumerate}
\item
We know that $13 - 1 = 12$ and $372 = 31 * 12$, so $ord_{13}(512) \mid 372$.\\  Therefore $512^{372} \pmod{13} = 1$.
\item
$3444^{3233} \pmod{17}$.
\item
$123^{456} \pmod{23}$.
\end{enumerate}
\end{proof}

%%%Exercise 4.20%%%%%%%%%%%%%%%%%%%%%%%%%%%%%%%%%%%%%%%%%%%%%%%%%%%%%%%%%%%%%%%%%%%%%
\begin{exer}
Find the remainder upon division of $314^{159}$ by $31$.
\end{exer}
\begin{proof}[Solution]
\end{proof}

\pagebreak
%%%Theorem 4.21%%%%%%%%%%%%%%%%%%%%%%%%%%%%%%%%%%%%%%%%%%%%%%%%%%%%%%%%%%%%%%%%%%%%%
\begin{thm}
Let $n$ and $m$ be natural numbers that are relatively prime, and
let $a$ be an integer.  If $x \equiv a \pmod{n}$ and $x \equiv a
\pmod{m}$, then $x \equiv a \pmod{nm}$.
\end{thm}
\begin{proof}[Proof]
  Let $n$ and $m$ be natural numbers with $(n, m) = 1$.  Let $a$ and $x$ be integers with $x \equiv a \pmod{n}$ and $x \equiv a \pmod{m}$.  So $n \mid x - a$ and $m \mid x - a$.  So by theorem 1.42 $nm \mid x - a$.  Therefore $x \equiv a \pmod{nm}$.
\end{proof}

%%%Exercise 4.22%%%%%%%%%%%%%%%%%%%%%%%%%%%%%%%%%%%%%%%%%%%%%%%%%%%%%%%%%%%%%%%%%%%%%
\begin{exer}
Find the remainder when $4^{72}$ is divided by $91$ $(=7\cdot 13)$.
\end{exer}
\begin{proof}[Solution]
\end{proof}

%%%Exercise 4.23%%%%%%%%%%%%%%%%%%%%%%%%%%%%%%%%%%%%%%%%%%%%%%%%%%%%%%%%%%%%%%%%%%%%%

\begin{exer}
Find the natural number $k < 117$ such that $2^{117} \equiv k
\pmod{117}$. (Notice that $117$ is not prime.)
\end{exer}
\begin{proof}[Solution]
\end{proof}
\end{document}
