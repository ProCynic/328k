
\documentclass[12pt,leqno]{article}

\usepackage{amsmath,amsfonts,amssymb,amscd,amsthm,amsbsy,upref}


\textheight=8.5truein
\textwidth=6.0truein
\hoffset=-.5truein
\voffset=-.5truein
\numberwithin{equation}{section}
\pagestyle{headings}
\footskip=36pt


\swapnumbers
\newtheorem{thm}{Theorem}[section]
\newtheorem{hthm}[thm]{*Theorem}
\newtheorem{lem}[thm]{Lemma}
\newtheorem{cor}[thm]{Corollary}
\newtheorem{prop}[thm]{Proposition}
\newtheorem{con}[thm]{Conjecture}
\newtheorem{exer}[thm]{Exercise}
\newtheorem{bpe}[thm]{Blank Paper Exercise}
\newtheorem{apex}[thm]{Applications Exercise}
\newtheorem{ques}[thm]{Question}
\newtheorem{scho}[thm]{Scholium}
\newtheorem*{Exthm}{Example Theorem}
\newtheorem*{Thm}{Theorem}
\newtheorem*{Con}{Conjecture}
\newtheorem*{Axiom}{Axiom}



\theoremstyle{definition}
\newtheorem*{Ex}{Example}
\newtheorem*{Def}{Definition}


\newcommand{\lcm}{\operatorname{lcm}}
\newcommand{\ord}{\operatorname{ord}}
\def\pfrac#1#2{{\left(\frac{#1}{#2}\right)}}


\makeindex

\begin{document}




\thispagestyle{plain}
\begin{flushright}
\large{\textbf{TYPE YOUR NAME HERE \\
HW 1: 1.1 - 1.3, 1.6 - 1.14\\
M328K \\
January 19th, 2012 \\}}
\end{flushright}

%\maketitle
\markboth{}{} \setcounter{section}{0} \baselineskip=18pt

\setcounter{tocdepth}{4}


%%%%%%%%This is where you can change the numbering to match the problem you
%%%%%%%%are on.  Set the section to  the current chapter.

\setcounter{section}{1}
%%%%%%%%%%%%1.1%%%%%%%%%%%%%%%%%%%%%%%%%%%%%%%%%%%%%%%%%%%%%%%%%%%%%
\begin{thm}
Let $a$, $b$, and $c$ be integers.  If $a|b$ and $a|c$, then $a|(b +
c)$.
\end{thm}

\begin{proof}[Proof]
Type your proof here!
\end{proof}

%%%%%%%%%%%%1.2%%%%%%%%%%%%%%%%%%%%%%%%%%%%%%%%%%%%%%%%%%%%%%%%%%%%%
\begin{thm}
Let $a$, $b$, and $c$ be integers. If $a|b$ and $a|c$, then $a|(b -
c)$.
\end{thm}

\begin{proof}[Proof]
Type your proof here!
\end{proof}

%%%%%%%%%%%%1.3%%%%%%%%%%%%%%%%%%%%%%%%%%%%%%%%%%%%%%%%%%%%%%%%%%%%%
\begin{thm}
Let $a$, $b$, and $c$ be integers. If $a|b$ and $a|c$, then $a|bc$.
\end{thm}

\begin{proof}[Proof]
Type your proof here!
\end{proof}

%%%%%%%%%Now, set the thorem number to one less than the first theorem in the assignment.
\setcounter{thm}{5}

%%%%%%%%%%%%1.6%%%%%%%%%%%%%%%%%%%%%%%%%%%%%%%%%%%%%%%%%%%%%%%%%%%%%
\begin{thm}
Let $a$, $b$, and $c$ be integers. If $a|b$, then $a|bc$.
\end{thm}

\begin{proof}[Proof]
Type your proof here!
\end{proof}

%%%%%%%%%%%%1.7%%%%%%%%%%%%%%%%%%%%%%%%%%%%%%%%%%%%%%%%%%%%%%%%%%%%%
\begin{exer} Answer each of the following questions, and prove that your answer is
correct. \end{exer}
\begin{enumerate}
\item[(1)] Is $45 \equiv 9\pmod{4}$?
\begin{proof}[Proof]
Type your proof here!
\end{proof}

\item[(2)] Is $37 \equiv 2 \pmod{5}$?
\begin{proof}[Proof]
Type your proof here!
\end{proof}

\item[(3)] Is ${37 \equiv 3 \pmod{5}}$?
\begin{proof}[Proof]
Type your proof here!
\end{proof}

\item[(4)] Is $31 \equiv -3 \pmod{5}$?
\begin{proof}[Proof]
Type your proof here!
\end{proof}

\end{enumerate}


%%%%%%%%%%%%1.8%%%%%%%%%%%%%%%%%%%%%%%%%%%%%%%%%%%%%%%%%%%%%%%%%%%%%
\begin{exer}  For each of the following congruences,
characterize all the integers $m$ that satisfy that congruence.
\end{exer}
\begin{enumerate}
    \item[(1)] $m \equiv 0 \pmod{3}$.
    \begin{proof}[Solution]
    Type your solution here!
    \end{proof}

    \item[(2)] $m \equiv 1 \pmod{3}$.
    \begin{proof}[Solution]
    Type your solution here!
    \end{proof}

    \item[(3)] $m \equiv 2 \pmod{3}$.
    \begin{proof}[Solution]
    Type your solution here!
    \end{proof}

    \item[(4)] $m \equiv 3 \pmod{3}$.
    \begin{proof}[Solution]
    Type your solution here!
    \end{proof}

    \item[(5)] $m \equiv 4 \pmod{3}$.
    \begin{proof}[Solution]
    Type your solution here!
    \end{proof}

\end{enumerate}

%%%%%%%%%%%%1.9%%%%%%%%%%%%%%%%%%%%%%%%%%%%%%%%%%%%%%%%%%%%%%%%%%%%%
\begin{thm}
Let $a$ and $n$ be integers with $n > 0$. Then $a \equiv a
\pmod{n}$.
\end{thm}

\begin{proof}[Proof]
Type your proof here!
\end{proof}

%%%%%%%%%%%%1.10%%%%%%%%%%%%%%%%%%%%%%%%%%%%%%%%%%%%%%%%%%%%%%%%%%%%%
\begin{thm}
Let $a$, $b$, and $n$ be integers with $n > 0$. If $a \equiv b
\pmod{n}$, then ${b \equiv a \pmod{n}}$.
\end{thm}

\begin{proof}[Proof]
Type your proof here!
\end{proof}

%%%%%%%%%%%%1.11%%%%%%%%%%%%%%%%%%%%%%%%%%%%%%%%%%%%%%%%%%%%%%%%%%%%%
\begin{thm}
Let $a$, $b$, $c$, and $n$ be integers with $n > 0$. If $a \equiv b
\pmod{n}$ and ${b \equiv c \pmod{n}}$, then $a \equiv c \pmod{n}$.
\end{thm}

\begin{proof}[Proof]
Type your proof here!
\end{proof}

%%%%%%%%%%%%1.12%%%%%%%%%%%%%%%%%%%%%%%%%%%%%%%%%%%%%%%%%%%%%%%%%%%%%
\begin{thm}
Let $a$, $b$, $c$, $d$, and $n$ be integers with $n > 0$. If $a
\equiv b \pmod{n}$ and~${c \equiv d \pmod{n}}$, then $a + c \equiv b
+ d \pmod{n}$.
\end{thm}

\begin{proof}[Proof]
Type your proof here!
\end{proof}

%%%%%%%%%%%%1.13%%%%%%%%%%%%%%%%%%%%%%%%%%%%%%%%%%%%%%%%%%%%%%%%%%%%%
\begin{thm}
Let $a$, $b$, $c$, $d$, and $n$ be integers with $n > 0$. If $a
\equiv b \pmod{n}$ and~${c \equiv d \pmod{n}}$, then $a - c \equiv b
- d \pmod{n}$.
\end{thm}

\begin{proof}[Proof]
Type your proof here!
\end{proof}

%%%%%%%%%%%%1.14%%%%%%%%%%%%%%%%%%%%%%%%%%%%%%%%%%%%%%%%%%%%%%%%%%%%%
\begin{thm}
Let $a$, $b$, $c$, $d$, and $n$ be integers
with $n > 0$. If $a \equiv b \pmod{n}$ and~${c \equiv d \pmod{n}}$, then
$ac \equiv bd \pmod{n}$.
\end{thm}

\begin{proof}[Proof]
Type your proof here!
\end{proof}


\end{document}
