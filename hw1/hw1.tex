
\documentclass[12pt,leqno]{article}

\usepackage{amsmath,amsfonts,amssymb,amscd,amsthm,amsbsy,upref}


\textheight=8.5truein
\textwidth=6.0truein
\hoffset=-.5truein
\voffset=-.5truein
\numberwithin{equation}{section}
\pagestyle{headings}
\footskip=36pt


\swapnumbers
\newtheorem{thm}{Theorem}[section]
\newtheorem{hthm}[thm]{*Theorem}
\newtheorem{lem}[thm]{Lemma}
\newtheorem{cor}[thm]{Corollary}
\newtheorem{prop}[thm]{Proposition}
\newtheorem{con}[thm]{Conjecture}
\newtheorem{exer}[thm]{Exercise}
\newtheorem{bpe}[thm]{Blank Paper Exercise}
\newtheorem{apex}[thm]{Applications Exercise}
\newtheorem{ques}[thm]{Question}
\newtheorem{scho}[thm]{Scholium}
\newtheorem*{Exthm}{Example Theorem}
\newtheorem*{Thm}{Theorem}
\newtheorem*{Con}{Conjecture}
\newtheorem*{Axiom}{Axiom}



\theoremstyle{definition}
\newtheorem*{Ex}{Example}
\newtheorem*{Def}{Definition}


\newcommand{\lcm}{\operatorname{lcm}}
\newcommand{\ord}{\operatorname{ord}}
\def\pfrac#1#2{{\left(\frac{#1}{#2}\right)}}


\makeindex

\begin{document}




\thispagestyle{plain}
\begin{flushright}
\large{\textbf{Geoffrey Parker - grp352\\
HW 1: 1.1 - 1.3, 1.6 - 1.14\\
M328K \\
January 19th, 2012 \\}}
\end{flushright}

%\maketitle
\markboth{}{} \setcounter{section}{0} \baselineskip=18pt

\setcounter{tocdepth}{4}


%%%%%%%%This is where you can change the numbering to match the problem you
%%%%%%%%are on.  Set the section to  the current chapter.

\setcounter{section}{1}
%%%%%%%%%%%%1.1%%%%%%%%%%%%%%%%%%%%%%%%%%%%%%%%%%%%%%%%%%%%%%%%%%%%%
\begin{thm}
Let $a$, $b$, and $c$ be integers.  If $a|b$ and $a|c$, then $a|(b +
c)$.
\end{thm}

\begin{proof}[Proof]
Let $a$, $b$, and $c$ be integers where $a|b$ and $a|c$.  We will show that, given this, $a|(b + c)$.  Since $a|b$ and $a|c$, then by the definition of divides $b = aj$ and $c = ak$ for some integers $j$ and $k$.  Therefore $b + c = aj + ak$ and $b + c = a(j + k)$.  Because $j$ and $k$ are both integers, $j + k$ is also an integer.  So by the definition of divides, $a|(b + c)$.
\end{proof}

%%%%%%%%%%%%1.2%%%%%%%%%%%%%%%%%%%%%%%%%%%%%%%%%%%%%%%%%%%%%%%%%%%%%
\begin{thm}
Let $a$, $b$, and $c$ be integers. If $a|b$ and $a|c$, then $a|(b -
c)$.
\end{thm}

\begin{proof}[Proof]
Let $a$, $b$, and $c$ be integers where $a|b$ and $a|c$.  We will show that, given this, $a|(b - c)$.  Since $a|b$ and $a|c$, then by the definition of divides $b = aj$ and $c = ak$ for some integers $j$ and $k$.  Therefore $b - c = aj - ak$ and $b - c = a(j - k)$.  Because $j$ and $k$ are both integers, $j - k$ is also an integer.  So by the definition of divides, $a|(b - c)$.\end{proof}

%%%%%%%%%%%%1.3%%%%%%%%%%%%%%%%%%%%%%%%%%%%%%%%%%%%%%%%%%%%%%%%%%%%%
\begin{thm}
Let $a$, $b$, and $c$ be integers. If $a|b$ and $a|c$, then $a|bc$.
\end{thm}

\begin{proof}[Proof]
Let $a$, $b$, and $c$ be integers where $a|b$ and $a|c$.  We will show that, given this, $a|bc$.  Since $a|b$ and $a|c$, then by the definition of divides $b = aj$ and $c = ak$ for some integers $j$ and $k$.  Therefore $bc = ajak$ and $b + c = a(ajk)$.  Because $a$, $j$, and $k$ are all integers, $ajk$ is also an integer.  So by the definition of divides, $a|bc$.\end{proof}

%%%%%%%%%Now, set the thorem number to one less than the first theorem in the assignment.
\setcounter{thm}{5}

%%%%%%%%%%%%1.6%%%%%%%%%%%%%%%%%%%%%%%%%%%%%%%%%%%%%%%%%%%%%%%%%%%%%
\begin{thm}
Let $a$, $b$, and $c$ be integers. If $a|b$, then $a|bc$.
\end{thm}

\begin{proof}[Proof]
Let $a$, $b$, and $c$ be integers where $a|b$.  We will show that $a|bc$.  Since $a|b$, then by the definition of divides $b = ak$ for some integer $k$.  Therefore $bc = akc$. Because $k$ and $c$ are both integers, $kc$ is also an integer.  So by the definition of divides, $a|bc$.
\end{proof}

\pagebreak
%%%%%%%%%%%%1.7%%%%%%%%%%%%%%%%%%%%%%%%%%%%%%%%%%%%%%%%%%%%%%%%%%%%%
\begin{exer} Answer each of the following questions, and prove that your answer is
correct. \end{exer}
\begin{enumerate}
\item[(1)] Is $45 \equiv 9\pmod{4}$?
\\Yes.
\begin{proof}[Proof]
$45-9 = 36.$  $36 = 4 \times 9.$  Therefore $4|36$.  So by the definition of congruence, $45 \equiv 9\pmod{4}.$
\end{proof}

\item[(2)] Is $37 \equiv 2 \pmod{5}$?
\\Yes.
\begin{proof}[Proof]
$37-2 = 35.$  $35 = 5 \times 7.$  Therefore $5|35$.  So by the definition of congruence, $37 \equiv 2\pmod{5}.$
\end{proof}

\item[(3)] Is ${37 \equiv 3 \pmod{5}}$?
\\No.
\begin{proof}[Proof]
$37-3 = 34.$  $5 \times 6 = 30$ and $5 \times 7 = 35$.  Since $30 < 34 < 35$, there is no integer x such that $5 \times x = 34$. Therefore by the definition of divides, $5 \nmid 34$.  So by the definition of congruence, $37 \not \equiv 3\pmod{5}.$
\end{proof}

\item[(4)] Is $31 \equiv -3 \pmod{5}$?
\\No.
\begin{proof}[Proof]
$31-(-3) = 34.$  $5 \times 6 = 30$ and $5 \times 7 = 35$.  Since $30 < 34 < 35$, there is no integer x such that $5 \times x = 34$. Therefore by the definition of divides, $5 \nmid 34$.  So by the definition of congruence, $31 \not \equiv -3\pmod{5}.$
\end{proof}

\end{enumerate}


%%%%%%%%%%%%1.8%%%%%%%%%%%%%%%%%%%%%%%%%%%%%%%%%%%%%%%%%%%%%%%%%%%%%
\begin{exer}  For each of the following congruences,
characterize all the integers $m$ that satisfy that congruence.
\end{exer}
\begin{enumerate}
    \item[(1)] $m \equiv 0 \pmod{3}$.
    \begin{proof}[Solution]
    This is satisfied when $m = 3n$ for any integer $n$.
    \end{proof}

	\pagebreak

    \item[(2)] $m \equiv 1 \pmod{3}$.
    \begin{proof}[Solution]
    This is satisfied when $m = 3n + 1$ for any integer $n$.
    \end{proof}

    \item[(3)] $m \equiv 2 \pmod{3}$.
    \begin{proof}[Solution]
    This is satisfied when $m = 3n + 2$ for any integer $n$.
    \end{proof}

    \item[(4)] $m \equiv 3 \pmod{3}$.
    \begin{proof}[Solution]
    This is satisfied when $m = 3n$ for any integer $n$.
    \end{proof}

    \item[(5)] $m \equiv 4 \pmod{3}$.
    \begin{proof}[Solution]
    This is satisfied when $m = 3n + 1$ for any integer $n$.
    \end{proof}

\end{enumerate}

%%%%%%%%%%%%1.9%%%%%%%%%%%%%%%%%%%%%%%%%%%%%%%%%%%%%%%%%%%%%%%%%%%%%
\begin{thm}
Let $a$ and $n$ be integers with $n > 0$. Then $a \equiv a
\pmod{n}$.
\end{thm}

\begin{proof}[Proof]
Let $a$ and $n$ be integers with $n > 0$.  We will show that $a \equiv a
\pmod{n}$.  To start with, $a - a = 0$.  Since all integers divide 0, $n 
\mid (a - a)$.  Therefore, by definition of congruence, $a \equiv a
\pmod{n}$.
\end{proof}

%%%%%%%%%%%%1.10%%%%%%%%%%%%%%%%%%%%%%%%%%%%%%%%%%%%%%%%%%%%%%%%%%%%%
\begin{thm}
Let $a$, $b$, and $n$ be integers with $n > 0$. If $a \equiv b
\pmod{n}$, then ${b \equiv a \pmod{n}}$.
\end{thm}
\begin{proof}[Proof]
Let $a$, $b$, and $n$ be integers with $n > 0$ and $a \equiv b
\pmod{n}$.  We will show that ${b \equiv a \pmod{n}}$.  Since $a \equiv b
\pmod{n}$, then by definition of congruence, $n|(a - b)$.  This in turn means, by definition of divides, that $a - b = nk$ for some integer $k$.  From here, we can say that $b - a = -(a - b) = -(nk) = n(-k)$.  Since $k$ is an integer, $-k$ is also an integer.  Therefore $n|(b - a)$, which means that ${b \equiv a \pmod{n}}$.
\end{proof}

%%%%%%%%%%%%1.11%%%%%%%%%%%%%%%%%%%%%%%%%%%%%%%%%%%%%%%%%%%%%%%%%%%%%
\begin{thm}
Let $a$, $b$, $c$, and $n$ be integers with $n > 0$. If $a \equiv b
\pmod{n}$ and ${b \equiv c \pmod{n}}$, then $a \equiv c \pmod{n}$.
\end{thm}

\begin{proof}[Proof]
Let $a$, $b$, $c$, and $n$ be integers with $n > 0$, $a \equiv b
\pmod{n}$, and ${b \equiv c \pmod{n}}$. We will show $a \equiv c \pmod{n}$.  By definition of congruence, we have $n|(a - b)$ and $n|(b - c)$.  By definition of divides, this gives us $a - b = nj$ and $b - c = nk$ for some integers $j$ and $k$.  So $a - c = (a - b) + (b - c) = nj + nk = n(j + k)$.  Since $j$ and $k$ are integers, $j + k$ is also an integer.  Therefore $n|(a - c)$, which means that $a \equiv c \pmod{n}$.
\end{proof}

%%%%%%%%%%%%1.12%%%%%%%%%%%%%%%%%%%%%%%%%%%%%%%%%%%%%%%%%%%%%%%%%%%%%
\begin{thm}
Let $a$, $b$, $c$, $d$, and $n$ be integers with $n > 0$. If $a
\equiv b \pmod{n}$ and~${c \equiv d \pmod{n}}$, then $a + c \equiv b
+ d \pmod{n}$.
\end{thm}

\begin{proof}[Proof]
Let $a$, $b$, $c$, $d$, and $n$ be integers with $n > 0$, $a
\equiv b \pmod{n}$, and ${c \equiv d \pmod{n}}$. We will show that $a + c \equiv b + d \pmod{n}$.  By definition of congruence, we have $n|(a - b)$ and $n|(c - d)$.  By definition of divides, this gives us $a - b = nj$ and $c - d = nk$ for some integers $j$ and $k$.  So $(a + c) - (b + d) = (a - b) + (c - d) = nj + nk = n(j + k)$.  Since $j$ and $k$ are integers, $j + k$ is also an integer.  Therefore $n|(a + c) - (b + d)$, which means that $a + c \equiv b + d \pmod{n}$.
\end{proof}

%%%%%%%%%%%%1.13%%%%%%%%%%%%%%%%%%%%%%%%%%%%%%%%%%%%%%%%%%%%%%%%%%%%%
\begin{thm}
Let $a$, $b$, $c$, $d$, and $n$ be integers with $n > 0$. If $a
\equiv b \pmod{n}$ and~${c \equiv d \pmod{n}}$, then $a - c \equiv b
- d \pmod{n}$.
\end{thm}

\begin{proof}[Proof]
Let $a$, $b$, $c$, $d$, and $n$ be integers with $n > 0$, $a
\equiv b \pmod{n}$, and ${c \equiv d \pmod{n}}$. We will show that $a - c \equiv b - d \pmod{n}$.  By definition of congruence, we have $n|(a - b)$ and $n|(c - d)$.  By definition of divides, this gives us $a - b = nj$ and $c - d = nk$ for some integers $j$ and $k$.  So $(a - c) - (b - d) = -(a - b) - (c - d) = -nj - nk = n(-j - k)$.  Since $j$ and $k$ are integers, $-j - k$ is also an integer.  Therefore $n|(a - c) - (b - d)$, which means that $a - c \equiv b - d \pmod{n}$.
\end{proof}

%%%%%%%%%%%%1.14%%%%%%%%%%%%%%%%%%%%%%%%%%%%%%%%%%%%%%%%%%%%%%%%%%%%%
\begin{thm}
Let $a$, $b$, $c$, $d$, and $n$ be integers
with $n > 0$. If $a \equiv b \pmod{n}$ and~${c \equiv d \pmod{n}}$, then
$ac \equiv bd \pmod{n}$.
\end{thm}

\begin{proof}[Proof]
Let $a$, $b$, $c$, $d$, and $n$ be integers with $n > 0$, $a
\equiv b \pmod{n}$, and ${c \equiv d \pmod{n}}$. We will show that $ac \equiv bd \pmod{n}$.  By definition of congruence, we have $n|(a - b)$ and $n|(c - d)$.  By definition of divides, this gives us $a - b = nj$ and $c - d = nk$ for some integers $j$ and $k$. So, $ac - bd = n(dj + bk + jkn)$ Since $b$, $d$, $j$, $k$, and $n$ are all integers, $dj + bk + jkn$ is also an integer.  Therefore $n|ac - bd$, which means that $ac \equiv bd \pmod{n}$.\end{proof}


\end{document}
