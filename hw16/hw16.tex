\documentclass[12pt,leqno]{article}

\usepackage{amsmath,amsfonts,amssymb,amscd,amsthm,amsbsy,upref}


\textheight=8.5truein
\textwidth=6.0truein
\hoffset=-.5truein
\voffset=-.5truein
\numberwithin{equation}{section}
\pagestyle{headings}
\footskip=36pt


\swapnumbers
\newtheorem{thm}{Theorem}[section]
\newtheorem{hthm}[thm]{*Theorem}
\newtheorem{lem}[thm]{Lemma}
\newtheorem{cor}[thm]{Corollary}
\newtheorem{prop}[thm]{Proposition}
\newtheorem{con}[thm]{Conjecture}
\newtheorem{exer}[thm]{Exercise}
\newtheorem{bpe}[thm]{Blank Paper Exercise}
\newtheorem{apex}[thm]{Applications Exercise}
\newtheorem{ques}[thm]{Question}
\newtheorem{scho}[thm]{Scholium}
\newtheorem*{Exthm}{Example Theorem}
\newtheorem*{Thm}{Theorem}
\newtheorem*{Con}{Conjecture}
\newtheorem*{Axiom}{Axiom}



\theoremstyle{definition}
\newtheorem*{Ex}{Example}
\newtheorem*{Def}{Definition}


\newcommand{\lcm}{\operatorname{lcm}}
\newcommand{\ord}{\operatorname{ord}}
\def\pfrac#1#2{{\left(\frac{#1}{#2}\right)}}


\makeindex

\begin{document}




\thispagestyle{plain}
\begin{flushright}
\large{\textbf{TYPE YOUR NAME HERE \\
HW 16: 3.13-3.17\\
M328K \\
March 22th, 2012 \\}}
\end{flushright}

%\maketitle
\markboth{}{} \setcounter{section}{0} \baselineskip=18pt

\setcounter{tocdepth}{4}


%%%%%%%%This is where you can change the numbering to match the problem you
%%%%%%%%are on.  Set the section to  the current chapter.

\setcounter{section}{3}

%%%%%%%%%Now, set the theorem number to one less than the first theorem in
%%%%%%%%%the assignment.
\setcounter{thm}{12}

%%%Theorem 3.13%%%%%%%%%%%%%%%%%%%%%%%%%%%%%%%%%%%%%%%%%%%%%%%%%%%%%%%%%%%%%%%%%%%%%
\begin{thm}
Suppose $f(x) = a_nx^n + a_{n-1}x^{n-1} + \hdots + a_0$ is a
polynomial of degree $n > 0$ with integer coefficients.  Then $f(x)$
is a composite number for infinitely many integers~$x$.
\end{thm}

\begin{proof}[Proof]
Type your proof here!
\end{proof}

%%%Theorem 3.14%%%%%%%%%%%%%%%%%%%%%%%%%%%%%%%%%%%%%%%%%%%%%%%%%%%%%%%%%%%%%%%%%%%%%
\begin{thm}Given any integer $a$ and any natural number $n$, there exists a unique integer~$t$ in the set $\{0, 1, 2, \hdots, n - 1\}$ such that $a \equiv t \pmod{n}$.
\end{thm}

\begin{proof}[Proof]
Type your proof here!
\end{proof}



%%%Exercise 3.15%%%%%%%%%%%%%%%%%%%%%%%%%%%%%%%%%%%%%%%%%%%%%%%%%%%%%%%%%%%%%%%%%%%%%
\begin{exer}
Find three complete residue systems modulo $4$: the canonical
complete residue system, one containing negative numbers, and one
containing no two consecutive numbers.
\end{exer}

\begin{proof}[Solution]
Type your solution here!
\end{proof}


%%%Theorem 3.16%%%%%%%%%%%%%%%%%%%%%%%%%%%%%%%%%%%%%%%%%%%%%%%%%%%%%%%%%%%%%%%%%%%%%
\begin{thm}
Let $n$ be a natural number. Every complete residue system modulo
$n$ contains $n$ elements.
\end{thm}

\begin{proof}[Proof]
Type your proof here!
\end{proof}


%%%Theorem 3.17%%%%%%%%%%%%%%%%%%%%%%%%%%%%%%%%%%%%%%%%%%%%%%%%%%%%%%%%%%%%%%%%%%%%%
\begin{thm}
Let $n$ be a natural number. Any set of $n$ integers $\{a_1, a_2,
\hdots, a_n\}$ for which no two are congruent modulo $n$ is a
complete residue system modulo $n$.
\end{thm}

\begin{proof}[Proof]
Type your proof here!
\end{proof}
\end{document}
