\documentclass[12pt,leqno]{article}

\usepackage{amsmath,amsfonts,amssymb,amscd,amsthm,amsbsy,upref}


\textheight=8.5truein
\textwidth=6.0truein
\hoffset=-.5truein
\voffset=-.5truein
\numberwithin{equation}{section}
\pagestyle{headings}
\footskip=36pt


\swapnumbers
\newtheorem{thm}{Theorem}[section]
\newtheorem{hthm}[thm]{*Theorem}
\newtheorem{lem}[thm]{Lemma}
\newtheorem{cor}[thm]{Corollary}
\newtheorem{prop}[thm]{Proposition}
\newtheorem{con}[thm]{Conjecture}
\newtheorem{exer}[thm]{Exercise}
\newtheorem{bpe}[thm]{Blank Paper Exercise}
\newtheorem{apex}[thm]{Applications Exercise}
\newtheorem{ques}[thm]{Question}
\newtheorem{scho}[thm]{Scholium}
\newtheorem*{Exthm}{Example Theorem}
\newtheorem*{Thm}{Theorem}
\newtheorem*{Con}{Conjecture}
\newtheorem*{Axiom}{Axiom}



\theoremstyle{definition}
\newtheorem*{Ex}{Example}
\newtheorem*{Def}{Definition}


\newcommand{\lcm}{\operatorname{lcm}}
\newcommand{\ord}{\operatorname{ord}}
\def\pfrac#1#2{{\left(\frac{#1}{#2}\right)}}

\newcommand{\card}[1]{\left| #1 \right|}


\makeindex

\begin{document}




\thispagestyle{plain}
\begin{flushright}
\large{\textbf{Geoffrey Parker - grp352 \\
HW 16: 3.13-3.17\\
M328K \\
March 22th, 2012 \\}}
\end{flushright}

%\maketitle
\markboth{}{} \setcounter{section}{0} \baselineskip=18pt

\setcounter{tocdepth}{4}


%%%%%%%%This is where you can change the numbering to match the problem you
%%%%%%%%are on.  Set the section to  the current chapter.

\setcounter{section}{3}

%%%%%%%%%Now, set the theorem number to one less than the first theorem in
%%%%%%%%%the assignment.
\setcounter{thm}{12}

%%%Theorem 3.13%%%%%%%%%%%%%%%%%%%%%%%%%%%%%%%%%%%%%%%%%%%%%%%%%%%%%%%%%%%%%%%%%%%%%
\begin{thm}
Suppose $f(x) = a_nx^n + a_{n-1}x^{n-1} + \hdots + a_0$ is a
polynomial of degree $n > 0$ with integer coefficients.  Then $f(x)$
is a composite number for infinitely many integers~$x$.
\end{thm}

\begin{proof}[Proof]
Suppose $f(x) = a_nx^n + a_{n-1}x^{n-1} + \hdots + a_0$ is a polynomial of degree $n > 0$ with integer coefficients and $a_n > 0$.  We will show that $f(x)$ is a composite number for infinitely many integers $x$.  Let $S$ be the set of integers $x$ such that $f(x)$ is composite and let $T$ be the set of numbers $f(x)$ such that $x$ is an element of $S$.  Assume by way of contradiction that $S$ is finite.  Let $m$ be an element of $S$ such that $f(m)$ is the largest element of $T$.  By theorem 3.12, there exists an integer $k$ such that for all integers $j > k$, $f(j) > f(m)$.  Now consider these two cases: $a_0 = 0$ and $a_0 \neq 0$.\\
In the case that $a_0 \neq 0$, let $y = (k+1)\card{a_0}$.  Since $a_0 \neq 0$, $y$ must be greater than $k$.  Therefore $f(y) > f(m)$.  Also, $a_0 \mid y$, so $y \equiv 0 \pmod{a_0}$.  By theorem 3.8, this means that $f(y) \equiv f(0) \pmod{a_0}$, or equivalently, $a_0 \mid f(y) - a_0$.  Then by theorem 1.1, $a_0 \mid f(y) - a_0 + a_0$, so $a_0 \mid f(y)$.  This means that $f(y)$ is composite.\\
In the other case, $a_0 = 0$, let $p$ be a natural number and let $y = kp$.  This implies that $y > m$ and that $y \equiv 0 \pmod{p}$.  By theorem 3.8, $f(y) \equiv f(0) \pmod{p}$, and since $a_0 = 0$, $f(y) \equiv 0 \pmod{p}$.  Now we have that $p \mid f(y)$, so $y$ is composite.\\
In either case, we have found an integer $y > m$ such that $f(y)$ is composite.  By theorem 3.12, $f(y) > f(m)$.  However, because $f(y)$ is composite, it must be an element of $T$ and $f(m)$ is the largest element of $T$.  Therefore we have contradicted our assumption and $S$ must be infinite.
\end{proof}

\pagebreak
%%%Theorem 3.14%%%%%%%%%%%%%%%%%%%%%%%%%%%%%%%%%%%%%%%%%%%%%%%%%%%%%%%%%%%%%%%%%%%%%
\begin{thm}Given any integer $a$ and any natural number $n$, there exists a unique integer~$t$ in the set $\{0, 1, 2, \hdots, n - 1\}$ such that $a \equiv t \pmod{n}$.
\end{thm}

\begin{proof}[Proof]
Let $a$ be an integer and $n$ be a natural number.  Let $S$ be the set $\{0, 1, 2, \hdots, n - 1\}$.  By the division algorithm there exist integers $q$ and $r$ such that $a = nq + r$ with $0 \leq r \leq n-1$.  So $nq = a - r$, which by the definition of dividies means that $n \mid a - r$.  By the definition of congruence mod n, $a \equiv r \pmod{n}$.  Since $0 \leq r \leq n-1$, if we let $t = r$, we have shown that there exists an integer $t$ element of $S$ such that $a \equiv t \pmod{n}$.
\end{proof}



%%%Exercise 3.15%%%%%%%%%%%%%%%%%%%%%%%%%%%%%%%%%%%%%%%%%%%%%%%%%%%%%%%%%%%%%%%%%%%%%
\begin{exer}
Find three complete residue systems modulo $4$: the canonical
complete residue system, one containing negative numbers, and one
containing no two consecutive numbers.
\end{exer}

\begin{proof}[Solution]
canonical complete: $\{0, 1, 2, 3\}$\\
negative: $\{-1,-2,-3,-4\}$\\
non-consecutive: $\{0, 2, 5, 7\}$
\end{proof}


\pagebreak
%%%Theorem 3.16%%%%%%%%%%%%%%%%%%%%%%%%%%%%%%%%%%%%%%%%%%%%%%%%%%%%%%%%%%%%%%%%%%%%%
\begin{thm}
Let $n$ be a natural number. Every complete residue system modulo
$n$ contains $n$ elements.
\end{thm}

\begin{proof}[Proof]
Let $n$ be a natural number.  Let $S$ be any complete residue system modulo $n$.  We will show that $S$ contains $n$ elements.  Consider $T$, the set of integers $\{0, 1, \ldots , n-1\}$.  Note that $T$ is the canonical complete residue system modulo $n$ and $\card{T} = n$.  Since each element of $T$ is congruent to itself modulo $n$ and by definition of complete residue systems modulo $n$ every integer is congruent modulo $n$ to exactly one element of $T$, no element of $T$ is congruent to another distinct element of $T$ modulo $n$.  Let $a$ and $b$ be any two distinct elements of $T$.  We know by definition of complete residue systems modulo $n$ again that $a$ and $b$ are congruent modulo $n$ to exactly one element of $S$ each.  We will call these elements of $S$ $a'$ and $b'$ respectively.  We know that $a' \neq b'$ because if they were equal, then we would have $a \equiv a' \pmod{n}$ and $a' \equiv b \pmod{n}$ implying by theorem 1.11 that $a \equiv b \pmod{n}$, which we know is not true.  Therefore every element of $T$ coorespondes to a distinct element of $S$, meaning that $\card{S} \geq \card{T}$.  Assume by way of contradiction that $\card{s} > n$.  We know, since $T$ is the canonical complete residue system modulo $n$, that every element of $S$ is congruent modulo $n$ to exactly one element of $T$.  Because $\card{S} > \card{T}$, the pigeon hole principle implies that there must be at least two elements of $S$, call them $x$ and $y$ that are congruent modulo $n$ to the same element of $T$, which we will call $z$. So $x \equiv z \pmod{n}$ and $y \equiv z \pmod{n}$, implying by theorem 1.11 that $x \equiv y \pmod{n}$.  However, since all integers are congruent mod $n$ to themselves, we now have that $y$ is congruent mod $n$ to two elements of $S$, contradicting it's definition as a canonical complete residue system modulo $n$.  Therefore $S$ contains exactly $n$ elements.
\end{proof}


\pagebreak
%%%Theorem 3.17%%%%%%%%%%%%%%%%%%%%%%%%%%%%%%%%%%%%%%%%%%%%%%%%%%%%%%%%%%%%%%%%%%%%%
\begin{thm}
Let $n$ be a natural number. Any set of $n$ integers $\{a_1, a_2,
\hdots, a_n\}$ for which no two are congruent modulo $n$ is a
complete residue system modulo $n$.
\end{thm}

\begin{proof}[Proof]
Let $n$ be a natural number. Let $S$ be a set of $n$ integers $\{a_1, a_2, \hdots, a_n\}$ for which no two are congruent modulo $n$.  Let $T$ be the canonical complete residue system modulo $n$.  Let $x$ be an arbitrary integer.  By definition of $T$, there must be some $y$ which is an element of $T$ such that $x \equiv y \pmod{n}$.  Let $z$ be an element of $S$ such that $z \equiv y \pmod{n}$.  We know that $z$ exists because $\card{S} = \card{T}$ and every element of $S$ is congruent modulo $n$ to a different element of $T$. If there were two elements of $S$, $j$ and $k$, that were congruent modulo $n$ to the same element of $T$. then by theorem 1.11 we would have $j \equiv k \pmod{n}$, which contradicts the definition of $S$. Now by theorem 1.11 again, $x \equiv z \pmod{n}$.  Let $z'$ some element of $S$ such that $x \equiv z' \pmod{n}$.  By theorem 1.11, this means that $z \equiv z' \pmod{n}$, which contradicts the definition of $S$ unless $z = z'$.  Therefore, since $x$ is arbitrary, every integer is congruent modulo $n$ to exactly one element of $S$, which is the definition of a complete residue system modulo $n$.
\end{proof}
\end{document}
