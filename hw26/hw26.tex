\documentclass[12pt,leqno]{article}

\usepackage{amsmath,amsfonts,amssymb,amscd,amsthm,amsbsy,upref}


\textheight=8.5truein
\textwidth=6.0truein
\hoffset=-.5truein
\voffset=-.5truein
\numberwithin{equation}{section}
\pagestyle{headings}
\footskip=36pt


\swapnumbers
\newtheorem{thm}{Theorem}[section]
\newtheorem{hthm}[thm]{*Theorem}
\newtheorem{lem}[thm]{Lemma}
\newtheorem{cor}[thm]{Corollary}
\newtheorem{prop}[thm]{Proposition}
\newtheorem{con}[thm]{Conjecture}
\newtheorem{exer}[thm]{Exercise}
\newtheorem{bpe}[thm]{Blank Paper Exercise}
\newtheorem{apex}[thm]{Applications Exercise}
\newtheorem{ques}[thm]{Question}
\newtheorem{scho}[thm]{Scholium}
\newtheorem*{Exthm}{Example Theorem}
\newtheorem*{Thm}{Theorem}
\newtheorem*{Con}{Conjecture}
\newtheorem*{Axiom}{Axiom}



\theoremstyle{definition}
\newtheorem*{Ex}{Example}
\newtheorem*{Def}{Definition}


\newcommand{\lcm}{\operatorname{lcm}}
\newcommand{\ord}{\operatorname{ord}}
\def\pfrac#1#2{{\left(\frac{#1}{#2}\right)}}


\makeindex

\begin{document}




\thispagestyle{plain}
\begin{flushright}
\large{\textbf{TYPE YOUR NAME HERE \\
HW 26: 5.1 - 5.5\\
M328K \\
May 1st, 2012 \\}}
\end{flushright}

%\maketitle
\markboth{}{} \setcounter{section}{0} \baselineskip=18pt

\setcounter{tocdepth}{4}


%%%%%%%%This is where you can change the numbering to match the problem you
%%%%%%%%are on.  Set the section to  the current chapter.

\setcounter{section}{5}

%%%%%%%%%Now, set the theorem number to one less than the first theorem in
%%%%%%%%%the assignment.
\setcounter{thm}{0}

%%%Theorem 5.1%%%%%%%%%%%%%%%%%%%%%%%%%%%%%%%%%%%%%%%%%%%%%%%%%%%%%%%%%%%%%%%%%%%%%

\begin{thm}
If $p$ and $q$ are distinct prime numbers and $W$ is a natural
number with $(W, pq)=1$, then $W^{(p-1)(q-1)}\equiv 1 \pmod{pq}$.
\end{thm}
\begin{proof}[Proof]
\end{proof}

%%%Theorem 5.2%%%%%%%%%%%%%%%%%%%%%%%%%%%%%%%%%%%%%%%%%%%%%%%%%%%%%%%%%%%%%%%%%%%%%

\begin{thm}
Let $p$ and $q$ be distinct primes, $k$ be a natural number, and $W$
be a natural number less than $pq$. Then
\[W^{1+k(p-1)(q-1)}\equiv W\pmod{pq}.\]
\end{thm}
\begin{proof}[Proof]
\end{proof}

%%%Theorem 5.3%%%%%%%%%%%%%%%%%%%%%%%%%%%%%%%%%%%%%%%%%%%%%%%%%%%%%%%%%%%%%%%%%%%%%
\begin{thm}
Let $p$ and $q$ be distinct primes and $E$ be a natural number
relatively prime to $(p-1)(q-1)$. Then there exist natural numbers
$D$ and $y$ such that
\[ED = 1+y(p-1)(q-1).\]
\end{thm}
\begin{proof}[Proof]
\end{proof}

%%%Theorem 5.4%%%%%%%%%%%%%%%%%%%%%%%%%%%%%%%%%%%%%%%%%%%%%%%%%%%%%%%%%%%%%%%%%%%%%
\begin{thm}
Let $p$ and $q$ be distinct primes, $W$ be a natural number less
than $pq$, and $E$, $D$, and $y$ be natural numbers such that
$ED=1+y(p-1)(q-1)$. Then
\[W^{ED}\equiv W\pmod{pq}.\]
\end{thm}
\begin{proof}[Proof]
\end{proof}

%%%Exercise 5.5%%%%%%%%%%%%%%%%%%%%%%%%%%%%%%%%%%%%%%%%%%%%%%%%%%%%%%%%%%%%%%%%%%%%%
\begin{exer}
Consider two distinct primes $p$ and $q$. Describe every step of the
RSA Public Key Coding System. State what numbers you choose to make
public, what messages can be encoded, how messages should be
encoded, and how messages are decoded.  What number should be called
the \textit{encoding exponent} and what number should be called the
\textit{decoding exponent}?
\end{exer}
\begin{proof}[Solution]
\end{proof}
\end{document}
