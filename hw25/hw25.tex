\documentclass[12pt,leqno]{article}

\usepackage{amsmath,amsfonts,amssymb,amscd,amsthm,amsbsy,upref}


\textheight=8.5truein
\textwidth=6.0truein
\hoffset=-.5truein
\voffset=-.5truein
\numberwithin{equation}{section}
\pagestyle{headings}
\footskip=36pt


\swapnumbers
\newtheorem{thm}{Theorem}[section]
\newtheorem{hthm}[thm]{*Theorem}
\newtheorem{lem}[thm]{Lemma}
\newtheorem{cor}[thm]{Corollary}
\newtheorem{prop}[thm]{Proposition}
\newtheorem{con}[thm]{Conjecture}
\newtheorem{exer}[thm]{Exercise}
\newtheorem{bpe}[thm]{Blank Paper Exercise}
\newtheorem{apex}[thm]{Applications Exercise}
\newtheorem{ques}[thm]{Question}
\newtheorem{scho}[thm]{Scholium}
\newtheorem*{Exthm}{Example Theorem}
\newtheorem*{Thm}{Theorem}
\newtheorem*{Con}{Conjecture}
\newtheorem*{Axiom}{Axiom}



\theoremstyle{definition}
\newtheorem*{Ex}{Example}
\newtheorem*{Def}{Definition}


\newcommand{\lcm}{\operatorname{lcm}}
\newcommand{\ord}{\operatorname{ord}}
\def\pfrac#1#2{{\left(\frac{#1}{#2}\right)}}
\newcommand{\card}[1]{\left| #1 \right|}



\makeindex

\begin{document}




\thispagestyle{plain}
\begin{flushright}
\large{\textbf{Geoffrey Parker - grp352 \\
HW 25: 4.36 - 4.42\\
M328K \\
April 26th, 2012 \\}}
\end{flushright}

%\maketitle
\markboth{}{} \setcounter{section}{0} \baselineskip=18pt

\setcounter{tocdepth}{4}


%%%%%%%%This is where you can change the numbering to match the problem you
%%%%%%%%are on.  Set the section to  the current chapter.

\setcounter{section}{4}

%%%%%%%%%Now, set the theorem number to one less than the first theorem in
%%%%%%%%%the assignment.
\setcounter{thm}{35}

%%%Theorem 4.36%%%%%%%%%%%%%%%%%%%%%%%%%%%%%%%%%%%%%%%%%%%%%%%%%%%%%%%%%%%%%%%%%%%%%
\begin{thm}
Let $p$ be a prime and let $a$ be an integer such that $1 \leq a <
p$.  Then there exists a unique natural number $b$ less than $p$
such that $ab \equiv 1 \pmod{p}$.
\end{thm}
\begin{proof}[Proof]
Let $p$ be a prime and let $a$ be an integer such that $1 \leq a < p$.  Then by theorem 4.13 $S = \{a, 2a, \dots, pa\}$ is a complete residue system modulo $p$.  So by definition of complete residue systems, one, being an integer, is congruent modulo $p$ to exactly one element of $S$, call it $t$.  So $t = ab$ where $b \leq p$ is a natural number.  However $ap \equiv 0 \pmod{p}$ so $b$ cannot be $p$.  Therefore there exists a unique natural number $b$ less than $p$ such that $ab \equiv 1 \pmod{p}$.
\end{proof}

%%%Exercise 4.37%%%%%%%%%%%%%%%%%%%%%%%%%%%%%%%%%%%%%%%%%%%%%%%%%%%%%%%%%%%%%%%%%%%%%
\begin{exer}
Let $p$ be a prime. Show that the natural numbers $1$ and $p-1$ are
their own inverses modulo $p$.
\end{exer}
\begin{proof}[Solution]$ $\\
$1 \cdot 1 = 1$ and $1 \equiv 1 \pmod{p}$.\\
$(p-1)(p-1) = p^2 - 2p + 1$.  And since $p^2 \equiv 0 \pmod{p}$ and $2p \equiv 0 \pmod{p}$, we know $p^2 - 2p + 1 \equiv 1 \pmod{p}$.
\end{proof}

%%%Theorem 4.38%%%%%%%%%%%%%%%%%%%%%%%%%%%%%%%%%%%%%%%%%%%%%%%%%%%%%%%%%%%%%%%%%%%%%
\begin{thm}
Let $p$ be a prime and let $a$ and $b$ be integers such that $1 < a, b < p-1$ and~\mbox{$ab \equiv 1 \pmod{p}$}. Then $a \neq b$.
\end{thm}
\begin{proof}[Proof]
Let $p$ be a prime and let $a$ and $b$ be integers such that $1 < a, b < p-1$ and~\mbox{$ab \equiv 1 \pmod{p}$}. Assume by way of contradiction that $a = b$.  Then $aa \equiv 1 \pmod{p}$ and $p \mid aa - 1$ or equivalently $p \mid (a + 1)(a - 1)$.  So by theorem 2.27 $p \mid a + 1$ or $p \mid a - 1$.  However, since $1 < a < p-1$ both of these are natural numbers less than $p$, so $p$ cannot divide either.  Therefore we have a contradiction and have shown that $a \neq b$.
\end{proof}

%%Exercise 4.39%%%%%%%%%%%%%%%%%%%%%%%%%%%%%%%%%%%%%%%%%%%%%%%%%%%%%%%%%%%%%%%%%%%%%
\begin{exer}
 Find all pairs of numbers $a$ and $b$ in
$\{2, 3, \dots, 11\}$ such that $ab \equiv 1 \pmod{13}$.
\end{exer}
\begin{proof}[Solution]
2, 7; 3, 9; 4, 10; 5, 8; 6, 11
\end{proof}

%%%Theorem 4.40%%%%%%%%%%%%%%%%%%%%%%%%%%%%%%%%%%%%%%%%%%%%%%%%%%%%%%%%%%%%%%%%%%%%%
\begin{thm}
If $p$ is a prime larger than $2$, then $2 \cdot 3 \cdot 4 \cdot
\hdots \cdot (p-2) \equiv 1 \pmod{p}$.
\end{thm}
\begin{proof}[Proof]
Let $p$ be a prime larger than 2.  Let $S$ be the set of numbers $\{2, 3, 4, \dots , (p-2)\}$.  Note that each element of $S$ is coprime with $p$.  By theorem 4.36 each element $a$ of $S$ has some natural number $b < p$ such that $ab \equiv 1 \pmod{p}$.  But $b$ cannot be 1 because that would imply that $p \mid a - 1$.  And if $b = p-1$ then $p \mid ap - a$, and since $p\mid ap$ then by theorem 1.1 $p \mid a$.  Since $1 < a < p-1$, neither of these can be true, so $b$ must be an element of the set $S$.  And by theorem 4.38 $a \neq b$.  So we can break the set $S$ into $n$ distinct $a$, $b$ pairs where $ab \equiv 1 \pmod{p}$ and $\card{S} = 2n$.  Then $2 \cdot 3 \cdot 4 \cdot \hdots \cdot (p-2) \equiv 1 \pmod{p}$ can be rewritten as $a_1b_1a_2b_2\dots a_nb_n$.  Since each of these pairs is congruent modulo $p$ to one, the entire product is congruent modulo $p$ to one.  Therefore $2 \cdot 3 \cdot 4 \cdot \hdots \cdot (p-2) \equiv 1 \pmod{p}$.
\end{proof}

%%%Theorem 4.41%%%%%%%%%%%%%%%%%%%%%%%%%%%%%%%%%%%%%%%%%%%%%%%%%%%%%%%%%%%%%%%%%%%%%

\begin{thm}[Wilson's Theorem]
If $p$ is a prime, then $(p - 1)! \equiv -1 \pmod{p}$.
\end{thm}
\begin{proof}[Proof]
\end{proof}

%%%Theorem 4.42%%%%%%%%%%%%%%%%%%%%%%%%%%%%%%%%%%%%%%%%%%%%%%%%%%%%%%%%%%%%%%%%%%%%%
\begin{thm}[Converse of Wilson's Theorem]
If $n$ is a natural number such that $(n - 1)! \equiv -1 \pmod{n}$,
then $n$ is prime.
\end{thm}
\begin{proof}[Proof]
\end{proof}
\end{document}
