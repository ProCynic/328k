\documentclass[12pt,leqno]{article}

\usepackage{amsmath,amsfonts,amssymb,amscd,amsthm,amsbsy,upref}


\textheight=8.5truein
\textwidth=6.0truein
\hoffset=-.5truein
\voffset=-.5truein
\numberwithin{equation}{section}
\pagestyle{headings}
\footskip=36pt


\swapnumbers
\newtheorem{thm}{Theorem}[section]
\newtheorem{hthm}[thm]{*Theorem}
\newtheorem{lem}[thm]{Lemma}
\newtheorem{cor}[thm]{Corollary}
\newtheorem{prop}[thm]{Proposition}
\newtheorem{con}[thm]{Conjecture}
\newtheorem{exer}[thm]{Exercise}
\newtheorem{bpe}[thm]{Blank Paper Exercise}
\newtheorem{apex}[thm]{Applications Exercise}
\newtheorem{ques}[thm]{Question}
\newtheorem{scho}[thm]{Scholium}
\newtheorem*{Exthm}{Example Theorem}
\newtheorem*{Thm}{Theorem}
\newtheorem*{Con}{Conjecture}
\newtheorem*{Axiom}{Axiom}



\theoremstyle{definition}
\newtheorem*{Ex}{Example}
\newtheorem*{Def}{Definition}


\newcommand{\lcm}{\operatorname{lcm}}
\newcommand{\ord}{\operatorname{ord}}
\def\pfrac#1#2{{\left(\frac{#1}{#2}\right)}}


\makeindex

\begin{document}




\thispagestyle{plain}
\begin{flushright}
\large{\textbf{TYPE YOUR NAME HERE \\
HW 25: 4.36 - 4.42\\
M328K \\
April 26th, 2012 \\}}
\end{flushright}

%\maketitle
\markboth{}{} \setcounter{section}{0} \baselineskip=18pt

\setcounter{tocdepth}{4}


%%%%%%%%This is where you can change the numbering to match the problem you
%%%%%%%%are on.  Set the section to  the current chapter.

\setcounter{section}{4}

%%%%%%%%%Now, set the theorem number to one less than the first theorem in
%%%%%%%%%the assignment.
\setcounter{thm}{35}

%%%Theorem 4.36%%%%%%%%%%%%%%%%%%%%%%%%%%%%%%%%%%%%%%%%%%%%%%%%%%%%%%%%%%%%%%%%%%%%%
\begin{thm}
Let $p$ be a prime and let $a$ be an integer such that $1 \leq a <
p$.  Then there exists a unique natural number $b$ less than $p$
such that $ab \equiv 1 \pmod{p}$.
\end{thm}
\begin{proof}[Proof]
\end{proof}

%%%Exercise 4.37%%%%%%%%%%%%%%%%%%%%%%%%%%%%%%%%%%%%%%%%%%%%%%%%%%%%%%%%%%%%%%%%%%%%%
\begin{exer}
Let $p$ be a prime. Show that the natural numbers $1$ and $p-1$ are
their own inverses modulo $p$.
\end{exer}
\begin{proof}[Solution]
\end{proof}

%%%Theorem 4.38%%%%%%%%%%%%%%%%%%%%%%%%%%%%%%%%%%%%%%%%%%%%%%%%%%%%%%%%%%%%%%%%%%%%%
\begin{thm}
Let $p$ be a prime and let $a$ and $b$ be integers such that $1 <
a,b < p-1$ and~\mbox{$ab \equiv 1 \pmod{p}$}. Then $a \neq b$.
\end{thm}
\begin{proof}[Proof]
\end{proof}

%%Exercise 4.39%%%%%%%%%%%%%%%%%%%%%%%%%%%%%%%%%%%%%%%%%%%%%%%%%%%%%%%%%%%%%%%%%%%%%
\begin{exer}
 Find all pairs of numbers $a$ and $b$ in
$\{2, 3, \dots, 11\}$ such that $ab \equiv 1 \pmod{13}$.
\end{exer}
\begin{proof}[Solution]
\end{proof}

%%%Theorem 4.40%%%%%%%%%%%%%%%%%%%%%%%%%%%%%%%%%%%%%%%%%%%%%%%%%%%%%%%%%%%%%%%%%%%%%
\begin{thm}
If $p$ is a prime larger than $2$, then $2 \cdot 3 \cdot 4 \cdot
\hdots \cdot (p-2) \equiv 1 \pmod{p}$.
\end{thm}
\begin{proof}[Proof]
\end{proof}

%%%Theorem 4.41%%%%%%%%%%%%%%%%%%%%%%%%%%%%%%%%%%%%%%%%%%%%%%%%%%%%%%%%%%%%%%%%%%%%%

\begin{thm}[Wilson's Theorem]
If $p$ is a prime, then $(p - 1)! \equiv -1 \pmod{p}$.
\end{thm}
\begin{proof}[Proof]
\end{proof}

%%%Theorem 4.42%%%%%%%%%%%%%%%%%%%%%%%%%%%%%%%%%%%%%%%%%%%%%%%%%%%%%%%%%%%%%%%%%%%%%
\begin{thm}[Converse of Wilson's Theorem]
If $n$ is a natural number such that $(n - 1)! \equiv -1 \pmod{n}$,
then $n$ is prime.
\end{thm}
\begin{proof}[Proof]
\end{proof}
\end{document}
