\documentclass[12pt,leqno]{article}

\usepackage{amsmath,amsfonts,amssymb,amscd,amsthm,amsbsy,upref}


\textheight=8.5truein
\textwidth=6.0truein
\hoffset=-.5truein
\voffset=-.5truein
\numberwithin{equation}{section}
\pagestyle{headings}
\footskip=36pt


\swapnumbers
\newtheorem{thm}{Theorem}[section]
\newtheorem{hthm}[thm]{*Theorem}
\newtheorem{lem}[thm]{Lemma}
\newtheorem{cor}[thm]{Corollary}
\newtheorem{prop}[thm]{Proposition}
\newtheorem{con}[thm]{Conjecture}
\newtheorem{exer}[thm]{Exercise}
\newtheorem{bpe}[thm]{Blank Paper Exercise}
\newtheorem{apex}[thm]{Applications Exercise}
\newtheorem{ques}[thm]{Question}
\newtheorem{scho}[thm]{Scholium}
\newtheorem*{Exthm}{Example Theorem}
\newtheorem*{Thm}{Theorem}
\newtheorem*{Con}{Conjecture}
\newtheorem*{Axiom}{Axiom}



\theoremstyle{definition}
\newtheorem*{Ex}{Example}
\newtheorem*{Def}{Definition}


\newcommand{\lcm}{\operatorname{lcm}}
\newcommand{\ord}{\operatorname{ord}}
\def\pfrac#1#2{{\left(\frac{#1}{#2}\right)}}


\makeindex

\begin{document}




\thispagestyle{plain}
\begin{flushright}
\large{\textbf{Geoffrey Parker - grp352 \\
HW 23: 4.27, 4.31 - 4.35\\
M328K \\
April 24th, 2012 \\}}
\end{flushright}

%\maketitle
\markboth{}{} \setcounter{section}{0} \baselineskip=18pt

\setcounter{tocdepth}{4}


%%%%%%%%This is where you can change the numbering to match the problem you
%%%%%%%%are on.  Set the section to  the current chapter.

\setcounter{section}{4}

%%%%%%%%%Now, set the theorem number to one less than the first theorem in
%%%%%%%%%the assignment.
\setcounter{thm}{26}

%%%Question 4.27%%%%%%%%%%%%%%%%%%%%%%%%%%%%%%%%%%%%%%%%%%%%%%%%%%%%%%%%%%%%%%%%%%%%%
\begin{ques}
The numbers $1$, $5$, $7$, and $11$ are all the natural numbers less
than or equal to $12$ that are relatively prime to $12$, so
$\phi(12) = 4$.
\begin{enumerate}
\item What is $\phi(7)$?
\item What is $\phi(15)$?
\item What is $\phi(21)$?
\item What is $\phi(35)$?
\end{enumerate}
\end{ques}
\begin{proof}[Answer]
\end{proof}

\setcounter{thm}{30}

%%%Theorem 4.31%%%%%%%%%%%%%%%%%%%%%%%%%%%%%%%%%%%%%%%%%%%%%%%%%%%%%%%%%%%%%%%%%%%%%
\begin{thm}
Let $n$ be a natural number and let $x_1$, $x_2$, $\hdots$,
$x_{\phi(n)}$ be the distinct natural numbers less than or equal to
$n$ that are relatively prime to $n$.  Let $a$ be a non-zero integer
relatively prime to $n$ and let $i$ and $j$ be different natural
numbers less than or equal to~$\phi(n)$.  Then~$ax_i \not \equiv
ax_j \pmod{n}$.
\end{thm}
\begin{proof}[Proof]
\end{proof}

%%%Theorem 4.32%%%%%%%%%%%%%%%%%%%%%%%%%%%%%%%%%%%%%%%%%%%%%%%%%%%%%%%%%%%%%%%%%%%%%
\begin{thm}[Euler's Theorem]
If $a$ and $n$ are integers with $n > 0$ and $(a, n) = 1$, then \[
a^{\phi(n)} \equiv 1 \pmod{n}. \]
\end{thm}
\begin{proof}[Proof]
\end{proof}

%%%Theorem 4.33%%%%%%%%%%%%%%%%%%%%%%%%%%%%%%%%%%%%%%%%%%%%%%%%%%%%%%%%%%%%%%%%%%%%%
\begin{cor}[Fermat's Little Theorem]
If $p$ is a prime and $a$ is an integer relatively prime to $p$,
then $a^{(p-1)} \equiv 1 \pmod{p}$.
\end{cor}
\begin{proof}[Proof]
\end{proof}


%%%Exercise 4.34%%%%%%%%%%%%%%%%%%%%%%%%%%%%%%%%%%%%%%%%%%%%%%%%%%%%%%%%%%%%%%%%%%%%%
\begin{exer}
Compute each of the following without the aid of a calculator or
computer.
\begin{enumerate}
\item $12^{49} \pmod{15}$.
\item $139^{112} \pmod{27}$.
\end{enumerate}
\end{exer}
\begin{proof}[Solution]
\end{proof}

%%%Exercise 4.35%%%%%%%%%%%%%%%%%%%%%%%%%%%%%%%%%%%%%%%%%%%%%%%%%%%%%%%%%%%%%%%%%%%%%
\begin{exer}
Find the last digit in the base $10$ representation of the integer
$13^{474}$.
\end{exer}
\begin{proof}[Solution]
\end{proof}
\end{document}
