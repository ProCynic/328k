\documentclass[12pt,leqno]{article}

\usepackage{amsmath,amsfonts,amssymb,amscd,amsthm,amsbsy,upref}


\textheight=8.5truein
\textwidth=6.0truein
\hoffset=-.5truein
\voffset=-.5truein
\numberwithin{equation}{section}
\pagestyle{headings}
\footskip=36pt


\swapnumbers
\newtheorem{thm}{Theorem}[section]
\newtheorem{hthm}[thm]{*Theorem}
\newtheorem{lem}[thm]{Lemma}
\newtheorem{cor}[thm]{Corollary}
\newtheorem{prop}[thm]{Proposition}
\newtheorem{con}[thm]{Conjecture}
\newtheorem{exer}[thm]{Exercise}
\newtheorem{bpe}[thm]{Blank Paper Exercise}
\newtheorem{apex}[thm]{Applications Exercise}
\newtheorem{ques}[thm]{Question}
\newtheorem{scho}[thm]{Scholium}
\newtheorem*{Exthm}{Example Theorem}
\newtheorem*{Thm}{Theorem}
\newtheorem*{Con}{Conjecture}
\newtheorem*{Axiom}{Axiom}



\theoremstyle{definition}
\newtheorem*{Ex}{Example}
\newtheorem*{Def}{Definition}


\newcommand{\lcm}{\operatorname{lcm}}
\newcommand{\ord}{\operatorname{ord}}
\def\pfrac#1#2{{\left(\frac{#1}{#2}\right)}}


\makeindex

\begin{document}

\thispagestyle{plain}
\begin{flushright}
\large{\textbf{Geoffrey Parker - grp352 \\
HW 23: 4.27, 4.31 - 4.35\\
M328K \\
April 24th, 2012 \\}}
\end{flushright}

%\maketitle
\markboth{}{} \setcounter{section}{0} \baselineskip=18pt

\setcounter{tocdepth}{4}


%%%%%%%%This is where you can change the numbering to match the problem you
%%%%%%%%are on.  Set the section to  the current chapter.

\setcounter{section}{4}

%%%%%%%%%Now, set the theorem number to one less than the first theorem in
%%%%%%%%%the assignment.
\setcounter{thm}{26}

%%%Question 4.27%%%%%%%%%%%%%%%%%%%%%%%%%%%%%%%%%%%%%%%%%%%%%%%%%%%%%%%%%%%%%%%%%%%%%
\begin{ques}
The numbers $1$, $5$, $7$, and $11$ are all the natural numbers less
than or equal to $12$ that are relatively prime to $12$, so
$\phi(12) = 4$.
\begin{enumerate}
\item What is $\phi(7)$?
\item What is $\phi(15)$?
\item What is $\phi(21)$?
\item What is $\phi(35)$?
\end{enumerate}
\end{ques}
\begin{proof}[Answer]
$\phi(7) = 6; \phi(15) = 8; \phi(21) = 12; \phi(35) = 24$
\end{proof}

\setcounter{thm}{30}

%%%Theorem 4.31%%%%%%%%%%%%%%%%%%%%%%%%%%%%%%%%%%%%%%%%%%%%%%%%%%%%%%%%%%%%%%%%%%%%%
\begin{thm}
Let $n$ be a natural number and let $x_1$, $x_2$, $\hdots$,
$x_{\phi(n)}$ be the distinct natural numbers less than or equal to
$n$ that are relatively prime to $n$.  Let $a$ be a non-zero integer
relatively prime to $n$ and let $i$ and $j$ be different natural
numbers less than or equal to~$\phi(n)$.  Then~$ax_i \not \equiv
ax_j \pmod{n}$.
\end{thm}
\begin{proof}[Proof]
Let $n$ be a natural number and let $x_1$, $x_2$, $\hdots$, $x_{\phi(n)}$ be the distinct natural numbers less than or equal to $n$ that are relatively prime to $n$.  Let $a$ be a non-zero integer relatively prime to $n$ and let $i$ and $j$ be different natural numbers less than or equal to~$\phi(n)$.  Assume by way of contradiction that $ax_i \equiv ax_j \pmod{n}$.  So by theorem 4.5, because $(a, n) = 1$, $x_i \equiv x_j \pmod{n}$.  However, since $x_i$ and $x_j$ are less than $n$, they are elements of the canonical complete residue system modulo $n$.  So by the definition of complete residue sytstems $x_i \not \equiv x_j \pmod{n}$ and we have a contradiction.  Therefore $ax_i \not \equiv ax_j \pmod{n}$.
\end{proof}

\pagebreak
%%%Theorem 4.32%%%%%%%%%%%%%%%%%%%%%%%%%%%%%%%%%%%%%%%%%%%%%%%%%%%%%%%%%%%%%%%%%%%%%
\begin{thm}[Euler's Theorem]
If $a$ and $n$ are integers with $n > 0$ and $(a, n) = 1$, then \[
a^{\phi(n)} \equiv 1 \pmod{n}. \]
\end{thm}
\begin{proof}[Proof]
Let $a$ and $n$ be integers with $n > 0$ and $(a, n) = 1$.  Let $x_1$, $x_2$, $\hdots$, $x_{\phi(n)}$ be the distinct natural numbers less than or equal to $n$ that are relatively prime to $n$. Let $S = \{ax_1, ax_2, \hdots , ax_{\phi(n)}\}$.  Then each element $ax_i$ of $S$ must be congruent modulo $n$ to some element $y_i$ of the canonical complete residue system modulo $n$.  Because $(n, x_i) = 1$ and $(n, a) = 1$ theorem 2.29 says $(n, ax_i) = 1$.  And by theorem 4.3 $(y_i, n) = 1)$  So the set of $y$'s is just the set of $x$'s with possibly different indicies.\\

Now we will show by induction that $a^{\phi(n)}x_1x_2\cdots x_{\phi(n)} \equiv y_1y_2\cdots y_{\phi(n)} \pmod{n}$ and $(y_1y_2\cdots y_{\phi(n)}, n) = 1$.\\
As a base case, consider $\phi(n) = 1$.  In this case $ax_1 \equiv y_1 \pmod{n}$ and $(x_1, n) = 1$ by definition.\\
Our induction hypothesis is that there exists some $k$ where $1 \leq k < \phi(n)$, \\$a^kx_1x_2\cdots x_k \equiv y_1y_2\cdots y_k$, and $(y_1y_2\cdots y_k, n) = 1$.\\
Now given that $ax_{k+1} \equiv y_{k+1}$ by definition of $y_{k+1}$ theorem 1.14 says that \\$a^{k+1}x_1x_2\cdots x_{k+1} \equiv y_1y_2\cdots y_{k+1}$.  And because $(y_1y_2\cdots y_k, n) = 1$ and $(y_{k+1}, n) = 1$, we know by theorem 2.29 that $(y_1y_2\cdots y_{k+1}, n) = 1$.\\

Now because the set of $x$'s is the same as the set of $y$'s the products of the elements of the sets are the same.  We will call this product $t$, and we have just shown that $a^{\phi(n)}t \equiv t \pmod{n}$ and $(t, n) = 1$.  Therefore by theorem 4.5 $a^{\phi(n)} \equiv 1 \pmod{n}$.
\end{proof}

%%%Theorem 4.33%%%%%%%%%%%%%%%%%%%%%%%%%%%%%%%%%%%%%%%%%%%%%%%%%%%%%%%%%%%%%%%%%%%%%
\begin{cor}[Fermat's Little Theorem]
If $p$ is a prime and $a$ is an integer relatively prime to $p$,
then $a^{(p-1)} \equiv 1 \pmod{p}$.
\end{cor}
\begin{proof}[Proof]
Let $p$ be a prime and $a$ be an integer relatively prime to $p$.  Then by Euler's Theorem $a^{\phi(p)} \equiv 1 \pmod{p}$.  However, since $p$ is prime, every natural number less than $p$ is coprime with $p$, meaning that $\phi(p) = p - 1$.  Therefore $a^{p-1} \equiv 1 \pmod{p}$.
\end{proof}

\pagebreak
%%%Exercise 4.34%%%%%%%%%%%%%%%%%%%%%%%%%%%%%%%%%%%%%%%%%%%%%%%%%%%%%%%%%%%%%%%%%%%%%
\begin{exer}
Compute each of the following without the aid of a calculator or
computer.
\begin{enumerate}
\item $12^{49} \pmod{15}$.
\item $139^{112} \pmod{27}$.
\end{enumerate}
\end{exer}
\begin{proof}[Solution]$ $\\
\begin{enumerate}
\item
First, $12^{49} \equiv 0 \pmod{3}$ and $\phi(5) = 4$. So $12^4 \equiv 1 \pmod{5}$ and $12^{49} \equiv 12^1 \equiv 2 \pmod{5}$.  Then we have $12^{49} \equiv 12 \pmod{3}$ and $12^{49} \equiv 12 \pmod{5}$, so by theorem 4.21 $12^{49} \equiv 12 \pmod{15}$

\item
$27 = 3^3$ and $3 \nmid 139$ so $(27, 139) = 1$.  The natural numbers coprime to 27 are those which 3 does not divide.  So $\phi(27) = 26 - 8 = 18$.  Then Euler's Theorem says $139^{18} \equiv 1 \pmod{n}$, and $112 = 6\cdot 18 + 4$, which gives us:
 \[139^{112} \equiv (139^{18})^6\cdot 139^4 \equiv 139^4 \pmod{27}\]
And since $139 = 5\cdot 27 + 4$, $139 \equiv 4 \pmod{27}$.  Then by theorem 1.18 $139^4 \equiv 4^4 \pmod{27}$.  Therefore $139^{112} \equiv 16 \pmod{27}$.
\end{enumerate}
\end{proof}

%%%Exercise 4.35%%%%%%%%%%%%%%%%%%%%%%%%%%%%%%%%%%%%%%%%%%%%%%%%%%%%%%%%%%%%%%%%%%%%%
\begin{exer}
Find the last digit in the base $10$ representation of the integer
$13^{474}$.
\end{exer}
\begin{proof}[Solution]
This is the same as $13^{474} \pmod{10}$.  Note that $(13, 10) = 1$ and $\phi(10) = 4$.  So by Euler's Theorem $13^4 \equiv 1 \pmod{10}$.  And $474 = 4 \cdot 118 + 2$, so:
\[13^{474} \equiv 13^{4\cdot 118 + 2} \equiv (13^4)^{118}\cdot 13^2 \equiv 1\cdot 13^2\\ \equiv 169 \equiv 9 \pmod{10}\]
\end{proof}
\end{document}
