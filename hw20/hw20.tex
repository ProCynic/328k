\documentclass[12pt,leqno]{article}

\usepackage{amsmath,amsfonts,amssymb,amscd,amsthm,amsbsy,upref}


\textheight=8.5truein
\textwidth=6.0truein
\hoffset=-.5truein
\voffset=-.5truein
\numberwithin{equation}{section}
\pagestyle{headings}
\footskip=36pt


\swapnumbers
\newtheorem{thm}{Theorem}[section]
\newtheorem{hthm}[thm]{*Theorem}
\newtheorem{lem}[thm]{Lemma}
\newtheorem{cor}[thm]{Corollary}
\newtheorem{prop}[thm]{Proposition}
\newtheorem{con}[thm]{Conjecture}
\newtheorem{exer}[thm]{Exercise}
\newtheorem{bpe}[thm]{Blank Paper Exercise}
\newtheorem{apex}[thm]{Applications Exercise}
\newtheorem{ques}[thm]{Question}
\newtheorem{scho}[thm]{Scholium}
\newtheorem*{Exthm}{Example Theorem}
\newtheorem*{Thm}{Theorem}
\newtheorem*{Con}{Conjecture}
\newtheorem*{Axiom}{Axiom}



\theoremstyle{definition}
\newtheorem*{Ex}{Example}
\newtheorem*{Def}{Definition}


\newcommand{\lcm}{\operatorname{lcm}}
\newcommand{\ord}{\operatorname{ord}}
\def\pfrac#1#2{{\left(\frac{#1}{#2}\right)}}


\makeindex

\begin{document}




\thispagestyle{plain}
\begin{flushright}
\large{\textbf{Geoffrey Parker - grp352 \\
HW 20: 4.1 - 4.6\\
M328K \\
April 5th, 2012 \\}}
\end{flushright}

%\maketitle
\markboth{}{} \setcounter{section}{0} \baselineskip=18pt

\setcounter{tocdepth}{4}


%%%%%%%%This is where you can change the numbering to match the problem you
%%%%%%%%are on.  Set the section to  the current chapter.

\setcounter{section}{4}

%%%%%%%%%Now, set the theorem number to one less than the first theorem in
%%%%%%%%%the assignment.
\setcounter{thm}{0}

%%%Exercise 4.1%%%%%%%%%%%%%%%%%%%%%%%%%%%%%%%%%%%%%%%%%%%%%%%%%%%%%%%%%%%%%%%%%%%%%

\begin{exer}For $i = 0$, $1$, $2$, $3$, $4$, $5$, and $6$,
find the number in the canonical complete residue
system\index{canonical complete residue system modulo $n$} to which
$2^i$ is congruent modulo $7$.  In other words, compute
$2^0\pmod{7}, 2^1\pmod{7}, 2^2\pmod{7}, \dots, 2^6\pmod{7}$.
\end{exer}
\begin{proof}[Solution]
\begin{align*}
2^0\pmod{7} &= 1\\
2^1\pmod{7} &= 2\\
2^2\pmod{7} &= 4\\
2^3\pmod{7} &= 1\\
2^4\pmod{7} &= 2\\
2^5\pmod{7} &= 4\\
2^6\pmod{7} &= 1
\end{align*}
\end{proof}

%%%Theorem 4.2%%%%%%%%%%%%%%%%%%%%%%%%%%%%%%%%%%%%%%%%%%%%%%%%%%%%%%%%%%%%%%%%%%%%%

\begin{thm}
Let $a$ and $n$ be natural numbers with $(a, n) = 1$.  Then $(a^j,
n) = 1$ for any natural number $j$.
\end{thm}
\begin{proof}[Proof]
Let $a$ and $n$ be natural numbers with $(a, n) = 1$ We will show by induction that $(a^j,n) = 1$ for any natural number $j$.
As a base case, consider $j = 1$.  In this case, $(a^j,n) = 1$ is simply $(a, n) = 1$, which is given.
Our inductive hypothesis is that there exists some natural number $k<j$ such that $(a^k,n) = 1$.
For our inductive step, since $(a^k,n) = 1$ and $(a, n) = 1$, then by theorem 1.43 $(a^{k+1},n) = 1$.

\end{proof}

\pagebreak
%%%Theorem 4.3%%%%%%%%%%%%%%%%%%%%%%%%%%%%%%%%%%%%%%%%%%%%%%%%%%%%%%%%%%%%%%%%%%%%%

\begin{thm}
Let $a$, $b$, and $n$ be integers with $n>0$ and $(a, n) = 1$.  If
$a\equiv b \pmod{n}$, then $(b, n) = 1$.
\end{thm}
\begin{proof}[Proof]
Let $a$, $b$, and $n$ be integers with $n>0$, $(a, n) = 1$, and $a\equiv b \pmod{n}$. We will show that $(b, n) = 1$.
Using the definitions of divides and congruence, we can say that:
\[n \mid a - b\]
\[nm = a - b\]
\[a = mn + b\]
for some integer $m$.  Because $n \neq 0$, we can say by theorem 1.33 that $(a, n) = (b, n)$.  Therefore $(b, n) = 1$.
\end{proof}

%%%Theorem 4.4%%%%%%%%%%%%%%%%%%%%%%%%%%%%%%%%%%%%%%%%%%%%%%%%%%%%%%%%%%%%%%%%%%%%%

\begin{thm}
Let $a$ and $n$ be natural numbers.  Then there exist natural
numbers $i$ and $j$, with $i \neq j$, such that $a^i \equiv a^j
\pmod{n}$.
\end{thm}
\begin{proof}[Proof]
Let $a$ and $n$ be natural numbers.  The definition of complete residue systems says that every natural number $x$ is 
congruent modulo $n$ to exactly one element of the canonical complete residue system modulo $n$, which has $n$ elements.
Consider the set of integers $S = \{a^1, a^2, \dots a^{n+1}\}$.  Since $S$ has $n+1$ elements and each element is congruent to
exactly one element of the canonical complete residue system modulo $n$, then by the pigeonhole principle there must be two elements of $S$,
call them $a^i$ and $a^j$, which are congruent modulo $n$ to the same element of the residue system, call it $x$.  And since $a^i \equiv x \pmod{n}$
and $a^j \equiv x \pmod{n}$, by theorem 1.11 $a^i \equiv a^j \pmod{n}$.
\end{proof}

%%%Theorem 4.5%%%%%%%%%%%%%%%%%%%%%%%%%%%%%%%%%%%%%%%%%%%%%%%%%%%%%%%%%%%%%%%%%%%%%

\begin{thm}
Let $a$, $b$, $c$, and $n$ be integers with $n
> 0$.  If $ac \equiv bc \pmod{n}$ and $(c, n) = 1$, then $a \equiv b
\pmod{n}$.
\end{thm}
\begin{proof}[Proof]
Let $a$, $b$, $c$, and $n$ be integers with $n > 0$,  $ac \equiv bc \pmod{n}$, and $(c, n) = 1$.
By the definition of congruence:
\[n \mid ac - bc\]
\[n \mid c(a - b)\]
and since $(c, n) = 1$, by theorem 1.41 $n \mid a - b$.  Therefore by the definition of congruence $a \equiv b \pmod{n}$.\\

Also, this is just theorem 1.45 again.
\end{proof}

%%%Theorem 4.6%%%%%%%%%%%%%%%%%%%%%%%%%%%%%%%%%%%%%%%%%%%%%%%%%%%%%%%%%%%%%%%%%%%%%

\begin{thm}
Let $a$ and $n$ be natural numbers with $(a, n) = 1$.  Then there
exists a natural number $k$ such that $a^k \equiv 1 \pmod{n}$.
\end{thm}
\begin{proof}[Proof]
Let $a$ and $n$ be natural numbers with $(a, n) = 1$.
\end{proof}

\pagebreak
\setcounter{section}{3}
\setcounter{thm}{28}

%%%Theorem 3.29%%%%%%%%%%%%%%%%%%%%%%%%%%%%%%%%%%%%%%%%%%%%%%%%%%%%%%%%%%%%%%%%%%%%%
\begin{thm}[Chinese Remainder Theorem]
Suppose $n_1, n_2, \hdots, n_L$ are positive integers that are
pairwise relatively prime, that is, $(n_i, n_j)=1$ for $i\neq j$,
$1\leq i, j \leq L$.  Then the system of $L$ congruences
\[ \begin{array}{c}
x \equiv a_1 \pmod{n_1} \\
x \equiv a_2 \pmod{n_2} \\
\vdots \\
x \equiv a_L \pmod{n_L} \end{array} \] has a unique solution modulo
the product $n_1 n_2  n_3 \cdots n_L$.
\end{thm}

\begin{proof}[Proof]
Suppose $n_1, n_2, \hdots, n_L$ are positive integers that are pairwise relatively prime.  We will show by induction that the system $L$ congruences
\[ \begin{array}{c}
x \equiv a_1 \pmod{n_1} \\
x \equiv a_2 \pmod{n_2} \\
\vdots \\
x \equiv a_L \pmod{n_L} \end{array} \]
has a unique solution modulo the product $n_1 n_2 \cdots n_L$.\\
As our basecase, suppose $L = 2$.  In this case, because $(n_1, n_2) = 1$, theorem 3.28 says that there is a unique solution to the system of equations modulo $n_1 n_2$.\\
As our induction hypothesis, assume that there exists some $k \geq 2$ such that a system of $k$ equations will have $x'$, a unique solution modulo $n_1 n_2 \cdots n_k$.\\

Consider the system of congruences
\begin{align*}
y &\equiv x' \pmod{n_1 n_2 \cdots n_k}\\
y &\equiv a_{k+1} \pmod{n_{k+1}}
\end{align*}
Since all the $n$'s are pairwise coprime, then by lemma 1 $(n_{k+1}, n_1 n_2 \cdots n_k) = 1$.  Therefore by theorem 3.28 the solution $y$ exists.   And because $y \equiv  x' \pmod{n_1 n_2 \cdots n_k}$, $y$ is a solution to the first $k$ congruences.


Lemma 1: Let $p$ be an integer and $n_1, n_2, \ldots , n_m$ be integers which are pairwise relativily prime.  Also, let $p$ be coprime with every $n_i$.  We will show that $(p, n_1 n_2 \cdots n_m) = 1$.  This will be a proof by induction.  As a base case, let $m = 1$.  So $(p, n_1) = 1$ by definition.  Our induction hypothesis is that there exists some $k \geq 1$ such that $(p, n_1 n_2 \cdots n_k) = 1$.  By definition, $(p, n_{k+1}) = 1$, so by theorem 1.43 $(p, n_1 n_2 \cdots n_{k+1}) = 1$. 

\end{proof}
\end{document}
