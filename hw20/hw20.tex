\documentclass[12pt,leqno]{article}

\usepackage{amsmath,amsfonts,amssymb,amscd,amsthm,amsbsy,upref}


\textheight=8.5truein
\textwidth=6.0truein
\hoffset=-.5truein
\voffset=-.5truein
\numberwithin{equation}{section}
\pagestyle{headings}
\footskip=36pt


\swapnumbers
\newtheorem{thm}{Theorem}[section]
\newtheorem{hthm}[thm]{*Theorem}
\newtheorem{lem}[thm]{Lemma}
\newtheorem{cor}[thm]{Corollary}
\newtheorem{prop}[thm]{Proposition}
\newtheorem{con}[thm]{Conjecture}
\newtheorem{exer}[thm]{Exercise}
\newtheorem{bpe}[thm]{Blank Paper Exercise}
\newtheorem{apex}[thm]{Applications Exercise}
\newtheorem{ques}[thm]{Question}
\newtheorem{scho}[thm]{Scholium}
\newtheorem*{Exthm}{Example Theorem}
\newtheorem*{Thm}{Theorem}
\newtheorem*{Con}{Conjecture}
\newtheorem*{Axiom}{Axiom}



\theoremstyle{definition}
\newtheorem*{Ex}{Example}
\newtheorem*{Def}{Definition}


\newcommand{\lcm}{\operatorname{lcm}}
\newcommand{\ord}{\operatorname{ord}}
\def\pfrac#1#2{{\left(\frac{#1}{#2}\right)}}


\makeindex

\begin{document}




\thispagestyle{plain}
\begin{flushright}
\large{\textbf{TYPE YOUR NAME HERE \\
HW 20: 4.1 - 4.6\\
M328K \\
April 5th, 2012 \\}}
\end{flushright}

%\maketitle
\markboth{}{} \setcounter{section}{0} \baselineskip=18pt

\setcounter{tocdepth}{4}


%%%%%%%%This is where you can change the numbering to match the problem you
%%%%%%%%are on.  Set the section to  the current chapter.

\setcounter{section}{4}

%%%%%%%%%Now, set the theorem number to one less than the first theorem in
%%%%%%%%%the assignment.
\setcounter{thm}{0}
%%%Exercise 4.1%%%%%%%%%%%%%%%%%%%%%%%%%%%%%%%%%%%%%%%%%%%%%%%%%%%%%%%%%%%%%%%%%%%%%

\begin{exer}For $i = 0$, $1$, $2$, $3$, $4$, $5$, and $6$,
find the number in the canonical complete residue
system\index{canonical complete residue system modulo $n$} to which
$2^i$ is congruent modulo $7$.  In other words, compute
$2^0\pmod{7}, 2^1\pmod{7}, 2^2\pmod{7}, \dots, 2^6\pmod{7}$.
\end{exer}
\begin{proof}[Solution]
Type your solution here!
\end{proof}

%%%Theorem 4.2%%%%%%%%%%%%%%%%%%%%%%%%%%%%%%%%%%%%%%%%%%%%%%%%%%%%%%%%%%%%%%%%%%%%%

\begin{thm}
Let $a$ and $n$ be natural numbers with $(a, n) = 1$.  Then $(a^j,
n) = 1$ for any natural number $j$.
\end{thm}
\begin{proof}[Proof]
Type your proof here!
\end{proof}

%%%Theorem 4.3%%%%%%%%%%%%%%%%%%%%%%%%%%%%%%%%%%%%%%%%%%%%%%%%%%%%%%%%%%%%%%%%%%%%%

\begin{thm}
Let $a$, $b$, and $n$ be integers with $n>0$ and $(a, n) = 1$.  If
$a\equiv b \pmod{n}$, then $(b, n) = 1$.
\end{thm}
\begin{proof}[Proof]
Type your proof here!
\end{proof}

%%%Theorem 4.4%%%%%%%%%%%%%%%%%%%%%%%%%%%%%%%%%%%%%%%%%%%%%%%%%%%%%%%%%%%%%%%%%%%%%

\begin{thm}
Let $a$ and $n$ be natural numbers.  Then there exist natural
numbers $i$ and $j$, with $i \neq j$, such that $a^i \equiv a^j
\pmod{n}$.
\end{thm}
\begin{proof}[Proof]
Type your proof here!
\end{proof}

%%%Theorem 4.5%%%%%%%%%%%%%%%%%%%%%%%%%%%%%%%%%%%%%%%%%%%%%%%%%%%%%%%%%%%%%%%%%%%%%

\begin{thm}
Let $a$, $b$, $c$, and $n$ be integers with $n
> 0$.  If $ac \equiv bc \pmod{n}$ and $(c, n) = 1$, then $a \equiv b
\pmod{n}$.
\end{thm}
\begin{proof}[Proof]
Type your proof here!
\end{proof}

%%%Theorem 4.6%%%%%%%%%%%%%%%%%%%%%%%%%%%%%%%%%%%%%%%%%%%%%%%%%%%%%%%%%%%%%%%%%%%%%

\begin{thm}
Let $a$ and $n$ be natural numbers with $(a, n) = 1$.  Then there
exists a natural number $k$ such that $a^k \equiv 1 \pmod{n}$.
\end{thm}
\begin{proof}[Proof]
Type your proof here!
\end{proof}
\end{document}
