\documentclass[12pt,leqno]{article}

\usepackage{amsmath,amsfonts,amssymb,amscd,amsthm,amsbsy,upref}


\textheight=8.5truein
\textwidth=6.0truein
\hoffset=-.5truein
\voffset=-.5truein
\numberwithin{equation}{section}
\pagestyle{headings}
\footskip=36pt


\swapnumbers
\newtheorem{thm}{Theorem}[section]
\newtheorem{hthm}[thm]{*Theorem}
\newtheorem{lem}[thm]{Lemma}
\newtheorem{cor}[thm]{Corollary}
\newtheorem{prop}[thm]{Proposition}
\newtheorem{con}[thm]{Conjecture}
\newtheorem{exer}[thm]{Exercise}
\newtheorem{bpe}[thm]{Blank Paper Exercise}
\newtheorem{apex}[thm]{Applications Exercise}
\newtheorem{ques}[thm]{Question}
\newtheorem{scho}[thm]{Scholium}
\newtheorem*{Exthm}{Example Theorem}
\newtheorem*{Thm}{Theorem}
\newtheorem*{Con}{Conjecture}
\newtheorem*{Axiom}{Axiom}



\theoremstyle{definition}
\newtheorem*{Ex}{Example}
\newtheorem*{Def}{Definition}


\newcommand{\lcm}{\operatorname{lcm}}
\newcommand{\ord}{\operatorname{ord}}
\def\pfrac#1#2{{\left(\frac{#1}{#2}\right)}}


\makeindex

\begin{document}




\thispagestyle{plain}
\begin{flushright}
\large{\textbf{TYPE YOUR NAME HERE \\
HW 22: 4.13 - 4.16\\
M328K \\
April 12th, 2012 \\}}
\end{flushright}

%\maketitle
\markboth{}{} \setcounter{section}{0} \baselineskip=18pt

\setcounter{tocdepth}{4}


%%%%%%%%This is where you can change the numbering to match the problem you
%%%%%%%%are on.  Set the section to  the current chapter.

\setcounter{section}{4}

%%%%%%%%%Now, set the theorem number to one less than the first theorem in
%%%%%%%%%the assignment.
\setcounter{thm}{12}

%%%Theorem 4.13%%%%%%%%%%%%%%%%%%%%%%%%%%%%%%%%%%%%%%%%%%%%%%%%%%%%%%%%%%%%%%%%%%%%%

\begin{thm}
Let $p$ be a prime and let $a$ be an integer not divisible by $p$;
that is, $(a, p) = 1$.  Then $\{a, 2a, 3a, \hdots, pa\}$ is a
complete residue system modulo $p$.
\end{thm}
\begin{proof}[Proof]
\end{proof}

%%%Theorem 4.14%%%%%%%%%%%%%%%%%%%%%%%%%%%%%%%%%%%%%%%%%%%%%%%%%%%%%%%%%%%%%%%%%%%%%

\begin{thm}
Let $p$ be a prime and let $a$ be an integer not divisible by $p$.
Then
\[a \cdot 2a \cdot 3a \cdot \hdots \cdot (p-1)a \equiv
    1 \cdot 2 \cdot 3 \cdot \hdots \cdot (p-1) \pmod{p}.\]
\end{thm}
\begin{proof}[Proof]
\end{proof}

%%%Theorem 4.15%%%%%%%%%%%%%%%%%%%%%%%%%%%%%%%%%%%%%%%%%%%%%%%%%%%%%%%%%%%%%%%%%%%%%

\begin{thm}[Fermat's Little Theorem, Version I]
If $p$ is a prime and $a$ is an integer relatively prime to $p$,
then $a^{(p-1)} \equiv 1 \pmod{p}$.
\end{thm}
\begin{proof}[Proof]
\end{proof}

%%%Theorem 4.16%%%%%%%%%%%%%%%%%%%%%%%%%%%%%%%%%%%%%%%%%%%%%%%%%%%%%%%%%%%%%%%%%%%%%
\begin{thm}[Fermat's Little Theorem, Version II]
If $p$ is a prime and $a$ is \emph{any} integer, then $a^p \equiv a
\pmod{p}$.
\end{thm}
\begin{proof}[Proof]
\end{proof}
\end{document}
