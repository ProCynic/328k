\documentclass[12pt,leqno]{article}

\usepackage{amsmath,amsfonts,amssymb,amscd,amsthm,amsbsy,upref}


\textheight=8.5truein
\textwidth=6.0truein
\hoffset=-.5truein
\voffset=-.5truein
\numberwithin{equation}{section}
\pagestyle{headings}
\footskip=36pt


\swapnumbers
\newtheorem{thm}{Theorem}[section]
\newtheorem{hthm}[thm]{*Theorem}
\newtheorem{lem}[thm]{Lemma}
\newtheorem{cor}[thm]{Corollary}
\newtheorem{prop}[thm]{Proposition}
\newtheorem{con}[thm]{Conjecture}
\newtheorem{exer}[thm]{Exercise}
\newtheorem{bpe}[thm]{Blank Paper Exercise}
\newtheorem{apex}[thm]{Applications Exercise}
\newtheorem{ques}[thm]{Question}
\newtheorem{scho}[thm]{Scholium}
\newtheorem*{Exthm}{Example Theorem}
\newtheorem*{Thm}{Theorem}
\newtheorem*{Con}{Conjecture}
\newtheorem*{Axiom}{Axiom}



\theoremstyle{definition}
\newtheorem*{Ex}{Example}
\newtheorem*{Def}{Definition}


\newcommand{\lcm}{\operatorname{lcm}}
\newcommand{\ord}{\operatorname{ord}}
\def\pfrac#1#2{{\left(\frac{#1}{#2}\right)}}


\makeindex

\begin{document}




\thispagestyle{plain}
\begin{flushright}
\large{\textbf{Geoffrey Parker - grp352\\
HW 22: 4.13 - 4.16\\
M328K \\
April 12th, 2012 \\}}
\end{flushright}

%\maketitle
\markboth{}{} \setcounter{section}{0} \baselineskip=18pt

\setcounter{tocdepth}{4}


%%%%%%%%This is where you can change the numbering to match the problem you
%%%%%%%%are on.  Set the section to  the current chapter.

\setcounter{section}{4}

%%%%%%%%%Now, set the theorem number to one less than the first theorem in
%%%%%%%%%the assignment.
\setcounter{thm}{12}

%%%Theorem 4.13%%%%%%%%%%%%%%%%%%%%%%%%%%%%%%%%%%%%%%%%%%%%%%%%%%%%%%%%%%%%%%%%%%%%%

\begin{thm}
Let $p$ be a prime and let $a$ be an integer not divisible by $p$;
that is, $(a, p) = 1$.  Then $\{a, 2a, 3a, \hdots, pa\}$ is a
complete residue system modulo $p$.
\end{thm}
\begin{proof}[Proof]
Let $p$ be a prime and let $a$ be an integer not divisible by $p$; that is, $(a, p) = 1$.  Let $S = \{a, 2a, 3a, \hdots, pa\}$.  Let $i$ and $j$ be arbitrary natural numbers with $1 \leq j < i \leq p$.  Since $p \nmid a$, then by theorem 2.26 $(a, p) = 1$.  Also, since $0 < j < i < p$, $0 < i - j < p$.  So $p \nmid i - j$ and by 2.26 again  $(p, i-j) = 1$.  Then by theorem 2.29 $(p, a(i-j)) = 1$ and by 2.26 $p \nmid a(i-j)$. Expanding this, we get $p \nmid ai - aj$, which means $ai \not \equiv aj \pmod{p}$.  Now we have shown that $S$ has $p$ elements and that all elements of $S$ are pairwise incongruent modulo $p$.  Therefore by theorem 3.17 $S$ is a complete residue system modulo $p$.
\end{proof}

%%%Theorem 4.14%%%%%%%%%%%%%%%%%%%%%%%%%%%%%%%%%%%%%%%%%%%%%%%%%%%%%%%%%%%%%%%%%%%%%

\begin{thm}
Let $p$ be a prime and let $a$ be an integer not divisible by $p$.
Then
\[a \cdot 2a \cdot 3a \cdot \hdots \cdot (p-1)a \equiv
    1 \cdot 2 \cdot 3 \cdot \hdots \cdot (p-1) \pmod{p}.\]
\end{thm}
\begin{proof}[Proof]
Let $p$ be a prime and let $a$ be an integer not divisible by $p$.  Let $S = \{a, 2a, \hdots, pa\}$.  By theorem 4.13 $S$ is a complete residue system modulo $p$, which means all elements of $S$ are pairwise incongruent modulo $p$.  This means that each element of $S$ must be congruent modulo $p$ to a \emph{distinct} element of the canonical complete residue system modulo $p$.  Note that since by theorem 1.6 $p \mid pa$, we have $p \mid pa - 0$ and $pa \equiv 0 \pmod{p}$.  Therefore each element of $T = \{a, 2a, \hdots (p-1)a\}$ is congruent to a distinct element of $R = \{1, 2, \hdots (p-1)\}$.\\

\newpage
Now we will show by induction that the product of the elements of $T$ is congruent modulo $p$ to the product of the elements of $R$.  We will name the elements of $T$ and $R$ as $t_1, t_2, \dots , t_{p-1}$ and $r_1, r_2, \dots , r_{p-1}$ respectively such that $t_1 \equiv r_1 \pmod{p}, t_2 \equiv r_2 \pmod{p}, \dots , t_{p-1} \equiv r_{p-1} \pmod{p}.$\\

As our base case, $t_1 \equiv r_1 \pmod{p}$ by definition.\\

Our inductive hypothesis is that there exists an integer $k$ such that $1 \leq k < p-1$  and $t_1 \cdot t_2 \cdot \hdots \cdot t_k \equiv r_1 \cdot r_2 \cdot \hdots \cdot r_k \pmod{p}$.\\

For our inductive step, since $t_1 \cdot t_2 \cdot \hdots \cdot t_k \equiv r_1 \cdot r_2 \cdot \hdots \cdot r_k \pmod{p}$ and $t_{k+1} \equiv r_{k+1}$ then by theorem 1.14  $t_1 \cdot t_2 \cdot \hdots \cdot t_{k+1} \equiv r_1 \cdot r_2 \cdot \hdots \cdot r_{k+1} \pmod{p}$.\\

Therefore $a \cdot 2a \cdot 3a \cdot \hdots \cdot (p-1)a \equiv 1 \cdot 2 \cdot 3 \cdot \hdots \cdot (p-1) \pmod{p}$.

\end{proof}

%%%Theorem 4.15%%%%%%%%%%%%%%%%%%%%%%%%%%%%%%%%%%%%%%%%%%%%%%%%%%%%%%%%%%%%%%%%%%%%%

\begin{thm}[Fermat's Little Theorem, Version I]
If $p$ is a prime and $a$ is an integer relatively prime to $p$,
then $a^{(p-1)} \equiv 1 \pmod{p}$.
\end{thm}
\begin{proof}[Proof]
Let  $p$ be prime and $a$ be an integer relatively prime to $p$.  Let $t = 1 \cdot 2 \cdot 3 \cdot \hdots \cdot (p-1)$.  Then by theorem 4.14 $a^{p-1}t \equiv t \pmod{p}$.  Using the definition of congruece, we obtain $p \mid a^{p-1}t - t$, or $p \mid t(a^{p-1}-1)$.  So by theorem 2.27  $p \mid t$ or $p \mid a^{p-1}-1$. By our lemma, $p \nmid t$.  Therefore $p \mid a^{p-1}-1$, or $a^{(p-1)} \equiv 1 \pmod{p}$.\\

Lemma:  If $p$ is a prime number then $p \nmid 1 \cdot 2 \cdot \hdots \cdot (p-1)$.\\
Let $p$ be a prime number.  We will show by induction that $p \nmid 1 \cdot 2 \cdot \hdots \cdot (p-1)$.\\

As our base case, consider $p \nmid 1$. Then $(p, 1) = 1$.

Our induction hypothesis is that there exists some $k$ where $1 \leq k < p-1$ and $(p, 1 \cdot 2 \cdot \hdots \cdot k) = 1$.

For induction step, we know that $(p, 1 \cdot 2 \cdot \hdots \cdot k) = 1$.  Also, since $p$ is prime and $k+1 < p$ we know that $(p, k+1) = 1$.  Therefore by theorem 2.29  $(p, 1 \cdot 2 \cdot \hdots \cdot k \cdot k+1) = 1$.\\

Now that we have shown that $(p, 1 \cdot 2 \cdot \hdots \cdot (p-1)) = 1$, we know by theorem 2.26 that $p \nmid 1 \cdot 2 \cdot \hdots \cdot (p-1)$.
\end{proof}

%%%Theorem 4.16%%%%%%%%%%%%%%%%%%%%%%%%%%%%%%%%%%%%%%%%%%%%%%%%%%%%%%%%%%%%%%%%%%%%%
\begin{thm}[Fermat's Little Theorem, Version II]
If $p$ is a prime and $a$ is \emph{any} integer, then $a^p \equiv a
\pmod{p}$.
\end{thm}
\begin{proof}[Proof]
Let $p$ be a prime and $a$ be any integer.  Consider two cases:\\

Case 1: $p \mid a$.  In this case, since $p \mid a$ then by theorem 1.6 $p \mid a \cdot a^{p-1}$, or $p \mid a^p$.  So by theorem 1.2 $p \mid a^p - p$ and by the definition of congruence $a^p \equiv a \pmod{p}$.\\

Case 2: $p \nmid a$.  In this case, theorem 4.15 states that $a^{(p-1)} \equiv 1 \pmod{p}$.  So by theorem 1.14   $a^{(p-1)}a \equiv 1a \pmod{p}$, or equivalently $a^p \equiv a \pmod{p}$.
\end{proof}
\end{document}
