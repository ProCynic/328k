
\documentclass[12pt,leqno]{article}

\usepackage{amsmath,amsfonts,amssymb,amscd,amsthm,amsbsy,upref}


\textheight=8.5truein
\textwidth=6.0truein
\hoffset=-.5truein
\voffset=-.5truein
\numberwithin{equation}{section}
\pagestyle{headings}
\footskip=36pt


\swapnumbers
\newtheorem{thm}{Theorem}[section]
\newtheorem{hthm}[thm]{*Theorem}
\newtheorem{lem}[thm]{Lemma}
\newtheorem{cor}[thm]{Corollary}
\newtheorem{prop}[thm]{Proposition}
\newtheorem{con}[thm]{Conjecture}
\newtheorem{exer}[thm]{Exercise}
\newtheorem{bpe}[thm]{Blank Paper Exercise}
\newtheorem{apex}[thm]{Applications Exercise}
\newtheorem{ques}[thm]{Question}
\newtheorem{scho}[thm]{Scholium}
\newtheorem*{Exthm}{Example Theorem}
\newtheorem*{Thm}{Theorem}
\newtheorem*{Con}{Conjecture}
\newtheorem*{Axiom}{Axiom}



\theoremstyle{definition}
\newtheorem*{Ex}{Example}
\newtheorem*{Def}{Definition}


\newcommand{\lcm}{\operatorname{lcm}}
\newcommand{\ord}{\operatorname{ord}}
\def\pfrac#1#2{{\left(\frac{#1}{#2}\right)}}


\makeindex

\begin{document}




\thispagestyle{plain}
\begin{flushright}
\large{\textbf{TYPE YOUR NAME HERE \\
HW 5: 1.38-1.43, 1.45\\
M328K \\
February 2nd, 2012 \\}}
\end{flushright}

%\maketitle
\markboth{}{} \setcounter{section}{0} \baselineskip=18pt

\setcounter{tocdepth}{4}


%%%%%%%%This is where you can change the numbering to match the problem you
%%%%%%%%are on.  Set the section to  the current chapter.

\setcounter{section}{1}

%%%%%%%%%Now, set the theorem number to one less than the first theorem in
%%%%%%%%%the assignment.
\setcounter{thm}{37}

%%%%%%%%%%%%1.38%%%%%%%%%%%%%%%%%%%%%%%%%%%%%%%%%%%%%%%%%%%%%%%%%%%%%
\begin{thm}
Let $a$ and $b$ be integers. If $(a, b) = 1$, then there exist
integers $x$ and $y$ such that $ax + by = 1$. \end{thm}

\begin{proof}[Proof]
Type your proof here!
\end{proof}

%%%%%%%%%%%%1.39%%%%%%%%%%%%%%%%%%%%%%%%%%%%%%%%%%%%%%%%%%%%%%%%%%%%%
\begin{thm}
Let $a$ and $b$ be integers. If there exist integers $x$ and $y$
with $ax + by = 1$, then $(a, b) = 1$.
\end{thm}

\begin{proof}[Proof]
Type your proof here!
\end{proof}

%%%%%%%%%%%%1.40%%%%%%%%%%%%%%%%%%%%%%%%%%%%%%%%%%%%%%%%%%%%%%%%%%%%%
\begin{thm}
For any integers $a$ and $b$ not both $0$, there are integers $x$
and $y$ such that \[ax + by = (a, b).\]
\end{thm}

\begin{proof}[Proof]
Type your proof here!
\end{proof}

%%%%%%%%%%%%1.41%%%%%%%%%%%%%%%%%%%%%%%%%%%%%%%%%%%%%%%%%%%%%%%%%%%%%
\begin{thm}
Let $a$, $b$, and $c$ be integers.  If $a|bc$ and $(a, b) = 1$, then
$a|c$.
\end{thm}

\begin{proof}[Proof]
Type your proof here!
\end{proof}

%%%%%%%%%%%%1.42%%%%%%%%%%%%%%%%%%%%%%%%%%%%%%%%%%%%%%%%%%%%%%%%%%%%%
\begin{thm}
Let $a$, $b$, and $n$ be integers.  If $a|n$, $b|n$ and $(a, b) =
1$, then $ab|n$.
\end{thm}

\begin{proof}[Proof]
Type your proof here!
\end{proof}

%%%%%%%%%%%%1.43%%%%%%%%%%%%%%%%%%%%%%%%%%%%%%%%%%%%%%%%%%%%%%%%%%%%%
\begin{thm}
Let $a$, $b$, and $n$ be integers. If $(a, n) = 1$ and $(b, n) = 1$,
then~\mbox{$(ab, n) = 1$}.
\end{thm}

\begin{proof}[Proof]
Type your proof here!
\end{proof}

\setcounter{thm}{44}

%%%%%%%%%%%%1.45%%%%%%%%%%%%%%%%%%%%%%%%%%%%%%%%%%%%%%%%%%%%%%%%%%%%%
\begin{thm}
Let $a$, $b$, $c$ and $n$ be integers with $n > 0$.  If $ac \equiv
bc \pmod{n}$ and $(c,n) = 1$, then $a \equiv b \pmod{n}$.
\end{thm}


\begin{proof}[Proof]
Type your proof here!
\end{proof}

\end{document}
