\documentclass[12pt,leqno]{article}

\usepackage{amsmath,amsfonts,amssymb,amscd,amsthm,amsbsy,upref,enumerate}


\textheight=8.5truein
\textwidth=6.0truein
\hoffset=-.5truein
\voffset=-.5truein
\numberwithin{equation}{section}
\pagestyle{headings}
\footskip=36pt


\swapnumbers
\newtheorem{thm}{Theorem}[section]
\newtheorem{hthm}[thm]{*Theorem}
\newtheorem{lem}[thm]{Lemma}
\newtheorem{cor}[thm]{Corollary}
\newtheorem{prop}[thm]{Proposition}
\newtheorem{con}[thm]{Conjecture}
\newtheorem{exer}[thm]{Exercise}
\newtheorem{bpe}[thm]{Blank Paper Exercise}
\newtheorem{apex}[thm]{Applications Exercise}
\newtheorem{ques}[thm]{Question}
\newtheorem{scho}[thm]{Scholium}
\newtheorem*{Exthm}{Example Theorem}
\newtheorem*{Thm}{Theorem}
\newtheorem*{Con}{Conjecture}
\newtheorem*{Axiom}{Axiom}



\theoremstyle{definition}
\newtheorem*{Ex}{Example}
\newtheorem*{Def}{Definition}


\newcommand{\lcm}{\operatorname{lcm}}
\newcommand{\ord}{\operatorname{ord}}
\def\pfrac#1#2{{\left(\frac{#1}{#2}\right)}}

\newcommand{\card}[1]{\left| #1 \right|}



\makeindex

\begin{document}




\thispagestyle{plain}
\begin{flushright}
\large{\textbf{Geoffrey Parker - grp352 \\
HW 5: 1.38-1.43, 1.45\\
M328K \\
February 2nd, 2012 \\}}
\end{flushright}

%\maketitle
\markboth{}{} \setcounter{section}{0} \baselineskip=18pt

\setcounter{tocdepth}{4}


%%%%%%%%This is where you can change the numbering to match the problem you
%%%%%%%%are on.  Set the section to  the current chapter.

\setcounter{section}{1}

%%%%%%%%%Now, set the theorem number to one less than the first theorem in
%%%%%%%%%the assignment.
\setcounter{thm}{37}

%%%%%%%%%%%%1.38%%%%%%%%%%%%%%%%%%%%%%%%%%%%%%%%%%%%%%%%%%%%%%%%%%%%%
\begin{thm}
Let $a$ and $b$ be integers. If $(a, b) = 1$, then there exist
integers $x$ and $y$ such that $ax + by = 1$.
\end{thm}

\begin{proof}[Proof]
Let $a$ and $b$ be integers with $(a, b) = 1$.  Since $(a, b) = 1$, it must be the case that $a$ and $b$ are not both $0$.  Therefore by theorem 1.40, there exist integers $x$ and $y$ such that $1 = (a, b) = ax + by$.
\end{proof}

%%%%%%%%%%%%1.39%%%%%%%%%%%%%%%%%%%%%%%%%%%%%%%%%%%%%%%%%%%%%%%%%%%%%
\begin{thm}
Let $a$ and $b$ be integers. If there exist integers $x$ and $y$
with $ax + by = 1$, then $(a, b) = 1$.
\end{thm}

\begin{proof}[Proof]
Let $a$, $b$, $x$, and $y$ be integers with $ax + by = 1$.  We will show that $(a, b) = 1$.  This is true by theorem 1.40.
\end{proof}

%%%%%%%%%%%%1.40%%%%%%%%%%%%%%%%%%%%%%%%%%%%%%%%%%%%%%%%%%%%%%%%%%%%%
\begin{thm}
For any integers $a$ and $b$ not both $0$, there are integers $x$
and $y$ such that \[ax + by = (a, b).\]
\end{thm}

\begin{proof}[Proof]
Let $a$ and $b$ be integers not both $0$.  First, we will redifine the Euclidean Algorithm as a collection of sequences, with a couple of extensions.  Call it the Extended Euclidean Algorithm.  Let $i$, $j$, $q$, $r$, $x$, and $y$ be sequences of integers, defined as follows: $i_k = j_{k-1}$, $j_k = r_{k-1}$, use the division algorithm to find $q_k$ and $r_k$ such that $i_k = j_kq_k + r_k$, $x_k = x_{k-2} - x_{k-1}$, and $y_k = y_{k-2} - y_{k-1}$.  Now take these initial values: $i_2 = a$, $j_2 = b$, $x_0 = 1$, $y_0 = 0$, $x_1 = 0$, and $y_1 = 1$.  Also let $r_1 = j_2$.  In effect, by filling out these sequences until you find a $k$ such that $r_k = 0$, you are performing the Euclidean Algorithm.  In addition, we will use induction to prove that for any $k \geq 2$, $r_k = ax_k + by_k$.\\

As a base case, take $k = 2$.  This gives us:
\begin{align*}
i_2 &= j_2q_2 + r_2\\
r_2 &= i_2 - q_2j_2\\
r_2 &= a - q_2b\\
r_2 &= (ax_0 + by_0) - q_2(ax_1 + by_1)\\
r_2 &= a(x_0 - q_2x_1) + b(y_0 + q_2y_1)\\
r_2 &= ax_2 + by_2
\end{align*}

Our induction hypothesis is that there exists some integer $N \geq 3$ such that $r_N = ax_N + by_N$.  We must now show that $r_{N+1} = ax_{N+1} + by_{N+1}$.
\begin{align*}
r_N &= ax_N + by_N\\
i_{N+1} &= j_N\\
j_{N+1} &= r_N\\
i_{N+1} &= q_{N+1}j_{N+1} + r_{N+1}\\
r_{N+1} &= i_{N+1} - q_{N+1}j_{N+1}\\
r_{N+1} &= r_{N-1} - q_{N+1}r_N\\
r_{N+1} &= (ax_{N-1} + by_{N-1}) - q_{N+1}(ax_N + by_N)\\
r_{N+1} &= a(x_{N-1} - q_{N+1}x_N) + b(y_{N-1} - q_{N+1}y_N)\\
r_{N+1} &= ax_{N+1} + by_{N+1}
\end{align*}

Now suppose we let $M$ be an integer such that $r_M = 0$.  Since this is the Euclidean Algorithm, this means that $(a, b) = j_M = r_{M-1} = ax_{M-1} + by_{M-1}$.
\end{proof}

%%%%%%%%%%%%1.41%%%%%%%%%%%%%%%%%%%%%%%%%%%%%%%%%%%%%%%%%%%%%%%%%%%%%
\begin{thm}
Let $a$, $b$, and $c$ be integers.  If $a|bc$ and $(a, b) = 1$, then
$a|c$.
\end{thm}

\begin{proof}[Proof]
Let $a$, $b$, and $c$ be integers with $a \mid bc$ and $(a, b) = 1$. We will show $a \mid c$.  First, if $a = 1$, then $a \mid c$ because $1$ divides all integers.  And $a$ can not be $0$ because then $(a, b)$ would be $0$.  Now consider the case of $\card{a} > 1$.  Suppose by way of contradiction that $a \mid b$.  Then $\card{a} \mid a$ and $\card{a} \mid b$, and because $\card{a} > 1$, $\card{a} > (a, b)$, which is a contradiction.  So $a \nmid b$.  Therefore $a \mid c$
\end{proof}

\pagebreak
%%%%%%%%%%%%1.42%%%%%%%%%%%%%%%%%%%%%%%%%%%%%%%%%%%%%%%%%%%%%%%%%%%%%
\begin{thm}
Let $a$, $b$, and $n$ be integers.  If $a|n$, $b|n$ and $(a, b) =
1$, then $ab|n$.
\end{thm}

\begin{proof}[Proof]
Let $a$, $b$, and $n$ be integers with $a \mid n$, $b \mid n$ and $(a, b) = 1$. We will show that $ab \mid n$. Consider the sets $A$ of integers that $a$ divides and $B$ the integers that $b$ divides.  Let $S = A \cap B$.  $S$ is the set of all integers of the form $abk$.  Since $n$ is an element of $S$, $ab \mid n$.
\end{proof}

%%%%%%%%%%%%1.43%%%%%%%%%%%%%%%%%%%%%%%%%%%%%%%%%%%%%%%%%%%%%%%%%%%%%
\begin{thm}
Let $a$, $b$, and $n$ be integers. If $(a, n) = 1$ and $(b, n) = 1$,
then~\mbox{$(ab, n) = 1$}.
\end{thm}

\begin{proof}[Proof]
Let $a$, $b$, and $n$ be integers with $(a, n) = 1$ and $(b, n) = 1$. We will show that $(ab, n) = 1$.  Let $d = (ab, n)$.  Assume by way of contradiction that $d > 1$.  However, since $(a, n) = 1$ and $(b, n) = 1$, it must be the case that $d \nmid a$ and $d \nmid b$.  
\end{proof}
\setcounter{thm}{44}

%%%%%%%%%%%%1.45%%%%%%%%%%%%%%%%%%%%%%%%%%%%%%%%%%%%%%%%%%%%%%%%%%%%%
\begin{thm}
Let $a$, $b$, $c$ and $n$ be integers with $n > 0$.  If $ac \equiv
bc \pmod{n}$ and $(c,n) = 1$, then $a \equiv b \pmod{n}$.
\end{thm}

\begin{proof}[Proof]
Let $a$, $b$, $c$ and $n$ be integers with $n > 0$,  $ac \equiv bc \pmod{n}$, and $(c,n) = 1$. We will show that $a \equiv b \pmod{n}$.  First, by definition of congruence mod n, $n \mid (ac - bc)$, so $n \mid c(a-b)$.  Since $(n, c) = 1$, then by theorem 1.41 $n \mid (a - b)$.  Therefore by definition of conguence mod n, $a \equiv b \pmod{n}$.
\end{proof}
\end{document}