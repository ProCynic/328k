
\documentclass[12pt,leqno]{article}

\usepackage{amsmath,amsfonts,amssymb,amscd,amsthm,amsbsy,upref}


\textheight=8.5truein
\textwidth=6.0truein
\hoffset=-.5truein
\voffset=-.5truein
\numberwithin{equation}{section}
\pagestyle{headings}
\footskip=36pt


\swapnumbers
\newtheorem{thm}{Theorem}[section]
\newtheorem{hthm}[thm]{*Theorem}
\newtheorem{lem}[thm]{Lemma}
\newtheorem{cor}[thm]{Corollary}
\newtheorem{prop}[thm]{Proposition}
\newtheorem{con}[thm]{Conjecture}
\newtheorem{exer}[thm]{Exercise}
\newtheorem{bpe}[thm]{Blank Paper Exercise}
\newtheorem{apex}[thm]{Applications Exercise}
\newtheorem{ques}[thm]{Question}
\newtheorem{scho}[thm]{Scholium}
\newtheorem*{Exthm}{Example Theorem}
\newtheorem*{Thm}{Theorem}
\newtheorem*{Con}{Conjecture}
\newtheorem*{Axiom}{Axiom}



\theoremstyle{definition}
\newtheorem*{Ex}{Example}
\newtheorem*{Def}{Definition}


\newcommand{\lcm}{\operatorname{lcm}}
\newcommand{\ord}{\operatorname{ord}}
\def\pfrac#1#2{{\left(\frac{#1}{#2}\right)}}


\makeindex

\begin{document}




\thispagestyle{plain}
\begin{flushright}
\large{\textbf{TYPE YOUR NAME HERE \\
HW 3: 1.21-1.23, 1.25-1.28\\
M328K \\
January 26th, 2012 \\}}
\end{flushright}

%\maketitle
\markboth{}{} \setcounter{section}{0} \baselineskip=18pt

\setcounter{tocdepth}{4}


%%%%%%%%This is where you can change the numbering to match the problem you
%%%%%%%%are on.  Set the section to  the current chapter.

\setcounter{section}{1}

%%%%%%%%%Now, set the theorem number to one less than the first theorem in
%%%%%%%%%the assignment.
\setcounter{thm}{20}
%%%%%%%%%%%%1.21%%%%%%%%%%%%%%%%%%%%%%%%%%%%%%%%%%%%%%%%%%%%%%%%%%%%%
\begin{thm}
Let a natural number $n$ be expressed in base $10$ as \[n = a_k
a_{k-1} \hdots a_1 a_0.\]  (Note that what we mean by this notation
is that each $a_i$ is a digit of a regular base $10$ number, not
that the $a_i$'s are being multiplied together.) If $m = a_k +
a_{k-1} + \hdots + a_1 + a_0$, then $n \equiv m \pmod{3}$.
\end{thm}

\begin{proof}[Proof]
Type your Proof here!
\end{proof}

%%%%%%%%%%%%1.22%%%%%%%%%%%%%%%%%%%%%%%%%%%%%%%%%%%%%%%%%%%%%%%%%%%%%
\begin{thm}
If a natural number is divisible by $3$, then, when expressed in
base $10$, the sum of its digits is divisible by $3$.
\end{thm}

\begin{proof}[Proof]
Type your proof here!
\end{proof}

%%%%%%%%%%%%1.23%%%%%%%%%%%%%%%%%%%%%%%%%%%%%%%%%%%%%%%%%%%%%%%%%%%%%
\begin{thm}
If the sum of the digits of a natural number expressed in base $10$
is divisible by $3$, then the number is divible by $3$ as well.
\end{thm}

\begin{proof}[Proof]
Type your proof here!
\end{proof}

\setcounter{thm}{24}

%%%%%%%%%%%%1.25%%%%%%%%%%%%%%%%%%%%%%%%%%%%%%%%%%%%%%%%%%%%%%%%%%%%%
\begin{exer} Illustrate the Division Algorithm for: \end{exer}
\begin{enumerate}
    \item[(1)] $m = 25$, $n = 7$.
    \begin{proof}[Solution]
    Type your solution here!
    \end{proof}

    \item[(2)] $m = 277$, $n = 4$.
    \begin{proof}[Solution]
    Type your solution here!
    \end{proof}

    \item[(3)] $m = 33$, $n = 11$.
    \begin{proof}[Solution]
    Type your solution here!
    \end{proof}

    \item[(4)] $m = 33$, $n = 45$.
    \begin{proof}[Solution]
    Type your solution here!
    \end{proof}

\end{enumerate}

%%%%%%%%%%%%1.26%%%%%%%%%%%%%%%%%%%%%%%%%%%%%%%%%%%%%%%%%%%%%%%%%%%%%
\begin{thm}
Prove the existence part of the Division Algorithm. (\textit{Hint}:
Given $n$ and $m$, how will you define $q$?  Once you choose this
$q$, then how is $r$ chosen?  Then show that $0 \leq r \leq n-1$.)
\end{thm}

\begin{proof}[Proof]
Type your proof here!
\end{proof}

%%%%%%%%%%%%1.27%%%%%%%%%%%%%%%%%%%%%%%%%%%%%%%%%%%%%%%%%%%%%%%%%%%%%
\begin{thm}
Prove the uniqueness part of the Division Algorithm.

(\textit{Hint}: If $nq + r = nq' + r'$, then $nq -nq' = r' - r$. Use
what you know about $r$ and $r'$ as part of your argument that $q =
q'$.)
\end{thm}

\begin{proof}[Proof]
Type your proof here!
\end{proof}

%%%%%%%%%%%%1.28%%%%%%%%%%%%%%%%%%%%%%%%%%%%%%%%%%%%%%%%%%%%%%%%%%%%%
\begin{thm}
Let $a$, $b$, and $n$ be integers with $n > 0$.  Then $a \equiv b
\pmod{n}$ if and only if $a$ and $b$ have the same remainder when
divided by $n$.  Equivalently, $a \equiv b \pmod{n}$ if and only if
when $a = nq_1 + r_1$ ($0 \leq r_1 \leq n-1$) and $b = nq_2 + r_2$
($0 \leq r_2 \leq n-1$), then~$r_1 = r_2$.
\end{thm}

\begin{proof}[Proof]
Type your proof here!
\end{proof}


\end{document}
