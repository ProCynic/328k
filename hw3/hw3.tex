
\documentclass[12pt,leqno]{article}

\usepackage{amsmath,amsfonts,amssymb,amscd,amsthm,amsbsy,upref}


\textheight=8.5truein
\textwidth=6.0truein
\hoffset=-.5truein
\voffset=-.5truein
\numberwithin{equation}{section}
\pagestyle{headings}
\footskip=36pt


\swapnumbers
\newtheorem{thm}{Theorem}[section]
\newtheorem{hthm}[thm]{*Theorem}
\newtheorem{lem}[thm]{Lemma}
\newtheorem{cor}[thm]{Corollary}
\newtheorem{prop}[thm]{Proposition}
\newtheorem{con}[thm]{Conjecture}
\newtheorem{exer}[thm]{Exercise}
\newtheorem{bpe}[thm]{Blank Paper Exercise}
\newtheorem{apex}[thm]{Applications Exercise}
\newtheorem{ques}[thm]{Question}
\newtheorem{scho}[thm]{Scholium}
\newtheorem*{Exthm}{Example Theorem}
\newtheorem*{Thm}{Theorem}
\newtheorem*{Con}{Conjecture}
\newtheorem*{Axiom}{Axiom}



\theoremstyle{definition}
\newtheorem*{Ex}{Example}
\newtheorem*{Def}{Definition}


\newcommand{\lcm}{\operatorname{lcm}}
\newcommand{\ord}{\operatorname{ord}}
\def\pfrac#1#2{{\left(\frac{#1}{#2}\right)}}


\makeindex

\begin{document}




\thispagestyle{plain}
\begin{flushright}
\large{\textbf{Geoffrey Parker - grp352 \\
HW 3: 1.21-1.23, 1.25-1.28\\
M328K \\
January 26th, 2012 \\}}
\end{flushright}

%\maketitle
\markboth{}{} \setcounter{section}{0} \baselineskip=18pt

\setcounter{tocdepth}{4}


%%%%%%%%This is where you can change the numbering to match the problem you
%%%%%%%%are on.  Set the section to  the current chapter.

\setcounter{section}{1}

%%%%%%%%%Now, set the theorem number to one less than the first theorem in
%%%%%%%%%the assignment.
\setcounter{thm}{20}
%%%%%%%%%%%%1.21%%%%%%%%%%%%%%%%%%%%%%%%%%%%%%%%%%%%%%%%%%%%%%%%%%%%%
\begin{thm}
Let a natural number $n$ be expressed in base $10$ as \[n = a_k
a_{k-1} \hdots a_1 a_0.\]  (Note that what we mean by this notation
is that each $a_i$ is a digit of a regular base $10$ number, not
that the $a_i$'s are being multiplied together.) If $m = a_k +
a_{k-1} + \hdots + a_1 + a_0$, then $n \equiv m \pmod{3}$.
\end{thm}

\begin{proof}[Proof]
Let a natural number $n$ be expressed in base $10$ as $n = a_k a_{k-1} \hdots a_1 a_0$.  Let $m$ be the sum of these digits, that is $m = a_k + a_{k-1} + \hdots + a_1 + a_0$.  We will show that $n \equiv m \pmod{3}$.  First, note that because $n$ is in base $10$, in can be expressed like such: 
\[n = a_k \cdot 10^k + a_{k-1} \cdot 10^{k-1} + \hdots + a_1 \cdot 10^1 + a_0 \cdot 10^0\]  So if we compute $n - m$, we get: 
\[n - m = a_k \cdot 10^k - a_k + a_{k-1} \cdot 10^{k-1} - a_{k-1} + \hdots + a_1 \cdot 10^1 - a_1 + a_0 \cdot 10^0 - a_0\]
Combining terms, we see that:
\[n - m = a_k (10^k-1) + a_{k-1} (10^{k-1}-1) + \hdots + a_1 (10^1-1)\]
Next we see that for any integer $j$, $10^j-1$ will be $9(10^{j-1}) + 9(10^{j-2} + \hdots + 9(10^1) + 9(10^0)$, that is $999\ldots 99$, with the number of nines equal to $10^{j-1}$.  We can also see that this must be equal to $3 \times (3(10^{j-1}) + 3(10^{j-2}) + \hdots + 3(10^1) + 3(10^0))$, because each term in the nines series is three times the corresponding term in the threes series.  Therefore $3$ must divide $10^k-1$, $10^{k-1}-1$, $\ldots$, and $10^1-1$.  So by theorem 1.3, $3$ divides each term in the $n-m$ series.  This means, by theorem 1.1, that $3$ divides the entire series, and thus $n - m$.  Since $3 \mid (n - m)$, by definition of congruence mod n $n \equiv m \pmod{3}$, which is what we set out to prove.
\end{proof}

\pagebreak
%%%%%%%%%%%%1.22%%%%%%%%%%%%%%%%%%%%%%%%%%%%%%%%%%%%%%%%%%%%%%%%%%%%%
\begin{thm}
If a natural number is divisible by $3$, then, when expressed in
base $10$, the sum of its digits is divisible by $3$.
\end{thm}

\begin{proof}[Proof]
Let $n$ be a natural number such that $3 \mid n$.  We will show that, when expressed in base $10$, $3$ also divides the sum of its digits.  First, let $n$ be expressed as: 
\[n = a_k=j a_{j-1} \hdots a_1 a_0\]
where each $a_k$ is a base $10$ digit of $n$.  Then $n$ can be rewritten:
\[n = a_j \cdot 10^j + a_{j-1} \cdot 10^{j-1} + \hdots + a_1 \cdot 10^1 + a_0 \cdot 10^0\]
Since, by the definition of divides there exists some integer $k$ such that $n = 3k$:
\[3k = a_j \cdot 10^j + a_{j-1} \cdot 10^{j-1} + \hdots + a_1 \cdot 10^1 + a_0 \cdot 10^0\]


\end{proof}

%%%%%%%%%%%%1.23%%%%%%%%%%%%%%%%%%%%%%%%%%%%%%%%%%%%%%%%%%%%%%%%%%%%%
\begin{thm}
If the sum of the digits of a natural number expressed in base $10$
is divisible by $3$, then the number is divible by $3$ as well.
\end{thm}

\begin{proof}[Proof]
Type your proof here!
\end{proof}

\setcounter{thm}{24}

%%%%%%%%%%%%1.25%%%%%%%%%%%%%%%%%%%%%%%%%%%%%%%%%%%%%%%%%%%%%%%%%%%%%
\begin{exer} Illustrate the Division Algorithm for: \end{exer}
\begin{enumerate}
    \item[(1)] $m = 25$, $n = 7$.
    \begin{proof}[Solution]
    $m = n \times 3 + 4$
    \end{proof}

    \item[(2)] $m = 277$, $n = 4$.
    \begin{proof}[Solution]
    $m = n \times 69 + 1$
    \end{proof}

    \item[(3)] $m = 33$, $n = 11$.
    \begin{proof}[Solution]
    $m = n \times 11 + 0$
    \end{proof}

    \item[(4)] $m = 33$, $n = 45$.
    \begin{proof}[Solution]
    $m = n \times 0 + 33$
    \end{proof}

\end{enumerate}

%%%%%%%%%%%%1.26%%%%%%%%%%%%%%%%%%%%%%%%%%%%%%%%%%%%%%%%%%%%%%%%%%%%%
\begin{thm}
Prove the existence part of the Division Algorithm. (\textit{Hint}:
Given $n$ and $m$, how will you define $q$?  Once you choose this
$q$, then how is $r$ chosen?  Then show that $0 \leq r \leq n-1$.)
\end{thm}

%Do I need to prove that q >= 0?
\begin{proof}[Proof]
Let $n$ and $m$ be natural numbers.  Let $S$ be the set of natural numbers of the form $m-nk + 1$ for some integer $k$.  Since $m$ is a natural number, $S$ always has at least one member, $m-n0 + 1$.  Therefore by the Well-Ordering Axiom for the natural numbers, there exists a smallest element of the set $S$.  Let $q$ be an integer such that $m-nq + 1$ is this smallest element.  Let $r = m - nq$.  So $m = nq + r$.  Now we will show that $0 \leq r \leq n-1$.  Since $r$ is defined to be drawn from a  subset of the natural numbers, $r \geq 0$.  Assume by way of contradiction that $r \geq n$.  This would mean that $m - nq \geq n$, or $m - nq - n \geq 0$.  So $m - n(q+1) \geq 0$, which makes $m - n(q+1) + 1$ an element of $S$.  But $m - n(q+1) + 1 < m - nq + 1$, which is a contradiction, since $m - nq + 1$ is the smallest element of $S$.  Therefore $r < n$, or, since $r$ and $n$ are integers, $r \leq n-1$.  So we have proved that $0 \leq r \leq n-1$.
\end{proof}

NOTE TO READER: The well ordering axiom we have is only defined on the natural numbers, which is why there are +1's every time S is mentioned.

%%%%%%%%%%%%1.27%%%%%%%%%%%%%%%%%%%%%%%%%%%%%%%%%%%%%%%%%%%%%%%%%%%%%
\begin{thm}
Prove the uniqueness part of the Division Algorithm.

(\textit{Hint}: If $nq + r = nq' + r'$, then $nq -nq' = r' - r$. Use
what you know about $r$ and $r'$ as part of your argument that $q =
q'$.)
\end{thm}

\begin{proof}[Proof]
Let there be integers $m,\ n,\ q,\ r,\ q',\ r'$ such that $m = nq +r$, $m = nq' + r'$, $0 \leq r \leq n$, and $0 \leq r' \leq n$.  Given this, $nq + r = nq' + r'$, or $r - r' = nq' - nq = n(q' - q')$, which means that $n \mid (r - r')$.  So by the definition of divides, there exists some integer $k$ such that $nk = r-r'$.  However, $\leq r \leq n$ and $0 \leq r' \leq n$, so $-n < r-r' < n$, which means that $k$ must be $0$ and $r-r'$ must be $0$.  Therefore $r = r'$.  Going back to  $r - r' = n(q' - q')$, we can say that $n(q' - q') = 0$.  Since $n$ is a natural number, and thus not $0$, $q' - q$ must be $0$.  Therefore $q = q'$.  
\end{proof}

\pagebreak
%%%%%%%%%%%%1.28%%%%%%%%%%%%%%%%%%%%%%%%%%%%%%%%%%%%%%%%%%%%%%%%%%%%%
\begin{thm}
Let $a$, $b$, and $n$ be integers with $n > 0$.  Then $a \equiv b
\pmod{n}$ if and only if $a$ and $b$ have the same remainder when
divided by $n$.  Equivalently, $a \equiv b \pmod{n}$ if and only if
when $a = nq_1 + r_1$ ($0 \leq r_1 \leq n-1$) and $b = nq_2 + r_2$
($0 \leq r_2 \leq n-1$), then~$r_1 = r_2$.
\end{thm}

\begin{proof}[Proof]
Let $a,\ b$, and $n$ be integers with $n > 0$.  This will be a two part proof.\\

First, assume $a \equiv b \pmod{n}$.  By definition of congruence mod n, this means that there exists some integer $k$ such that $kn = a-b$, giving $a = kn + b$ and $b = -kn + a$.  By the Division Algorithm, there exist integers $q_1,\ r_1,\ q_2,\ r_2$ with $0 \leq r_1 \leq n-1$ and $0 \leq r_2 \leq n-1$ such that $a = nq_1 + r_1$ and $b = nq_2 + r_2$
\end{proof}
\end{document}
