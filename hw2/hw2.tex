
\documentclass[12pt,leqno]{article}

\usepackage{amsmath,amsfonts,amssymb,amscd,amsthm,amsbsy,upref}


\textheight=8.5truein
\textwidth=6.0truein
\hoffset=-.5truein
\voffset=-.5truein
\numberwithin{equation}{section}
\pagestyle{headings}
\footskip=36pt


\swapnumbers
\newtheorem{thm}{Theorem}[section]
\newtheorem{hthm}[thm]{*Theorem}
\newtheorem{lem}[thm]{Lemma}
\newtheorem{cor}[thm]{Corollary}
\newtheorem{prop}[thm]{Proposition}
\newtheorem{con}[thm]{Conjecture}
\newtheorem{exer}[thm]{Exercise}
\newtheorem{bpe}[thm]{Blank Paper Exercise}
\newtheorem{apex}[thm]{Applications Exercise}
\newtheorem{ques}[thm]{Question}
\newtheorem{scho}[thm]{Scholium}
\newtheorem*{Exthm}{Example Theorem}
\newtheorem*{Thm}{Theorem}
\newtheorem*{Con}{Conjecture}
\newtheorem*{Axiom}{Axiom}



\theoremstyle{definition}
\newtheorem*{Ex}{Example}
\newtheorem*{Def}{Definition}


\newcommand{\lcm}{\operatorname{lcm}}
\newcommand{\ord}{\operatorname{ord}}
\def\pfrac#1#2{{\left(\frac{#1}{#2}\right)}}


\makeindex

\begin{document}




\thispagestyle{plain}
\begin{flushright}
\large{\textbf{Geoffrey Parker - grp352 \\
HW 2: 1.15 - 1.20, A.10, A.18\\
M328K \\
January 24th, 2012 \\}}
\end{flushright}

%\maketitle
\markboth{}{} \setcounter{section}{0} \baselineskip=18pt

\setcounter{tocdepth}{4}


%%%%%%%%This is where you can change the numbering to match the problem you
%%%%%%%%are on.  Set the section to  the current chapter.

\setcounter{section}{1}

%%%%%%%%%Now, set the theorem number to one less than the first theorem in
%%%%%%%%%the assignment.
\setcounter{thm}{14}

%%%%%%%%%%%%1.15%%%%%%%%%%%%%%%%%%%%%%%%%%%%%%%%%%%%%%%%%%%%%%%%%%%%%
\begin{exer}
Let $a$, $b$, and $n$ be integers with $n
> 0$.  Show that if $a \equiv b \pmod{n}$,
then $a^2 \equiv b^2 \pmod{n}$.
\end{exer}

\begin{proof}[Proof]
Let $a$, $b$, and $n$ be integers with $n > 0$ and $a \equiv b \pmod{n}$. We will show that $a^2 \equiv b^2 \pmod{n}$.  First, since $a \equiv b \pmod{n}$, by definition of congruence mod n, $n \mid (a - b)$.  Then, by Theorem 1.8, we can show that $n \mid (a - b)(a + b)$, since $(a + b)$ is an integer.  So $n \mid (a^2 - b^2)$.  Therefore, by the definition of congruence mod n, $a^2 \equiv b^2 \pmod{n}$.
\end{proof}

%%%%%%%%%%%%1.16%%%%%%%%%%%%%%%%%%%%%%%%%%%%%%%%%%%%%%%%%%%%%%%%%%%%%
\begin{exer}
Let $a$, $b$, and $n$ be integers with $n
> 0$.  Show that if $a \equiv b \pmod{n}$,
then $a^3 \equiv b^3 \pmod{n}$.
\end{exer}

\begin{proof}[Proof]
Let $a$, $b$, and $n$ be integers with $n > 0$ and $a \equiv b \pmod{n}$. We will show that $a^3 \equiv b^3 \pmod{n}$.  First, since $a \equiv b \pmod{n}$, by definition of congruence mod n, $n \mid (a - b)$.  Then, by Theorem 1.8, we can show that $n \mid (a - b)(a^2 + ab + b^2)$, since $(a^2 + ab + b^2)$ is an integer.  So $n \mid (a^3 - b^3)$.  Therefore, by the definition of congruence mod n, $a^3 \equiv b^3 \pmod{n}$.\end{proof}

%%%%%%%%%%%%1.17%%%%%%%%%%%%%%%%%%%%%%%%%%%%%%%%%%%%%%%%%%%%%%%%%%%%
\begin{exer}
Let $a$, $b$, $k$, and $n$ be integers with $n
> 0$ and $k > 1$. Show that if $a \equiv b \pmod{n}$ and
$a^{k-1} \equiv b^{k-1} \pmod{n}$,
then \[a^k \equiv b^k \pmod{n}.\]
\end{exer}
\begin{proof}[Proof]
Let $a$, $b$, $k$, and $n$ be integers with $n > 0$ and $k > 1$.  Also, $a \equiv b \pmod{n}$ and $a^{k-1} \equiv b^{k-1} \pmod{n}$.  We will show that $a^k \equiv b^k \pmod{n}$.  Since $a^{k-1} \equiv b^{k-1} \pmod{n}$, $n \mid (a^{k-1} - b^{k-1})$, by the definition of congruence mod p. Then, by theorem 1.8, $n \mid (a^{k-1} - b^{k-1}) \times {(a b (a^k-b^k)) \over (b a^k-a b^k)}$. So $n | (a^k - b^k)$ which, by the definition of congruence mod n, means that $a^k \equiv b^k \pmod{n}$.
\end{proof}

\pagebreak
%%%%%%%%%%%%1.18%%%%%%%%%%%%%%%%%%%%%%%%%%%%%%%%%%%%%%%%%%%%%%%%%%%%
\begin{thm}
Let $a$, $b$, $k$, and $n$ be integers with $n
> 0$ and $k > 0$.  If $a \equiv b \pmod{n}$,
then \[a^k \equiv b^k \pmod{n}.\]
\end{thm}

\begin{proof}[Proof]
Let $a$, $b$, $k$, and $n$ be integers with $n > 0$ and $k > 0$ and $a \equiv b \pmod{n}$. We will show that $a^k \equiv b^k \pmod{n}$.  First, by theorem 1.15, we know that $a^2 \equiv b^2 \pmod{n}$.  Since theorem 1.17 demonstrates that for any integer $j \geq 1$, $a^{j} \equiv b^{j} \pmod{n}$ implies that $a^{j+1} \equiv b^{j+1} \pmod{n}$.  So we have proved by induction that $a^k \equiv b^k \pmod{n}$.
\end{proof}

%%%%%%%%%%%%1.19%%%%%%%%%%%%%%%%%%%%%%%%%%%%%%%%%%%%%%%%%%%%%%%%%%%%

\begin{exer}
Illustrate each of Theorems $1.12$-$1.18$ with
an example using actual numbers.
\end{exer}

\begin{enumerate}

\item Example for 1.12: $9 \equiv 5 \pmod{4}$ and $7 \equiv 3 \pmod{4}$.  So, $(9 + 7) - (5 + 3) = 16 - 8 = 8$ and $4 \mid 8$.
\item Example for 1.13: $12 \equiv 3 \pmod{3}$ and $17 \equiv 8 \pmod{3}$.  So, $(12 - 17) - (3 - 8) = (-5) - (-5) = 0$ and $3 \mid 0$.
\item Example for 1.14: $6 \equiv 4 \pmod{2}$ and $7 \equiv 3 \pmod{2}$.  So, $6 \times 7 - 4 \times 3 = 42 - 12 = 30$ and $2 \mid 30$.
\item Example for 1.15: $17 \equiv 12 \pmod{5}$.  So $17^2 - 12^2 = 289 - 144 = 145$ and $5 \mid 145$.
\item Example for 1.16: $13 \equiv 4 \pmod{3}$.  So $13^2 - 4^2 = 169 - 16 = 153$ and $3 \mid 153$.
\item Example for 1.17: $19 \equiv 7 \pmod{6}$ and  $19^4 \equiv 7^4 \pmod{6}$ (that is $130321 \equiv 2401 \pmod{6}$.  $130321 - 2401 = 127920 = 21320 \times 6$.)  $19^5 - 7^5 = 2476099 - 16807 = 2459292 = 409882 * 6$
\item Example for 1.18: $7 \equiv 2 \pmod{5}$.  So $7^6 - 2^6 = 117649 - 64 = 117585 = 23517$

\end{enumerate}

%%%%%%%%%%%%1.20%%%%%%%%%%%%%%%%%%%%%%%%%%%%%%%%%%%%%%%%%%%%%%%%%%%%%
\begin{ques}
Let $a$, $b$, $c$, and $n$ be integers for which $ac \equiv bc
\pmod{n}$. Can we conclude that $a \equiv b \pmod{n}$? If you answer
``yes", try and give a proof. If you answer ``no", try and give a
counterexample.
\end{ques}

\begin{proof}[Solution]
No.  $10 \equiv 15 \pmod{5}$, so $2 \times 5 \equiv 3 \times 5 \pmod{5}$, yet $2 \not \equiv 3 \pmod{5}$.
\end{proof}

%%% Notice that the theorems below don't get included in the numbering system
%%% because the "Thm" is capitalized. This is how it's defined in the beginning
%%% of the file.


\pagebreak
%%%%%%%%%%%%A.10%%%%%%%%%%%%%%%%%%%%%%%%%%%%%%%%%%%%%%%%%%%%%%%%%%%%%

\noindent \textbf{A.10 Theorem.} \emph{Let n be a natural number. Then $1+2+3+\cdots + n=\frac{(n)(n+1)}{2}$}

\begin{proof}[Proof]
Proof by induction.
\begin{itemize}
\item
Base Case $(n = 1): {1(1+1) \over 2} = 1$.  True.
\item
Induction Hypothesis:  There exists some natural number N such that $1+2+3+\cdots + N=\frac{(N)(N+1)}{2}$
\item
Then, ${(N+1)(N+2) \over 2}  = {(N)(N+1) + 2(N+1) \over 2} = {(N)(N+1) \over 2} + (N+1)$ \\
So by the induction hypothesis, this equals $1+2+3+\cdots + N + (N + 1)$\\
Therefore ${(N)(N+1) + 2(N+1) \over 2} = 1+2+3+\cdots + (N+1)$.

\end{itemize}
\end{proof}


%%%%%%%%%%%%A.18%%%%%%%%%%%%%%%%%%%%%%%%%%%%%%%%%%%%%%%%%%%%%%%%%%%%%
\noindent \textbf{A.18 Theorem.} \emph{For every natural number n,  $1+2+2^2+\cdots + 2^n=2^{n+1}-1$}

\begin{proof}[Proof]
Proof by induction:
\begin{itemize}
\item
Base Case $(n = 1): 1 + 2^1 = 3$ and $2^{1+1}-1 = 4-1 = 3$.  True.
\item
Induction Hypothesis: There exists some natural number N such that $1+2+2^2+\cdots + 2^N=2^{N+1}-1$.
\item
So, $2^{(N+1)+1}-1 = 2(2^{N+1}) - 1 = 2(2^{N+1} - 1) + 1$. By the induction hypothesis, this equals $(1+2+2^2+\cdots + 2^N)2 + 1 = (2+2^2+2^3+\cdots + 2^{N+1}) + 1$.  Therefore, $2^{(N+1)+1}-1 = 1+2+2^2+\cdots + 2^{N+1}$.
\end{itemize}
\end{proof}



\end{document}
