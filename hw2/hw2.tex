
\documentclass[12pt,leqno]{article}

\usepackage{amsmath,amsfonts,amssymb,amscd,amsthm,amsbsy,upref}


\textheight=8.5truein
\textwidth=6.0truein
\hoffset=-.5truein
\voffset=-.5truein
\numberwithin{equation}{section}
\pagestyle{headings}
\footskip=36pt


\swapnumbers
\newtheorem{thm}{Theorem}[section]
\newtheorem{hthm}[thm]{*Theorem}
\newtheorem{lem}[thm]{Lemma}
\newtheorem{cor}[thm]{Corollary}
\newtheorem{prop}[thm]{Proposition}
\newtheorem{con}[thm]{Conjecture}
\newtheorem{exer}[thm]{Exercise}
\newtheorem{bpe}[thm]{Blank Paper Exercise}
\newtheorem{apex}[thm]{Applications Exercise}
\newtheorem{ques}[thm]{Question}
\newtheorem{scho}[thm]{Scholium}
\newtheorem*{Exthm}{Example Theorem}
\newtheorem*{Thm}{Theorem}
\newtheorem*{Con}{Conjecture}
\newtheorem*{Axiom}{Axiom}



\theoremstyle{definition}
\newtheorem*{Ex}{Example}
\newtheorem*{Def}{Definition}


\newcommand{\lcm}{\operatorname{lcm}}
\newcommand{\ord}{\operatorname{ord}}
\def\pfrac#1#2{{\left(\frac{#1}{#2}\right)}}


\makeindex

\begin{document}




\thispagestyle{plain}
\begin{flushright}
\large{\textbf{TYPE YOUR NAME HERE \\
HW 2: 1.15 - 1.20, A.10, A.18\\
M328K \\
January 24th, 2012 \\}}
\end{flushright}

%\maketitle
\markboth{}{} \setcounter{section}{0} \baselineskip=18pt

\setcounter{tocdepth}{4}


%%%%%%%%This is where you can change the numbering to match the problem you
%%%%%%%%are on.  Set the section to  the current chapter.

\setcounter{section}{1}

%%%%%%%%%Now, set the theorem number to one less than the first theorem in
%%%%%%%%%the assignment.
\setcounter{thm}{14}

%%%%%%%%%%%%1.15%%%%%%%%%%%%%%%%%%%%%%%%%%%%%%%%%%%%%%%%%%%%%%%%%%%%%
\begin{exer}
Let $a$, $b$, and $n$ be integers with $n
> 0$.  Show that if $a \equiv b \pmod{n}$,
then $a^2 \equiv b^2 \pmod{n}$.
\end{exer}

\begin{proof}[Proof]
Type your proof here!
\end{proof}

%%%%%%%%%%%%1.16%%%%%%%%%%%%%%%%%%%%%%%%%%%%%%%%%%%%%%%%%%%%%%%%%%%%%
\begin{exer}
Let $a$, $b$, and $n$ be integers with $n
> 0$.  Show that if $a \equiv b \pmod{n}$,
then $a^3 \equiv b^3 \pmod{n}$.
\end{exer}

\begin{proof}[Proof]
Type your proof here!
\end{proof}

%%%%%%%%%%%%1.17%%%%%%%%%%%%%%%%%%%%%%%%%%%%%%%%%%%%%%%%%%%%%%%%%%%%
\begin{exer}
Let $a$, $b$, $k$, and $n$ be integers with $n
> 0$ and $k > 1$. Show that if $a \equiv b \pmod{n}$ and
$a^{k-1} \equiv b^{k-1} \pmod{n}$,
then \[a^k \equiv b^k \pmod{n}.\]
\end{exer}

\begin{proof}[Proof]
Type your proof here!
\end{proof}

%%%%%%%%%%%%1.18%%%%%%%%%%%%%%%%%%%%%%%%%%%%%%%%%%%%%%%%%%%%%%%%%%%%
\begin{thm}
Let $a$, $b$, $k$, and $n$ be integers with $n
> 0$ and $k > 0$.  If $a \equiv b \pmod{n}$,
then \[a^k \equiv b^k \pmod{n}.\]
\end{thm}

\begin{proof}[Proof]
Type your proof here!
\end{proof}

%%%%%%%%%%%%1.19%%%%%%%%%%%%%%%%%%%%%%%%%%%%%%%%%%%%%%%%%%%%%%%%%%%%

\begin{exer}
Illustrate each of Theorems $1.12$-$1.18$ with
an example using actual numbers.
\end{exer}

\begin{enumerate}

\item Example for 1.12:
\item Example for 1.13:
\item Example for 1.14:
\item Example for 1.15:
\item Example for 1.16:
\item Example for 1.17:
\item Example for 1.18:

\end{enumerate}

%%%%%%%%%%%%1.20%%%%%%%%%%%%%%%%%%%%%%%%%%%%%%%%%%%%%%%%%%%%%%%%%%%%%
\begin{ques}
Let $a$, $b$, $c$, and $n$ be integers for which $ac \equiv bc
\pmod{n}$. Can we conclude that $a \equiv b \pmod{n}$? If you answer
``yes", try and give a proof. If you answer ``no", try and give a
counterexample.
\end{ques}

\begin{proof}[Solution]
Type your solution here!
\end{proof}

%%% Notice that the theorems below don't get included in the numbering system
%%% because the "Thm" is capitalized. This is how it's defined in the beginning
%%% of the file.


%%%%%%%%%%%%A.10%%%%%%%%%%%%%%%%%%%%%%%%%%%%%%%%%%%%%%%%%%%%%%%%%%%%%

\noindent \textbf{A.10 Theorem.} \emph{Let n be a natural number. Then $1+2+3+\cdots + n=\frac{(n)(n+1)}{2}$}

\begin{proof}[Proof]
Type your proof here!
\end{proof}


%%%%%%%%%%%%A.18%%%%%%%%%%%%%%%%%%%%%%%%%%%%%%%%%%%%%%%%%%%%%%%%%%%%%
\noindent \textbf{A.18 Theorem.} \emph{For every natural number n,  $1+2+2^2+\cdots + 2^n=2^{n+1}-1$}

\begin{proof}[Proof]
Type your proof here!
\end{proof}



\end{document}
